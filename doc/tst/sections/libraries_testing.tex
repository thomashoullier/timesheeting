\section{Libraries testing}
\subsection{config\textunderscore lib}
\subsubsection{Utilities}
\paragraph{LT-CON-010 -- Tilde expansion nominal}
The tilde expansion function shall perform the following application,
\lstinline{~/.config/timesheeting/timesheeting.toml}
$\rightarrow$ \lstinline{<home>/timesheeting/timesheeting.toml},
with \lstinline{<home>} the current \lstinline{$HOME}.

\textit{Tests: } LR-CON-010 -- Tilde expansion utility

\paragraph{LT-CON-020 -- Tilde expansion none}
The tilde expansion function shall return an identical filepath to the
input \lstinline{/etc/timesheeting/timesheeting.toml}.

\textit{Tests: } LR-CON-010 -- Tilde expansion utility

\paragraph{LT-CON-030 -- First existing file found}
Given a directory structure as follows,
\dirtree{%
  .1 ...
  .2 dir1.
  .3 subdir1.
  .2 dir2.
  .3 timesheeting.toml.
  .2 dir3.
}
the \emph{first existing file utility}, when provided with the list of folders
\lstinline{dir1/subdir1, dir2, dir3} and the suffix
\lstinline{timesheeting.toml}
must return the path \lstinline{dir2/timesheeting.toml}.

\textit{Tests: } LR-CON-020 -- First existing file

\paragraph{LT-CON-040 -- First existing file not found}
Given a directory structure as follows,
\dirtree{%
  .1 ...
  .2 dir1.
  .3 subdir1.
  .2 dir2.
  .3 timesheeting.toml.
  .2 dir3.
}
the \emph{first existing file utility}, when provided with the list of folders
\lstinline{dir1/subdir1, dir1, dir3} and the suffix
\lstinline{timesheeting.toml} must return a nullopt.

\textit{Tests: } LR-CON-020 -- First existing file

\subsubsection{Configuration loader}
\begin{minipage}{\linewidth}
  \begin{lstlisting}[caption={TOML test configuration file.},
                     label={lst:toml_file}]
[par1]
string_arg = "Hello there."
empty_string_arg = ""

[par2]
  [par2.sub1]
  filepath_arg = "~/"
  filepath_nonexistent_arg = "/dev/null/nonexistent/nonexistent"
  [par2.sub2]
  float_arg = 8.3
  [par2.sub3]
  vector_strings_arg = [ "hello", "there", "fox", "delta" ]
  vector_nonstrings_arg = [ 31, 1, 43 ]
\end{lstlisting} \end{minipage}

\begin{minipage}{\linewidth}
  \begin{lstlisting}[caption={TOML test configuration file.},
                     label={lst:toml_file_empty}]
[par1]
string_arg = "Hello there."
empty_arg = 
\end{lstlisting} \end{minipage}

\paragraph{LT-CON-050 -- Configuration file loading}
The configuration loader shall successfully initialize given the filepath
for \cref{lst:toml_file}.

\textit{Tests: } LR-CON-030 -- Configuration file format

\paragraph{LT-CON-060 -- Configuration file non-existent}
The configuration loader shall emit an exception when provided the
filepath \lstinline{/dev/null/nonexistent}.

\textit{Tests: } LR-CON-120 -- Configuration file nonexistent

\paragraph{LT-CON-070 -- String parameter reading}
Given \cref{lst:toml_file}, the configuration loader shall read the
\lstinline{par1.string_arg} node as a \emph{string} and return
\lstinline{"Hello there."}.

\textit{Tests: } LR-CON-040 -- String loading

\paragraph{LT-CON-080 -- String parameter empty}
Given \cref{lst:toml_file}, the configuration loader shall read the
\lstinline{par1.empty_string_arg} node as a \emph{string} and emit an exception.

\textit{Tests: } LR-CON-050 -- String empty case

\paragraph{LT-CON-090 -- Filepath reading}
Given \cref{lst:toml_file}, the configuration loader shall read the
\lstinline{par2.sub1.filepath_arg} node as a \emph{filepath}
and return \lstinline{$HOME}, set to the current \lstinline{HOME} environment
variable.

\textit{Tests: } LR-CON-060 -- Filepath loading

\paragraph{LT-CON-100 -- Filepath non-existent}
Given \cref{lst:toml_file}, the configuration loader shall read the
\lstinline{par2.sub1.filepath_nonexistent_arg} node as a \emph{filepath}
and emit an exception.

\textit{Tests: } LR-CON-070 -- Filepath parent non-existent

\paragraph{LT-CON-110 -- Float reading}
Given \cref{lst:toml_file}, the configuration loader shall read the
\lstinline{par2.sub2.float_arg} node as a \emph{float} and return
\lstinline{8.3}.

\textit{Tests: } LR-CON-080 -- Float loading

\paragraph{LT-CON-120 -- Empty parameter case}
Given \cref{lst:toml_file_empty}, the configuration loader shall emit an
exception upon loading the configuration file.

\textit{Tests: } LR-CON-090 -- Parameter empty case

\paragraph{LT-CON-130 -- Vector of strings reading}
Given \cref{lst:toml_file}, the configuration loader shall read the
\lstinline{par2.sub3.vector_strings_arg} node as a \emph{vector of strings}
and return the vector \lstinline{"hello", "there", "fox", "delta"} in order.

\textit{Tests: } LR-CON-100 -- Vector of strings loading

\paragraph{LT-CON-140 -- Vector of non-strings case}
Given \cref{lst:toml_file}, the configuration loader shall read the
\lstinline{par2.sub3.vector_nonstrings_arg} node as a \emph{vector of strings}
and emit an exception.

\textit{Tests: } LR-CON-110 -- Vector of non-strings case

\subsection{time\textunderscore lib}
\subsubsection{time\textunderscore zone}
\paragraph{LT-TMZ-010 -- Invalid time zone}
The time\textunderscore zone object initialization shall be called with
the identifier string \lstinline{GOOFY} and emit an exception.

\textit{Tests: } LR-TMZ-030 -- Invalid time zone

\paragraph{LT-TMZ-020 -- Time zone initialization}
The time\textunderscore zone object shall be initialized with the TZ
identifier string \lstinline{Europe/Paris} without error.

\textit{Tests: } LR-TMZ-010 -- Time zone initialization

\paragraph{LT-TMZ-030 -- Time zone name}
The time\textunderscore zone singleton initialized in LT-TMZ-020 shall
be retrieved, its name queried and equal to \lstinline{Europe/Paris}.

\textit{Tests: } LR-TMZ-020 -- Time zone singleton,
                 LR-TMZ-040 -- Time zone name

\paragraph{LT-TMZ-040 -- std time\textunderscore zone}
The time\textunderscore zone singleton initialized in LT-TMZ-020 shall
be retrieved, its \lstinline{std::chrono::time_zone} object retrieved
and compared with
\lstinline{std::chrono::get_tzdb().locate_zone("Europe/Paris")}.


\textit{Tests: } LR-TMZ-020 -- Time zone singleton,
                 LR-TMZ-050 -- std time\textunderscore zone

\subsubsection{Date}
The tests for Date reuse the TimeZone singleton initialized
in LT-TMZ-020.
                 
\paragraph{LT-DAT-010 -- std::time\textunderscore point initialization}
The Date shall be initialized without error using the time point
\lstinline{std::chrono::system_clock::now()}.

\textit{Tests: } LR-DAT-020 \lstinline{std::time_point} initialization

\paragraph{LT-DAT-020 -- Date std::time\textunderscore point access}
The Date initialized in LT-DAT-010 shall be accessed through its
internal \lstinline{std::chrono::time_point} representation.
The duration between the initialization time point and the retrieved
time point shall be at most 1 second.

\textit{Tests: } LR-DAT-075 Date std::time\textunderscore point access,
LR-DAT-130 Second resolution

\paragraph{LT-DAT-030 -- Current time initialization}
The Date shall be initialized at the current time without error
and the duration between the internal \lstinline{std::chrono::time_point}
and \lstinline{std::chrono::system_clock::now()} shall be at most 1 second.

\textit{Tests: } LR-DAT-010 Current time initialization

\paragraph{LT-DAT-040 -- UNIX timestamp initialization}
The Date shall be initialized with the UNIX timestamp in seconds
\lstinline{1737283885} without error.

\textit{Tests: } LR-DAT-040 UNIX timestamp initialization

\paragraph{LT-DAT-050 -- UNIX timestamp output}
The Date initialized in LT-DAT-040 shall be converted to a UNIX
timestamp in seconds. The timestamp shall be be equal to \lstinline{1737283885}.

\textit{Tests: } LR-DAT-110 Date output UNIX timestamp

\paragraph{LT-DAT-060 -- Date string initialization}
The Date shall be initialized with the date string
\lstinline{19Jan2025 11:51:25} without error, and a UNIX timestamp output
will be queried, which must be equal to \lstinline{1737283885}.
This assumes the TimeZone singleton is set to \lstinline{Europe/Paris}.

\textit{Tests: } LR-DAT-050 Date string initialization

\paragraph{LT-DAT-070 -- Date string shortcuts 1}
The Date shall be initialized with the date string
\lstinline{19Jan2025 11:51} without error.
A UNIX timestamp output will be queried, which must be equal to
\lstinline{1737283860}.

\textit{Tests: } LR-DAT-060 Date string shortcuts

\paragraph{LT-DAT-080 -- Date string shortcuts 2}
The Date shall be initialized with the date string
\lstinline{19Jan2025 11} without error.
A UNIX timestamp output will be queried, which must be equal to
\lstinline{1737280800}.

\textit{Tests: } LR-DAT-060 Date string shortcuts

\paragraph{LT-DAT-090 -- Date string shortcuts 3}
The Date shall be initialized with the date string
\lstinline{19Jan2025} without error.
A UNIX timestamp output will be queried, which must be equal to
\lstinline{1737241200}.

\textit{Tests: } LR-DAT-060 Date string shortcuts

\paragraph{LT-DAT-100 -- Date string invalid}
The Date initialization shall emit an exception when provided
with the string \lstinline{GOOFY}.

\textit{Tests: } LR-DAT-070 Date string shortcuts

\paragraph{LT-DAT-110 -- Date output string}
The Date shall be initialized with a UNIX timestamp in seconds
\lstinline{1737283885}, the output string shall be queried and
must be equal to \lstinline{19Jan2025 11:51:25}.

\textit{Tests: } LR-DAT-080 Date output string

\paragraph{LT-DAT-120 -- Date output hours/minutes}
The Date shall be initialized with a UNIX timestamp in seconds
\lstinline{1737283885}, the output hours/minutes shall be queried
and must be equal to \lstinline{11:51}.

\textit{Tests: } LR-DAT-090 Date output hours/minutes

\paragraph{LT-DAT-130 -- Date output unambiguous string}
The Date shall be initialized with a UNIX timestamp in seconds
\lstinline{1737283885}, the output unambiguous string shall
be queried and must be equal to \lstinline{19Jan2025 11:51:25 +0100}.

\textit{Tests: } LR-DAT-100 Date output unambiguous string

\paragraph{LT-DAT-140 -- Date output day/month/year}
The Date shall be initialized with a UNIX timestamp in seconds
\lstinline{1737283885}, the output day/month/year string shall
be queried and must be equal to \lstinline{19Jan2025}.

\textit{Tests: } LR-DAT-120 Date output day/month/year

\paragraph{LT-DAT-150 -- Beginning of year}
The Date shall be initialized using the DatePoint YearBegin without error.
The day/month/year string is then queried, it must be equal to
\lstinline{01JanYYYY} with \lstinline{YYYY} whatever the current year
currently is in the time zone currently set in TimeZone.

\textit{Tests: } LR-DAT-030 Beginning of year

\paragraph{LT-DAT-160 -- Date comparison}
Two Date instances shall be initialized, the first with
UNIX timestamp in seconds \lstinline{1737283885}, the second with
\lstinline{1737283886}.
The comparison must show that,
\begin{itemize}
\item the first Date is \emph{lesser than} the second Date,
\item the second Date is \emph{greater than} the first Date.
\end{itemize}

\textit{Tests: } LR-DAT-140 Date comparison
