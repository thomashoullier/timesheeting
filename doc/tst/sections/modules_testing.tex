\section{Modules testing}
\subsection{version}
The version module is tested using automated unit tests.

\paragraph{MT-VER-010 -- Program version number}
We get the current program version number as a \emph{string}.
We check that its format is compatible with the following \emph{regular
  expression}: \lstinline{[0123456789]+\.[0123456789]+(?:dev)?}.

\textit{Tests: } MR-VER-010 -- Program version number

\paragraph{MT-VER-020 -- DB version number}
We get the current DB version as an integer. We test it is greater than zero.

\textit{Tests: } MR-VER-020 -- DB version number

\subsection{config}
The following automated unit tests are used for the config module.

\begin{minipage}{\linewidth}
  \begin{lstlisting}[caption={timesheeting nominal configuration file},
                     label={lst:config_file}]
[db]
file = "/tmp/timesheeting.db"

[time]
timezone = "Europe/Paris"
hours_per_workday = 7.7

[log]
file = "/tmp/timesheeting.log"
active_levels = [ "debug", "error", "info" ]
max_log_age = 604800

[keys]
  [keys.navigation]
  up = ["e", "UP"]
  down = ["n", "DOWN"]
  left = ["h", "LEFT"]
  right = ["i", "RIGHT"]
  subtabs = ["TAB"]
  previous = [","]
  next = ["."]
  duration_display = ["d"]
  entries_screen = ["1"]
  projects_screen = ["2"]
  locations_screen = ["3"]
  project_report_screen = ["4"]
  weekly_report_screen = ["5"]
  active_visibility = ["v"]
  quit = ["q"]
  [keys.actions]
  commit_entry = ["ENTER"]
  set_now = ["SPACE"]
  add = ["a"]
  rename = ["r"]
  remove = ["x"]
  active_toggle = ["b"]
  task_project_change = ["p"]
  [keys.edit_mode]
  validate = ["ENTER"]
  cancel = ["ESCAPE"]
  select_suggestion = ["TAB"]
\end{lstlisting} \end{minipage}

\begin{minipage}{\linewidth}
  \begin{lstlisting}[caption={timesheeting configuration file
                              with duplicate binding},
                     label={lst:config_file_duplicate}]
[db]
file = "/tmp/timesheeting.db"

[time]
timezone = "Europe/Paris"
hours_per_workday = 7.7

[log]
file = "/tmp/timesheeting.log"
active_levels = [ "debug", "error", "info" ]
max_log_age = 604800

[keys]
  [keys.navigation]
  up = ["e", "UP"]
  down = ["n", "DOWN"]
  left = ["h", "LEFT"]
  right = ["i", "RIGHT"]
  subtabs = ["TAB"]
  previous = [","]
  next = ["."]
  duration_display = ["d"]
  entries_screen = ["1"]
  projects_screen = ["2"]
  locations_screen = ["3"]
  project_report_screen = ["4"]
  weekly_report_screen = ["5"]
  active_visibility = ["v"]
  quit = ["q"]
  [keys.actions]
  commit_entry = ["ENTER"]
  set_now = ["SPACE"]
  add = ["a", "v"]
  rename = ["r"]
  remove = ["x"]
  active_toggle = ["b"]
  task_project_change = ["p"]
  [keys.edit_mode]
  validate = ["ENTER"]
  cancel = ["ESCAPE"]
  select_suggestion = ["TAB"]
\end{lstlisting} \end{minipage}

\begin{minipage}{\linewidth}
  \begin{lstlisting}[caption={timesheeting configuration file with invalid
                              special key},
                     label={lst:config_file_invalidstr}]
[db]
file = "/tmp/timesheeting.db"

[time]
timezone = "Europe/Paris"
hours_per_workday = 7.7

[log]
file = "/tmp/timesheeting.log"
active_levels = [ "debug", "error", "info" ]
max_log_age = 604800

[keys]
  [keys.navigation]
  up = ["e", "UP", "GOOFY"]
  down = ["n", "DOWN"]
  left = ["h", "LEFT"]
  right = ["i", "RIGHT"]
  subtabs = ["TAB"]
  previous = [","]
  next = ["."]
  duration_display = ["d"]
  entries_screen = ["1"]
  projects_screen = ["2"]
  locations_screen = ["3"]
  project_report_screen = ["4"]
  weekly_report_screen = ["5"]
  active_visibility = ["v"]
  quit = ["q"]
  [keys.actions]
  commit_entry = ["ENTER"]
  set_now = ["SPACE"]
  add = ["a"]
  rename = ["r"]
  remove = ["x"]
  active_toggle = ["b"]
  task_project_change = ["p"]
  [keys.edit_mode]
  validate = ["ENTER"]
  cancel = ["ESCAPE"]
  select_suggestion = ["TAB"]
\end{lstlisting} \end{minipage}

\paragraph{MT-CON-010 -- Config file loading}
Given the filepath to a file containing \cref{lst:config_file},
the \emph{config} module loader shall return the internal configuration
parameters representation without error.

\textit{Tests: } MR-CON-010 -- Load from path.

\paragraph{MT-CON-020 -- Config file non-existent}
Given the filepath \lstinline{/dev/null/nonexistent} as the path
to the configuration file, the \emph{config} module loader shall
emit an exception.

\textit{Tests: } MR-CON-020 -- Non-existent path.

\paragraph{MT-CON-030 -- Log filepath}
The configuration parameters loaded from \cref{lst:config_file}
shall contain a filepath \lstinline{log_filepath}.

\textit{Tests: } MR-CON-050 -- Log filepath

\paragraph{MT-CON-040 -- Log levels}
The configuration parameters loaded from \cref{lst:config_file}
shall contain a vector of strings \lstinline{log_levels_to_activate}.

\textit{Tests: } MR-CON-060 -- Log levels

\paragraph{MT-CON-050 -- DB filepath}
The configuration parameters loaded from \cref{lst:config_file}
shall contain a filepath \lstinline{db_filepath}.

\textit{Tests: } MR-CON-070 -- DB filepath

\paragraph{MT-CON-060 -- Timezone string}
The configuration parameters loaded from \cref{lst:config_file}
shall contain a string \lstinline{timezone}.

\textit{Tests: } MR-CON-080 -- Timezone string

\paragraph{MT-CON-070 -- Duration display days}
The configuration parameters loaded from \cref{lst:config_file}
shall contain a float \lstinline{hours_per_workday}.

\textit{Tests: } MR-CON-090 -- Duration display days

\paragraph{MT-CON-080 -- Bindings loading}
The configuration parameters at the node \lstinline{keys} loaded from
in \cref{lst:config_file} shall be tested for one to one mapping between
key and action.

\textit{Tests: }
\begin{itemize}
\item MR-CON-100 -- Binding list
\item MR-CON-110 -- Regular keys
\item MR-CON-120 -- Special keys
\item MR-CON-130 -- Multiple keys for an action
\end{itemize}

\paragraph{MT-CON-090 -- Binding duplicate}
Loading the configuration file \cref{lst:config_file_duplicate} shall
result in an exception being emitted.

\textit{Tests: } MR-CON-140 -- Protection against duplicates

\paragraph{MT-CON-100 -- Backspace mapping}
The configuration parameters loaded from \cref{lst:config_file}
shall include a binding for the backspace key to the \emph{backspace}
in the \emph{edit map}.

\textit{Tests: } MR-CON-150 -- Backspace mapping

\paragraph{MT-CON-110 -- Unbound mappings}
From the configuration parameters loaded from \cref{lst:config_file},
querying the action corresponding to key \lstinline{u} shall return
\lstinline{unbound} for both the \emph{normal map} and the \emph{edit map}.

\textit{Tests: } MR-CON-160 -- Unbound keys

\paragraph{MT-CON-120 -- Invalid special key}
Loading the configuration file \cref{lst:config_file_invalidstr} shall
result in an exception being emitted.

\textit{Tests: } MR-CON-170 -- Invalid key strings

\paragraph{MT-CON-130 -- Maximum log age}
The configuration parameters loaded from \cref{lst:config_file}
shall contain an unsigned integer \lstinline{max_log_age}.

\textit{Tests: } MR-CON-065 -- Maximum log age

\subsection{core}
\paragraph{MT-COR-010 -- Project generic item}
A generic item of type \lstinline{Project} shall be instantiated without
error with id 4, name \lstinline{project1}, and active flag set to true.
The id, name and active flag shall be retrieved and be found equal to
their input values.

\textit{Tests: } MR-COR-010 -- Generic item

\paragraph{MT-COR-020 -- Task generic item}
A generic item of type \lstinline{Task} shall be instantiated without
error with id 2, name \lstinline{task1}, and active flag set to false.
The id, name and active flag shall be retrieved and be found equal to
their input values.

\textit{Tests: } MR-COR-010 -- Generic item

\paragraph{MT-COR-030 -- Location generic item}
A generic item of type \lstinline{Location} shall be instantiated without
error with id 3, name \lstinline{location1}, and active flag set to true.
The id, name and active flag shall be retrieved and be found equal to
their input values.

\textit{Tests: } MR-COR-010 -- Generic item

\paragraph{MT-COR-040 -- Vector of generic item names}
A vector of 3 generic items of type \lstinline{Location} is created,
with the following id, name and activity flags:
\lstinline{1, location1, true}, \lstinline{2, location2, true},
\lstinline{3, location3, true}.
It shall be converted to a vector of the item names.
The result vector must be equal to \lstinline{location1, location2, location3}.

\textit{Tests: } MR-COR-020 -- Vector of generic item names

\paragraph{MT-COR-050 -- Entry object}
An Entry with id 5, project name \lstinline{project1}, task name
\lstinline{task1}, start date instantiated from UNIX timestamp
\lstinline{1745837288}, stop date from UNIX timestamp
\lstinline{1745838288} and location name \lstinline{location1}
shall be instantiated without error.
Each attribute shall be queried and found equal to the input.
The dates are compared using a conversion to UNIX timestamp.

\textit{Tests: } MR-COR-030 -- Entry

\paragraph{MT-COR-060 -- Entry to strings}
The entry from MT-COR-050 shall be converted to a vector of strings.
This vector must be equal to
\lstinline{project1, task1, start date display string in long format, stop date display string in long format, location1},
with the date strings equal to what is returned by the method for long
display format.

\textit{Tests: } MR-COR-040 -- Entry to strings

\paragraph{MT-COR-070 -- Entry to short strings}
The entry from MT-COR-060 shall be converted to a vector of short strings.
The result shall be equal to that in MT-COR-070, except the date display
strings must be in short format.

\textit{Tests: } MR-COR-050 -- Entry to short strings

\paragraph{MT-COR-080 -- Entry staging with values}
An entry staging object with project name \lstinline{project1},
task name \lstinline{task1}, start date instantiated from UNIX timestamp
\lstinline{1745837288}, stop date from UNIX timestamp
\lstinline{1745838288} and location name \lstinline{location1}
shall be instantiated without error.
Each attribute shall be queried and found equal to the input.
The dates are compared using a conversion to UNIX timestamp.

\textit{Tests: } MR-COR-060 -- Entry staging

\paragraph{MT-COR-090 -- Entry staging without values}
An entry staging with every attribute as a nullopt shall be instantiated
without error.

\textit{Tests: } MR-COR-070 -- Entry staging

\paragraph{MT-COR-100 -- Entry staging with values to strings}
The entry staging object from MT-COR-080 shall be converted to
a vector of string representations. The resulting vector shall be
found equal to the inputs, with dates in long display format.

\textit{Tests: } MR-COR-070 -- Entry staging to strings

\paragraph{MT-COR-110 -- Entry staging without values to strings}
The entry staging object from MT-COR-090 shall be converted to
a vector of string representations. The result vector shall
contain 5 strings with a single whitespace.

\textit{Tests: } MR-COR-070 -- Entry staging to strings

\paragraph{MT-COR-120 -- Entry staging with values to short strings}
The entry staging object from MT-COR-080 shall be converted to
a vector of short string representations. The resulting vector shall be
found equal to the inputs, with dates in short display format.

\textit{Tests: } MR-COR-080 -- Entry staging to short strings

\paragraph{MT-COR-130 -- Entry staging without values to short strings}
The entry staging object from MT-COR-090 shall be converted to
a vector of short string representations. The result vector shall
contain 5 strings with a single whitespace.

\textit{Tests: } MR-COR-080 -- Entry staging to short strings

\paragraph{MT-COR-140 -- Export row}
An export row shall be instantiated without error with the following
attributes: entry id 4, project id 15, project name \lstinline{project15},
task id 1, task name \lstinline{task33}, location id 4, location name
\lstinline{location44}, start date initialized to UNIX timestamp
\lstinline{1745837288}, stop date initialized to UNIX timestamp
\lstinline{1745838288}.
Each attribute will be queried and be found equal to its input value.

\textit{Tests: } MR-COR-090 -- Export row

\paragraph{MT-COR-150 -- Export row csv string}
The export row object from MT-COR-140 shall be converted to csv string
format. This string shall be equal to
\lstinline{4, 15, project15, 1, task33, 4, location44, 1745837288, 1745838288}.

\textit{Tests: } MR-COR-100 -- Export csv string

\paragraph{MT-COR-160 -- Project total instantiation}
The project total object shown as example in MR-COR-120 shall
be instantiated without error.

\textit{Tests: } MR-COR-110 Project total

\paragraph{MT-COR-170 -- Project total to menu items}
The project total object shown as example in MR-COR-120 shall
be converted to menu items without error.
The number of menu items shall be 8.
Their cell string and display strings shall be found equal
to either the project and task names, or the duration in
and string format. The face of the first two items shall
be bold, the face of the 6 next items shall be normal.

\textit{Tests: } MR-COR-120 Project total to menu items.

\paragraph{MT-COR-180 -- Weekly totals instantiation}
The weekly totals shown as example in MR-COR-140 shall be
instantiated without error.

\textit{Tests: } MR-COR-130 Weekly totals.

\paragraph{MT-COR-190 -- Weekly totals to menu items}
The weekly totals shown as example in MR-COR-140 shall be converted to menu
items without error. The number of menu items shall be 72.

\textit{Tests: } MR-COR-140 Weekly totals to menu items.

\subsection{db}
The \gls{DB} being a singleton, care is taken to make each test
executable while sharing a single DB. The tests must nonetheless
be run in sequence.

\paragraph{MT-DBI-010 -- DB singleton grab}
The DB singleton shall be initialized to a valid temporary file
and grabbed without error.

\textit{Tests: } MR-DBI-010 DB loading.

\paragraph{MT-DBI-020 -- DB get user version}
The DB user version shall be retrieved and found equal to the version
given by the version module.

\textit{Tests: } MR-DBI-020 DB get user version.

\paragraph{MT-DBI-030 -- Projects addition, querying and deletion}
A project named \lstinline{project_MT-DBI-030} shall be added without
error, returning true.
The list of projects shall then be queried, and only one
project returned, with its name equal to the input project name.
Finally, the id of the project shall be retrieved, and used to
delete the project from the \gls{DB} without error, returning true.
The projects list shall be queried again and found empty.

\textit{Tests: } MR-DBI-150, MR-DBI-280, MR-DBI-340.

\paragraph{MT-DBI-040 -- Locations addition, querying and deletion}
A location named \lstinline{location_MT-DBI-040} shall be added without
error, returning true.
The list of locations shall then be queried, and only one
location returned, with its name equal to the input location name.
Finally, the id of the location shall be retrieved, and used to
delete the location from the \gls{DB} without error, returning true.
The locations list shall be queried again and found empty.

\textit{Tests: } MR-DBI-190, MR-DBI-320, MR-DBI-360.

\paragraph{MT-DBI-050 -- Tasks addition, querying and deletion}
A parent project named \lstinline{project_MT-DBI-050} shall be
added.
A task named \lstinline{task_MT-DBI-050} shall be added without
error, returning true.
The list of tasks for the parent project shall then be queried, and only one
task returned, with its name equal to the input task name.
Finally, the id of the task shall be retrieved, and used to
delete the task from the \gls{DB} without error, returning true.
The tasks list shall be queried again and found empty.
The parent project is finally deleted.

\textit{Tests: } MR-DBI-170, MR-DBI-300, MR-DBI-350.

\paragraph{MT-DBI-060 -- Unique project names}
A project named \lstinline{project_MT-DBI-060} shall be added.
The same project name shall be tried for addition, the operation
must return false.
The list of projects shall be queried and found to contain only
one element.
Finally, the added project shall be deleted.

\textit{Tests: } MR-DBI-040 Unique project names.

\paragraph{MT-DBI-070 -- Unique location names}
A location named \lstinline{location_MT-DBI-070} shall be added.
The same location name shall be tried for addition, the
operation must return false.
The list of locations shall be queried and found to contain only
one element.
Finally, the added location shall be deleted.

\textit{Tests: } MR-DBI-090 Unique location names.

\paragraph{MT-DBI-080 -- Unique task names}
Two parent projects named \lstinline{project1_MT-DBI-080}
and \lstinline{project2_MT-DBI-080} shall be added.
A task named \lstinline{task_MT-DBI-080} shall be added
to \lstinline{project1_MT-DBI-080}.
The same task name shall be retried for addition on the same
project, the operation must return false.
The list of tasks for this project shall be queried and found
to contain only one element.
Next, the same task name shall be used for addition
to the other project \lstinline{project2_MT-DBI-080}
without error, the operation must return true.
Finally, both tasks shall be deleted, followed by the deletion
of both parent projects.

\textit{Tests: } MR-DBI-060 Unique task names per project.

\paragraph{MT-DBI-090 -- Project's tasks deletion}
A parent project named \lstinline{project_MT-DBI-090}
shall be added.
A task named \lstinline{task_MT-DBI-090} shall be added to it.
The parent project shall be deleted without error.
The list of tasks for the former project id shall be queried
and found empty, and idem for the list of projects.

\textit{Tests: } MR-DBI-070 Project's task deletion.

\paragraph{MT-DBI-100 -- Projects active default}
A project named \lstinline{project_MT-DBI-100} shall be added.
The list of projects shall be queried, the project retrieved.
The project active flag must be true.
Finally the project is removed.

\textit{Tests: } MR-DBI-160 Projects active default

\paragraph{MT-DBI-110 -- Locations active default}
A location named \lstinline{location_MT-DBI-110} shall be added.
The list of locations shall be queried, the location retrieved.
The location active flag must be true.
Finally the location is removed.

\textit{Tests: } MR-DBI-200 Locations active default

\paragraph{MT-DBI-120 -- Tasks active default}
A parent project named \lstinline{project_MT-DBI-120} shall
be added. A task named \lstinline{task_MT-DBI-120} shall be added
to it.
The list of tasks for the project shall be queried and the only
task retrieved. The task active flag must be true.
Finally, the task and project are removed.

\textit{Tests: } MR-DBI-180 Tasks active default.

\paragraph{MT-DBI-130 -- Query projects ordering}
Two projects, named \lstinline{projectB_MT-DBI-130}
and \lstinline{projectA_MT-DBI-130} shall be added
(purposefully in reverse alphabetical order).
The list of projects shall be queried, the first
project name shall be equal to the A project,
the second project name shall be equal to the B project.
Finally, both projects shall be deleted.

\textit{Tests: } MR-DBI-280 Query projects.

\paragraph{MT-DBI-140 -- Query locations ordering}
Two locations, named \lstinline{locationB_MT-DBI-130}
and \lstinline{locationA_MT-DBI-130} shall be added
(purposefully in reverse alphabetical order).
The list of locations shall be queried, the first
location name shall be equal to the A location,
the second location name shall be equal to the B location.
Finally, both locations shall be deleted.

\textit{Tests: } MR-DBI-320 Query locations.

\paragraph{MT-DBI-150 -- Query tasks ordering}
A parent project named \lstinline{project_MT-DBI-150}
shall be added.
Two tasks named \lstinline{taskB_MT-DBI-150} and \lstinline{taskA_MT-DBI-150}
shall be added (purposefully in reverse alphabetical order) to it.
The list of tasks for the parent project shall be queried.
The first task name shall be equal to the A task, the second task name
shall be equal to the B task.
Finally both tasks and the project shall be deleted.

\textit{Tests: } MR-DBI-300 Query tasks of project.

\paragraph{MT-DBI-160 -- Toggle project active}
A project named \lstinline{project_MT-DBI-160} shall be added.
The project list shall be queried and the project retrieved.
Its active flag must be equal to true.
Then, the project's active flag shall be toggled without error.
The project shall be re-queried and its active flag found to be false.
The project's active flag shall be toggled again without error.
The project shall be re-queried and its active flag found to be true.
Finally, the project shall be deleted.

\textit{Tests: } MR-DBI-250 Toggle project active.

\paragraph{MT-DBI-170 -- Toggle location active}
A location named \lstinline{location_MT-DBI-170} shall be added.
The location list shall be queried and the location retrieved.
Its active flag must be equal to true.
Then, the location's active flag shall be toggled without error.
The location shall be re-queried and its active flag found to be false.
The location's active flag shall be toggled again without error.
The location shall be re-queried and its active flag found to be true.
Finally, the location shall be deleted.

\textit{Tests: } MR-DBI-270 Toggle location active.

\paragraph{MT-DBI-180 -- Toggle task active}
A parent project named \lstinline{project_MT-DBI-180} shall be added.
A task named \lstinline{task_MT-DBI-180} shall be added to it.
The task list for the parent project shall be queried, the task retrieved.
The task active flag must be equal to true.
The task active flag shall be toggled without error.
The task shall be re-queried and its active flag found to be false.
The task active flag shall be toggled without error.
The task shall be re-queried and its active flag found to be true.
Finally, the task and project shall be deleted.

\textit{Tests: } MR-DBI-260 Toggle task active.

\paragraph{MT-DBI-190 -- Query active projects}
Two projects, named \lstinline{project_active_MT-DBI-190}
and \lstinline{project_inactive_MT-DBI-190} shall be added.
The second project active flag shall be toggled.
The list of active projects shall be queried.
The list shall be found to contain only one item.
This item shall have a name equal to the input above.
Finally the two projects shall be removed.

\textit{Tests: } MR-DBI-290 Query active projects.

\paragraph{MT-DBI-200 -- Query active locations}
Two locations, named \lstinline{location_active_MT-DBI-190}
and \lstinline{location_inactive_MT-DBI-190} shall be added.
The second location active flag shall be toggled.
The list of active locations shall be queried.
The list shall be found to contain only one item.
This item shall have a name equal to the input above.
Finally the two locations shall be removed.

\textit{Tests: } MR-DBI-330 Query active locations.

\paragraph{MT-DBI-210 -- Query active tasks}
A parent project named \lstinline{project_MT-DBI-210}
shall be added.
Two tasks named \lstinline{task_active_MT-DBI-210}
and \lstinline{task_inactive_MT-DBI-210} shall be added
to it.
The second task's active flag shall be toggled.
The lsit of tasks shall be queried. The list shall be found to
contain only one item. This item shall have a name equal
to the input above.
Finally, the tasks and project shall be removed.

\textit{Tests: } MR-DBI-310 Query active tasks of project.

\paragraph{MT-DBI-220 -- Query active projects ordering}
Two projects, named \lstinline{projectB_MT-DBI-220}
and \lstinline{projectA_MT-DBI-220} shall be added
(purposefully in reverse alphabetical order).
The list of active projects shall be queried, the first
project name shall be equal to the A project,
the second project name shall be equal to the B project.
Finally, both projects shall be deleted.

\textit{Tests: } MR-DBI-290 Query active projects.

\paragraph{MT-DBI-230 -- Query active locations ordering}
Two locations, named \lstinline{locationB_MT-DBI-230}
and \lstinline{locationA_MT-DBI-230} shall be added
(purposefully in reverse alphabetical order).
The list of active locations shall be queried, the first
location name shall be equal to the A location,
the second location name shall be equal to the B location.
Finally, both locations shall be deleted.

\textit{Tests: } MR-DBI-330 Query active locations.

\paragraph{MT-DBI-240 -- Query active tasks ordering}
A parent project named \lstinline{project_MT-DBI-240} shall be added.
Two tasks named \lstinline{taskB_MT-DBI-240} and \lstinline{taskA_MT-DBI-240}
shall be added (purposefully in reverse alphabetical order) to it.
The list of active tasks for the parent project shall be queried.
The first task name shall be equal to the A task, the second task name
shall be equal to the B task.
Finally both tasks and the project shall be deleted.

\textit{Tests: } MR-DBI-310 Query active tasks of project.

\paragraph{MT-DBI-250 -- Project name edit}
A project named \lstinline{project_MT-DBI-250} shall be added.
Then, the name of this project shall be edited to \lstinline{new_MT-DBI-250}
without error, returning true.
The list of projects shall be queried. The list must contain only one item.
This item's name must be equal to the new name.
Finally, the project shall be deleted.

\textit{Tests: } MR-DBI-210 Project name edit

\paragraph{MT-DBI-260 -- Project name edit unique}
Two projects, named \lstinline{projectA_MT-DBI-260} and
\lstinline{projectB_MT-DBI-260} shall be added.
The renaming of projectA to projectB shall be tried, the operation
returning false.
The list of projects shall be queried. The list must contain two items.
Each project name must be as inputted.
Finally, both projects shall be deleted.

\textit{Tests: } MR-DBI-210 Project name edit, MR-DBI-040 Unique project names.

\paragraph{MT-DBI-270 -- Location name edit}
A location named \lstinline{location_MT-DBI-270} shall be added.
Then, the name of this location shall be edited to \lstinline{new_MT-DBI-270}
without error, returning true.
The list of locations shall be queried. The list must contain only one item.
This item's name must be equal to the new name.
Finally, the location shall be deleted.

\textit{Tests: } MR-DBI-240 Location name edit

\paragraph{MT-DBI-280 -- Location name edit unique}
Two locations, named \lstinline{locationA_MT-DBI-280} and
\lstinline{locationB_MT-DBI-280} shall be added.
The renaming of locationA to locationB shall be tried, the operation
returning false.
The list of locations shall be queried. The list must contain two items.
Each location name must be as inputted.
Finally, both locations shall be deleted.

\textit{Tests: } MR-DBI-240 Location name edit, MR-DBI-090 Unique location
names.

\paragraph{MT-DBI-290 -- Task name edit}
A parent project named \lstinline{project_MT-DBI-290} shall be added.
A task named \lstinline{task_MT-DBI-290} shall be added.
Then, the name of this task shall be edited to \lstinline{new_MT-DBI-290}
without error, returning true.
The list of tasks shall be queried. The list must contain only one item.
This item's name must be equal to the new name.
Finally, the task and project shall be deleted.

\textit{Tests: } MR-DBI-220 Task name edit.

\paragraph{MT-DBI-300 -- Task name edit unique}
A parent project named \lstinline{project_MT-DBI-300} shall be added.
Two tasks, named \lstinline{taskA_MT-DBI-300} and \lstinline{taskB_MT-DBI-300}
shall be added.
The renaming of taskA to taskB shall be tried, the operation returning false.
The list of tasks shall be queried. The list must contain two items.
Each task name must be as inputted.
Finally, both tasks and the parent project shall be deleted.

\textit{Tests: } MR-DBI-220 Task name edit, MR-DBI-060 Unique task names per
project.

\paragraph{MT-DBI-310 -- Task project edit}
Two parent projects, named \lstinline{projectA_MT-DBI-310} and
\lstinline{projectB_MT-DBI-310} shall be added.
A task named \lstinline{task_MT-DBI-310} shall be added to projectA.
The task shall then be edited to be affixed to projectB instead,
without error, the operation returning true.
The tasks of projectA shall be queried, and found to be empty.
The tasks of projectB shall be queried, one item shall be found,
its name shall be equal to the input task name.
Finally, the task and the two projects shall be removed.

\textit{Tests: } MR-DBI-230 Task project edit.

\paragraph{MT-DBI-320 -- Task project edit unique}
Two parent projects, named \lstinline{projectA_MT-DBI-320} and
\lstinline{projectB_MT-DBI-320} shall be added.
A task shall be added in each, each named the same, \lstinline{task_MT-DBI-320}.
A move of the task from projectA to projectB shall be tried, the operation
returning false.
The tasks for each project shall be queried, one item shall be found in each.
Finally, the task and the two projects shall be removed.

\textit{Tests: } MR-DBI-230 Task project edit, MR-DBI-060 Unique task names
per project.

\paragraph{MT-DBI-330 -- Task project edit nonexistent}
Two parent projects, named \lstinline{projectA_MT-DBI-330} and
\lstinline{projectB_MT-DBI-330} shall be added.
A task named \lstinline{task_MT-DBI-330} shall be added to projectA.
The move of the task from projectA to a project named
\lstinline{nonexistent_project} shall be tried, the operation returning false.
The list of tasks of projectA shall be queried, and found to contain
one item with name equal to the input.
Finally, the task and the two projects shall be removed.

\textit{Tests: } MR-DBI-230 Task project edit.

\paragraph{MT-DBI-340 -- Entry staging empty at first}
The initial entry staging shall be queried.
Every field is checked and must be empty.

\textit{Tests: } MR-DBI-430 Entry staging query.

\paragraph{MT-DBI-350 -- Entries empty at first}
The initial entries shall be queried in the date range
from UNIX UTC timestamp \lstinline{1745817889}
to UNIX UTC timestamp \lstinline{1746249889}. The result must
be empty.

\textit{Tests: } MR-DBI-500 Entries query

\paragraph{MT-DBI-360 -- Entry staging project name}
A project named \lstinline{project_MT-DBI-360} shall be added.
A task named \lstinline{task_MT-DBI-360} shall be added to it.
The entry staging project name shall be set to the above project name.
The entry staging shall be queried, the project name field shall
be found to contain a value equal to the project name.
The project shall not be removed, as it will be reused by subsequent tests.

\textit{Tests: } MR-DBI-370 Entry staging project name.

\paragraph{MT-DBI-370 -- Entry staging task auto-fill}
The entry staging shall be queried. The task field shall be found
to contain a value equal to the task name of the previous test.

\textit{Tests: } MR-DBI-370 Entry staging project name.

\paragraph{MT-DBI-380 -- Entry staging non-existent project name}
The entry staging project name shall be edited to
\lstinline{nonexistent_project}, the operation returning true.
The entry staging shall be queried, the project and task fields
must be empty.

\textit{Tests: } MR-DBI-370 Entry staging project name.

\paragraph{MT-DBI-390 -- Entry staging active projects}
A project named \lstinline{project_MT-DBI-390} shall be added.
A task named \lstinline{task_MT-DBI-390} shall be added to it.
The project shall be set to inactive.
The entrystaging project shall be set to this project name,
returning true.
The entrystaging shall be queried, the project and task names
must not hold values.
Finally, the project is deleted from the DB.

\textit{Tests: } MR-DBI-370 Entry staging project name.

\paragraph{MT-DBI-400 -- Entry staging task name without project}
The entry staging project name shall be checked, it must be empty.
The entry staging task name shall be set to the existing
\lstinline{task_MT-DBI-360}, the operation returning true.
The entry staging shall be queried, the project and task names
must be empty.

\textit{Tests: } MR-DBI-380 Entry staging task name.

\paragraph{MT-DBI-410 -- Entry staging task name}
A second task named \lstinline{task_MT-DBI-410}
shall be added to the existing project \lstinline{project_MT-DBI-360}.
The entry staging project name shall be set to this project.
Then, the entry staging task name shall be set to the new task
without error, the operation returning true.
The entry staging shall be queried, and the project and task
names must be found equal to the ones mentioned above.

\textit{Tests: } MR-DBI-380 Entry staging task name.

\paragraph{MT-DBI-420 -- Entry staging task name active}
A third task named \lstinline{task_MT-DBI-420}
shall be added to the existing project \lstinline{project_MT-DBI-360}.
This task shall be set to inactive.
The entry staging project name shall be set to the above project.
The entry staging task name shall be set to the new inactive task,
the operation returning true.
The entry staging shall be queried, the project and task names
must be found empty.

\textit{Tests: } MR-DBI-380 Entry staging task name.

\paragraph{MT-DBI-430 -- Entry staging task name non-existent}
The entry staging project name shall be set to the existing
project \lstinline{project_MT-DBI-360}.
The entry staging task name shall be set to the non-existent task
name \lstinline{nonexistent_task}, the operation returning true.
The entry staging shall be queried, the project and task names
must be empty.

\textit{Tests: } MR-DBI-380 Entry staging task name.

\paragraph{MT-DBI-440 -- Entry staging location name}
A location named \lstinline{location_MT-DBI-440} shall be added.
The entry staging location name shall be set to this location
name without error, the operation returning true.
The entry staging shall be queried, the location name must
hold a value and be equal to the input name.
The location shall be kept for subsequent tests.

\textit{Tests: } MR-DBI-410 Entry staging location name.

\paragraph{MT-DBI-450 -- Entry staging location name active}
A location named \lstinline{location_MT-DBI-450} shall
be added and set to inactive.
The entry staging location name shall be set to this location
name, the operation returning true.
The entry staging shall be queried, the location name must be empty.
The location shall be kept for subsequent tests.

\textit{Tests: } MR-DBI-410 Entry staging location name.

\paragraph{MT-DBI-460 -- Entry staging location name non-existent}
The entry staging location name shall be set to
the non-existent location name \lstinline{nonexistent_location},
the operation returning true.
The entry staging shall be queried, the location name must be empty.

\textit{Tests: } MR-DBI-410 Entry staging location name.

\paragraph{MT-DBI-470 -- Entry staging start date}
Given the start date initialized to UTC UNIX timestamp
\lstinline{1745818889}, the entry staging start date shall be
set to this date without error, the operation returning true.
The entry staging shall be queried, the start date must hold
a value equal to the input.

\textit{Tests: } MR-DBI-390 Entry staging start.

\paragraph{MT-DBI-480 -- Entry staging stop date}
Given the stop date initialized to UTC UNIX timestamp
\lstinline{1745819889}, the entry staging stop date shall be
set to this date without error, the operation returning true.
The entry staging shall be queried, the stop date must hold
a value equal to the input.

\textit{Tests: } MR-DBI-400 Entry staging stop.

\paragraph{MT-DBI-490 -- Entry staging project id}
TODO: maybe modify the requirement and implementation so that
an optional is returned here.

\paragraph{MT-DBI-500 -- Entry staging project id no project}

\paragraph{MT-DBI-510 -- Entry staging set project which has no tasks}
