\section{Modules testing}
\subsection{version}
The version module is tested using automated unit tests.

\paragraph{MT-VER-010 -- Program version number}
We get the current program version number as a \emph{string}.
We check that its format is compatible with the following \emph{regular
  expression}: \lstinline{[0123456789]+\.[0123456789]+(?:dev)?}.

\textit{Tests: } MR-VER-010 -- Program version number

\paragraph{MT-VER-020 -- DB version number}
We get the current DB version as an integer. We test it is greater than zero.

\textit{Tests: } MR-VER-020 -- DB version number

\subsection{config}
The following automated unit tests are used for the config module.

\begin{minipage}{\linewidth}
  \begin{lstlisting}[caption={timesheeting nominal configuration file},
                     label={lst:config_file}]
[db]
file = "/tmp/timesheeting.db"

[time]
timezone = "Europe/Paris"
hours_per_workday = 7.7

[log]
file = "/tmp/timesheeting.log"
active_levels = [ "debug", "error", "info" ]
max_log_age = 604800

[keys]
  [keys.navigation]
  up = ["e", "UP"]
  down = ["n", "DOWN"]
  left = ["h", "LEFT"]
  right = ["i", "RIGHT"]
  subtabs = ["TAB"]
  previous = [","]
  next = ["."]
  duration_display = ["d"]
  entries_screen = ["1"]
  projects_screen = ["2"]
  locations_screen = ["3"]
  project_report_screen = ["4"]
  weekly_report_screen = ["5"]
  active_visibility = ["v"]
  quit = ["q"]
  [keys.actions]
  commit_entry = ["ENTER"]
  set_now = ["SPACE"]
  add = ["a"]
  rename = ["r"]
  remove = ["x"]
  active_toggle = ["b"]
  task_project_change = ["p"]
  [keys.edit_mode]
  validate = ["ENTER"]
  cancel = ["ESCAPE"]
  select_suggestion = ["TAB"]
\end{lstlisting} \end{minipage}

\begin{minipage}{\linewidth}
  \begin{lstlisting}[caption={timesheeting configuration file
                              with duplicate binding},
                     label={lst:config_file_duplicate}]
[db]
file = "/tmp/timesheeting.db"

[time]
timezone = "Europe/Paris"
hours_per_workday = 7.7

[log]
file = "/tmp/timesheeting.log"
active_levels = [ "debug", "error", "info" ]
max_log_age = 604800

[keys]
  [keys.navigation]
  up = ["e", "UP"]
  down = ["n", "DOWN"]
  left = ["h", "LEFT"]
  right = ["i", "RIGHT"]
  subtabs = ["TAB"]
  previous = [","]
  next = ["."]
  duration_display = ["d"]
  entries_screen = ["1"]
  projects_screen = ["2"]
  locations_screen = ["3"]
  project_report_screen = ["4"]
  weekly_report_screen = ["5"]
  active_visibility = ["v"]
  quit = ["q"]
  [keys.actions]
  commit_entry = ["ENTER"]
  set_now = ["SPACE"]
  add = ["a", "v"]
  rename = ["r"]
  remove = ["x"]
  active_toggle = ["b"]
  task_project_change = ["p"]
  [keys.edit_mode]
  validate = ["ENTER"]
  cancel = ["ESCAPE"]
  select_suggestion = ["TAB"]
\end{lstlisting} \end{minipage}

\begin{minipage}{\linewidth}
  \begin{lstlisting}[caption={timesheeting configuration file with invalid
                              special key},
                     label={lst:config_file_invalidstr}]
[db]
file = "/tmp/timesheeting.db"

[time]
timezone = "Europe/Paris"
hours_per_workday = 7.7

[log]
file = "/tmp/timesheeting.log"
active_levels = [ "debug", "error", "info" ]
max_log_age = 604800

[keys]
  [keys.navigation]
  up = ["e", "UP", "GOOFY"]
  down = ["n", "DOWN"]
  left = ["h", "LEFT"]
  right = ["i", "RIGHT"]
  subtabs = ["TAB"]
  previous = [","]
  next = ["."]
  duration_display = ["d"]
  entries_screen = ["1"]
  projects_screen = ["2"]
  locations_screen = ["3"]
  project_report_screen = ["4"]
  weekly_report_screen = ["5"]
  active_visibility = ["v"]
  quit = ["q"]
  [keys.actions]
  commit_entry = ["ENTER"]
  set_now = ["SPACE"]
  add = ["a"]
  rename = ["r"]
  remove = ["x"]
  active_toggle = ["b"]
  task_project_change = ["p"]
  [keys.edit_mode]
  validate = ["ENTER"]
  cancel = ["ESCAPE"]
  select_suggestion = ["TAB"]
\end{lstlisting} \end{minipage}

\paragraph{MT-CON-010 -- Config file loading}
Given the filepath to a file containing \cref{lst:config_file},
the \emph{config} module loader shall return the internal configuration
parameters representation without error.

\textit{Tests: } MR-CON-010 -- Load from path.

\paragraph{MT-CON-020 -- Config file non-existent}
Given the filepath \lstinline{/dev/null/nonexistent} as the path
to the configuration file, the \emph{config} module loader shall
emit an exception.

\textit{Tests: } MR-CON-020 -- Non-existent path.

\paragraph{MT-CON-030 -- Log filepath}
The configuration parameters loaded from \cref{lst:config_file}
shall contain a filepath \lstinline{log_filepath}.

\textit{Tests: } MR-CON-050 -- Log filepath

\paragraph{MT-CON-040 -- Log levels}
The configuration parameters loaded from \cref{lst:config_file}
shall contain a vector of strings \lstinline{log_levels_to_activate}.

\textit{Tests: } MR-CON-060 -- Log levels

\paragraph{MT-CON-050 -- DB filepath}
The configuration parameters loaded from \cref{lst:config_file}
shall contain a filepath \lstinline{db_filepath}.

\textit{Tests: } MR-CON-070 -- DB filepath

\paragraph{MT-CON-060 -- Timezone string}
The configuration parameters loaded from \cref{lst:config_file}
shall contain a string \lstinline{timezone}.

\textit{Tests: } MR-CON-080 -- Timezone string

\paragraph{MT-CON-070 -- Duration display days}
The configuration parameters loaded from \cref{lst:config_file}
shall contain a float \lstinline{hours_per_workday}.

\textit{Tests: } MR-CON-090 -- Duration display days

\paragraph{MT-CON-080 -- Bindings loading}
The configuration parameters at the node \lstinline{keys} loaded from
in \cref{lst:config_file} shall be tested for one to one mapping between
key and action.

\textit{Tests: }
\begin{itemize}
\item MR-CON-100 -- Binding list
\item MR-CON-110 -- Regular keys
\item MR-CON-120 -- Special keys
\item MR-CON-130 -- Multiple keys for an action
\end{itemize}

\paragraph{MT-CON-090 -- Binding duplicate}
Loading the configuration file \cref{lst:config_file_duplicate} shall
result in an exception being emitted.

\textit{Tests: } MR-CON-140 -- Protection against duplicates

\paragraph{MT-CON-100 -- Backspace mapping}
The configuration parameters loaded from \cref{lst:config_file}
shall include a binding for the backspace key to the \emph{backspace}
in the \emph{edit map}.

\textit{Tests: } MR-CON-150 -- Backspace mapping

\paragraph{MT-CON-110 -- Unbound mappings}
From the configuration parameters loaded from \cref{lst:config_file},
querying the action corresponding to key \lstinline{u} shall return
\lstinline{unbound} for both the \emph{normal map} and the \emph{edit map}.

\textit{Tests: } MR-CON-160 -- Unbound keys

\paragraph{MT-CON-120 -- Invalid special key}
Loading the configuration file \cref{lst:config_file_invalidstr} shall
result in an exception being emitted.

\textit{Tests: } MR-CON-170 -- Invalid key strings

\paragraph{MT-CON-130 -- Maximum log age}
The configuration parameters loaded from \cref{lst:config_file}
shall contain an integer \lstinline{max_log_age}.

\textit{Tests: } MR-CON-065 -- Maximum log age

\subsection{core}
\paragraph{MT-COR-010 -- Project generic item}
A generic item of type \lstinline{Project} shall be instantiated without
error with id 4, name \lstinline{project1}, and active flag set to true.
The id, name and active flag shall be retrieved and be found equal to
their input values.

\textit{Tests: } MR-COR-010 -- Generic item

\paragraph{MT-COR-020 -- Task generic item}
A generic item of type \lstinline{Task} shall be instantiated without
error with id 2, name \lstinline{task1}, and active flag set to false.
The id, name and active flag shall be retrieved and be found equal to
their input values.

\textit{Tests: } MR-COR-010 -- Generic item

\paragraph{MT-COR-030 -- Location generic item}
A generic item of type \lstinline{Location} shall be instantiated without
error with id 3, name \lstinline{location1}, and active flag set to true.
The id, name and active flag shall be retrieved and be found equal to
their input values.

\textit{Tests: } MR-COR-010 -- Generic item

\paragraph{MT-COR-040 -- Vector of generic item names}
A vector of 3 generic items of type \lstinline{Location} is created,
with the following id, name and activity flags:
\lstinline{1, location1, true}, \lstinline{2, location2, true},
\lstinline{3, location3, true}.
It shall be converted to a vector of the item names.
The result vector must be equal to \lstinline{location1, location2, location3}.

\textit{Tests: } MR-COR-020 -- Vector of generic item names

\paragraph{MT-COR-050 -- Entry object}
An Entry with id 5, project name \lstinline{project1}, task name
\lstinline{task1}, start date instantiated from UNIX timestamp
\lstinline{1745837288}, stop date from UNIX timestamp
\lstinline{1745838288} and location name \lstinline{location1}
shall be instantiated without error.
Each attribute shall be queried and found equal to the input.
The dates are compared using a conversion to UNIX timestamp.

\textit{Tests: } MR-COR-030 -- Entry

\paragraph{MT-COR-060 -- Entry to strings}
The entry from MT-COR-050 shall be converted to a vector of strings.
This vector must be equal to
\lstinline{project1, task1, start date display string in long format, stop date display string in long format, location1},
with the date strings equal to what is returned by the method for long
display format.

\textit{Tests: } MR-COR-040 -- Entry to strings

\paragraph{MT-COR-070 -- Entry to short strings}
The entry from MT-COR-060 shall be converted to a vector of short strings.
The result shall be equal to that in MT-COR-070, except the date display
strings must be in short format.

\textit{Tests: } MR-COR-050 -- Entry to short strings

\paragraph{MT-COR-080 -- Entry staging with values}
An entry staging object with project name \lstinline{project1},
task name \lstinline{task1}, start date instantiated from UNIX timestamp
\lstinline{1745837288}, stop date from UNIX timestamp
\lstinline{1745838288} and location name \lstinline{location1}
shall be instantiated without error.
Each attribute shall be queried and found equal to the input.
The dates are compared using a conversion to UNIX timestamp.

\textit{Tests: } MR-COR-060 -- Entry staging

\paragraph{MT-COR-090 -- Entry staging without values}
An entry staging with every attribute as a nullopt shall be instantiated
without error.

\textit{Tests: } MR-COR-060 -- Entry staging.

\paragraph{MT-COR-100 -- Entry staging with values to strings}
The entry staging object from MT-COR-080 shall be converted to
a vector of string representations. The resulting vector shall be
found equal to the inputs, with dates in long display format.

\textit{Tests: } MR-COR-070 -- Entry staging to strings

\paragraph{MT-COR-110 -- Entry staging without values to strings}
The entry staging object from MT-COR-090 shall be converted to
a vector of string representations. The result vector shall
contain 5 strings with a single whitespace.

\textit{Tests: } MR-COR-070 -- Entry staging to strings

\paragraph{MT-COR-120 -- Entry staging with values to short strings}
The entry staging object from MT-COR-080 shall be converted to
a vector of short string representations. The resulting vector shall be
found equal to the inputs, with dates in short display format.

\textit{Tests: } MR-COR-080 -- Entry staging to short strings

\paragraph{MT-COR-130 -- Entry staging without values to short strings}
The entry staging object from MT-COR-090 shall be converted to
a vector of short string representations. The result vector shall
contain 5 strings with a single whitespace.

\textit{Tests: } MR-COR-080 -- Entry staging to short strings

\paragraph{MT-COR-140 -- Export row}
An export row shall be instantiated without error with the following
attributes: entry id 4, project id 15, project name \lstinline{project15},
task id 1, task name \lstinline{task33}, location id 4, location name
\lstinline{location44}, start date initialized to UNIX timestamp
\lstinline{1745837288}, stop date initialized to UNIX timestamp
\lstinline{1745838288}.
Each attribute will be queried and be found equal to its input value.

\textit{Tests: } MR-COR-090 -- Export row

\paragraph{MT-COR-150 -- Export row csv string}
The export row object from MT-COR-140 shall be converted to csv string
format. This string shall be equal to
\lstinline{4, 15, project15, 1, task33, 4, location44, 1745837288, 1745838288}.

\textit{Tests: } MR-COR-100 -- Export csv string

\paragraph{MT-COR-160 -- Project total instantiation}
The project total object shown as example in MR-COR-120 shall
be instantiated without error.

\textit{Tests: } MR-COR-110 Project total

\paragraph{MT-COR-170 -- Project total to menu items}
The project total object shown as example in MR-COR-120 shall
be converted to menu items without error.
The number of menu items shall be 8.
Their cell string and display strings shall be found equal
to either the project and task names, or the duration in
and string format. The face of the first two items shall
be bold, the face of the 6 next items shall be normal.

\textit{Tests: } MR-COR-120 Project total to menu items.

\paragraph{MT-COR-180 -- Weekly totals instantiation}
The weekly totals shown as example in MR-COR-140 shall be
instantiated without error.

\textit{Tests: } MR-COR-130 Weekly totals.

\paragraph{MT-COR-190 -- Weekly totals to menu items}
The weekly totals shown as example in MR-COR-140 shall be converted to menu
items without error. The number of menu items shall be 72.

\textit{Tests: } MR-COR-140 Weekly totals to menu items.

\subsection{db}
The \gls{DB} being a singleton, care is taken to make each test
executable while sharing a single DB. The tests must nonetheless
be run in sequence.

\paragraph{MT-DBI-010 -- DB singleton grab}
The DB singleton shall be initialized to a valid temporary file
and grabbed without error.

\textit{Tests: } MR-DBI-010 DB loading.

\paragraph{MT-DBI-020 -- DB get user version}
The DB user version shall be retrieved and found equal to the version
given by the version module.

\textit{Tests: } MR-DBI-020 DB get user version.

\paragraph{MT-DBI-030 -- Projects addition, querying and deletion}
A project named \lstinline{project_MT-DBI-030} shall be added without
error, returning true.
The list of projects shall then be queried, and only one
project returned, with its name equal to the input project name.
Finally, the id of the project shall be retrieved, and used to
delete the project from the \gls{DB} without error, returning true.
The projects list shall be queried again and found empty.

\textit{Tests: } MR-DBI-150, MR-DBI-280, MR-DBI-340.

\paragraph{MT-DBI-040 -- Locations addition, querying and deletion}
A location named \lstinline{location_MT-DBI-040} shall be added without
error, returning true.
The list of locations shall then be queried, and only one
location returned, with its name equal to the input location name.
Finally, the id of the location shall be retrieved, and used to
delete the location from the \gls{DB} without error, returning true.
The locations list shall be queried again and found empty.

\textit{Tests: } MR-DBI-190, MR-DBI-320, MR-DBI-360.

\paragraph{MT-DBI-050 -- Tasks addition, querying and deletion}
A parent project named \lstinline{project_MT-DBI-050} shall be
added.
A task named \lstinline{task_MT-DBI-050} shall be added without
error, returning true.
The list of tasks for the parent project shall then be queried, and only one
task returned, with its name equal to the input task name.
Finally, the id of the task shall be retrieved, and used to
delete the task from the \gls{DB} without error, returning true.
The tasks list shall be queried again and found empty.
The parent project is finally deleted.

\textit{Tests: } MR-DBI-170, MR-DBI-300, MR-DBI-350.

\paragraph{MT-DBI-060 -- Unique project names}
A project named \lstinline{project_MT-DBI-060} shall be added.
The same project name shall be tried for addition, the operation
must return false.
The list of projects shall be queried and found to contain only
one element.
Finally, the added project shall be deleted.

\textit{Tests: } MR-DBI-040 Unique project names.

\paragraph{MT-DBI-070 -- Unique location names}
A location named \lstinline{location_MT-DBI-070} shall be added.
The same location name shall be tried for addition, the
operation must return false.
The list of locations shall be queried and found to contain only
one element.
Finally, the added location shall be deleted.

\textit{Tests: } MR-DBI-090 Unique location names.

\paragraph{MT-DBI-080 -- Unique task names}
Two parent projects named \lstinline{project1_MT-DBI-080}
and \lstinline{project2_MT-DBI-080} shall be added.
A task named \lstinline{task_MT-DBI-080} shall be added
to \lstinline{project1_MT-DBI-080}.
The same task name shall be retried for addition on the same
project, the operation must return false.
The list of tasks for this project shall be queried and found
to contain only one element.
Next, the same task name shall be used for addition
to the other project \lstinline{project2_MT-DBI-080}
without error, the operation must return true.
Finally, both tasks shall be deleted, followed by the deletion
of both parent projects.

\textit{Tests: } MR-DBI-060 Unique task names per project.

\paragraph{MT-DBI-090 -- Project's tasks deletion}
A parent project named \lstinline{project_MT-DBI-090}
shall be added.
A task named \lstinline{task_MT-DBI-090} shall be added to it.
The parent project shall be deleted without error.
The list of tasks for the former project id shall be queried
and found empty, and idem for the list of projects.

\textit{Tests: } MR-DBI-070 Project's task deletion.

\paragraph{MT-DBI-100 -- Projects active default}
A project named \lstinline{project_MT-DBI-100} shall be added.
The list of projects shall be queried, the project retrieved.
The project active flag must be true.
Finally the project is removed.

\textit{Tests: } MR-DBI-160 Projects active default

\paragraph{MT-DBI-110 -- Locations active default}
A location named \lstinline{location_MT-DBI-110} shall be added.
The list of locations shall be queried, the location retrieved.
The location active flag must be true.
Finally the location is removed.

\textit{Tests: } MR-DBI-200 Locations active default

\paragraph{MT-DBI-120 -- Tasks active default}
A parent project named \lstinline{project_MT-DBI-120} shall
be added. A task named \lstinline{task_MT-DBI-120} shall be added
to it.
The list of tasks for the project shall be queried and the only
task retrieved. The task active flag must be true.
Finally, the task and project are removed.

\textit{Tests: } MR-DBI-180 Tasks active default.

\paragraph{MT-DBI-130 -- Query projects ordering}
Two projects, named \lstinline{projectB_MT-DBI-130}
and \lstinline{projectA_MT-DBI-130} shall be added
(purposefully in reverse alphabetical order).
The list of projects shall be queried, the first
project name shall be equal to the A project,
the second project name shall be equal to the B project.
Finally, both projects shall be deleted.

\textit{Tests: } MR-DBI-280 Query projects.

\paragraph{MT-DBI-140 -- Query locations ordering}
Two locations, named \lstinline{locationB_MT-DBI-130}
and \lstinline{locationA_MT-DBI-130} shall be added
(purposefully in reverse alphabetical order).
The list of locations shall be queried, the first
location name shall be equal to the A location,
the second location name shall be equal to the B location.
Finally, both locations shall be deleted.

\textit{Tests: } MR-DBI-320 Query locations.

\paragraph{MT-DBI-150 -- Query tasks ordering}
A parent project named \lstinline{project_MT-DBI-150}
shall be added.
Two tasks named \lstinline{taskB_MT-DBI-150} and \lstinline{taskA_MT-DBI-150}
shall be added (purposefully in reverse alphabetical order) to it.
The list of tasks for the parent project shall be queried.
The first task name shall be equal to the A task, the second task name
shall be equal to the B task.
Finally both tasks and the project shall be deleted.

\textit{Tests: } MR-DBI-300 Query tasks of project.

\paragraph{MT-DBI-160 -- Toggle project active}
A project named \lstinline{project_MT-DBI-160} shall be added.
The project list shall be queried and the project retrieved.
Its active flag must be equal to true.
Then, the project's active flag shall be toggled without error.
The project shall be re-queried and its active flag found to be false.
The project's active flag shall be toggled again without error.
The project shall be re-queried and its active flag found to be true.
Finally, the project shall be deleted.

\textit{Tests: } MR-DBI-250 Toggle project active.

\paragraph{MT-DBI-170 -- Toggle location active}
A location named \lstinline{location_MT-DBI-170} shall be added.
The location list shall be queried and the location retrieved.
Its active flag must be equal to true.
Then, the location's active flag shall be toggled without error.
The location shall be re-queried and its active flag found to be false.
The location's active flag shall be toggled again without error.
The location shall be re-queried and its active flag found to be true.
Finally, the location shall be deleted.

\textit{Tests: } MR-DBI-270 Toggle location active.

\paragraph{MT-DBI-180 -- Toggle task active}
A parent project named \lstinline{project_MT-DBI-180} shall be added.
A task named \lstinline{task_MT-DBI-180} shall be added to it.
The task list for the parent project shall be queried, the task retrieved.
The task active flag must be equal to true.
The task active flag shall be toggled without error.
The task shall be re-queried and its active flag found to be false.
The task active flag shall be toggled without error.
The task shall be re-queried and its active flag found to be true.
Finally, the task and project shall be deleted.

\textit{Tests: } MR-DBI-260 Toggle task active.

\paragraph{MT-DBI-190 -- Query active projects}
Two projects, named \lstinline{project_active_MT-DBI-190}
and \lstinline{project_inactive_MT-DBI-190} shall be added.
The second project active flag shall be toggled.
The list of active projects shall be queried.
The list shall be found to contain only one item.
This item shall have a name equal to the input above.
Finally the two projects shall be removed.

\textit{Tests: } MR-DBI-290 Query active projects.

\paragraph{MT-DBI-200 -- Query active locations}
Two locations, named \lstinline{location_active_MT-DBI-190}
and \lstinline{location_inactive_MT-DBI-190} shall be added.
The second location active flag shall be toggled.
The list of active locations shall be queried.
The list shall be found to contain only one item.
This item shall have a name equal to the input above.
Finally the two locations shall be removed.

\textit{Tests: } MR-DBI-330 Query active locations.

\paragraph{MT-DBI-210 -- Query active tasks}
A parent project named \lstinline{project_MT-DBI-210}
shall be added.
Two tasks named \lstinline{task_active_MT-DBI-210}
and \lstinline{task_inactive_MT-DBI-210} shall be added
to it.
The second task's active flag shall be toggled.
The lsit of tasks shall be queried. The list shall be found to
contain only one item. This item shall have a name equal
to the input above.
Finally, the tasks and project shall be removed.

\textit{Tests: } MR-DBI-310 Query active tasks of project.

\paragraph{MT-DBI-220 -- Query active projects ordering}
Two projects, named \lstinline{projectB_MT-DBI-220}
and \lstinline{projectA_MT-DBI-220} shall be added
(purposefully in reverse alphabetical order).
The list of active projects shall be queried, the first
project name shall be equal to the A project,
the second project name shall be equal to the B project.
Finally, both projects shall be deleted.

\textit{Tests: } MR-DBI-290 Query active projects.

\paragraph{MT-DBI-230 -- Query active locations ordering}
Two locations, named \lstinline{locationB_MT-DBI-230}
and \lstinline{locationA_MT-DBI-230} shall be added
(purposefully in reverse alphabetical order).
The list of active locations shall be queried, the first
location name shall be equal to the A location,
the second location name shall be equal to the B location.
Finally, both locations shall be deleted.

\textit{Tests: } MR-DBI-330 Query active locations.

\paragraph{MT-DBI-240 -- Query active tasks ordering}
A parent project named \lstinline{project_MT-DBI-240} shall be added.
Two tasks named \lstinline{taskB_MT-DBI-240} and \lstinline{taskA_MT-DBI-240}
shall be added (purposefully in reverse alphabetical order) to it.
The list of active tasks for the parent project shall be queried.
The first task name shall be equal to the A task, the second task name
shall be equal to the B task.
Finally both tasks and the project shall be deleted.

\textit{Tests: } MR-DBI-310 Query active tasks of project.

\paragraph{MT-DBI-250 -- Project name edit}
A project named \lstinline{project_MT-DBI-250} shall be added.
Then, the name of this project shall be edited to \lstinline{new_MT-DBI-250}
without error, returning true.
The list of projects shall be queried. The list must contain only one item.
This item's name must be equal to the new name.
Finally, the project shall be deleted.

\textit{Tests: } MR-DBI-210 Project name edit

\paragraph{MT-DBI-260 -- Project name edit unique}
Two projects, named \lstinline{projectA_MT-DBI-260} and
\lstinline{projectB_MT-DBI-260} shall be added.
The renaming of projectA to projectB shall be tried, the operation
returning false.
The list of projects shall be queried. The list must contain two items.
Each project name must be as inputted.
Finally, both projects shall be deleted.

\textit{Tests: } MR-DBI-210 Project name edit, MR-DBI-040 Unique project names.

\paragraph{MT-DBI-270 -- Location name edit}
A location named \lstinline{location_MT-DBI-270} shall be added.
Then, the name of this location shall be edited to \lstinline{new_MT-DBI-270}
without error, returning true.
The list of locations shall be queried. The list must contain only one item.
This item's name must be equal to the new name.
Finally, the location shall be deleted.

\textit{Tests: } MR-DBI-240 Location name edit

\paragraph{MT-DBI-280 -- Location name edit unique}
Two locations, named \lstinline{locationA_MT-DBI-280} and
\lstinline{locationB_MT-DBI-280} shall be added.
The renaming of locationA to locationB shall be tried, the operation
returning false.
The list of locations shall be queried. The list must contain two items.
Each location name must be as inputted.
Finally, both locations shall be deleted.

\textit{Tests: } MR-DBI-240 Location name edit, MR-DBI-090 Unique location
names.

\paragraph{MT-DBI-290 -- Task name edit}
A parent project named \lstinline{project_MT-DBI-290} shall be added.
A task named \lstinline{task_MT-DBI-290} shall be added.
Then, the name of this task shall be edited to \lstinline{new_MT-DBI-290}
without error, returning true.
The list of tasks shall be queried. The list must contain only one item.
This item's name must be equal to the new name.
Finally, the task and project shall be deleted.

\textit{Tests: } MR-DBI-220 Task name edit.

\paragraph{MT-DBI-300 -- Task name edit unique}
A parent project named \lstinline{project_MT-DBI-300} shall be added.
Two tasks, named \lstinline{taskA_MT-DBI-300} and \lstinline{taskB_MT-DBI-300}
shall be added.
The renaming of taskA to taskB shall be tried, the operation returning false.
The list of tasks shall be queried. The list must contain two items.
Each task name must be as inputted.
Finally, both tasks and the parent project shall be deleted.

\textit{Tests: } MR-DBI-220 Task name edit, MR-DBI-060 Unique task names per
project.

\paragraph{MT-DBI-310 -- Task project edit}
Two parent projects, named \lstinline{projectA_MT-DBI-310} and
\lstinline{projectB_MT-DBI-310} shall be added.
A task named \lstinline{task_MT-DBI-310} shall be added to projectA.
The task shall then be edited to be affixed to projectB instead,
without error, the operation returning true.
The tasks of projectA shall be queried, and found to be empty.
The tasks of projectB shall be queried, one item shall be found,
its name shall be equal to the input task name.
Finally, the task and the two projects shall be removed.

\textit{Tests: } MR-DBI-230 Task project edit.

\paragraph{MT-DBI-320 -- Task project edit unique}
Two parent projects, named \lstinline{projectA_MT-DBI-320} and
\lstinline{projectB_MT-DBI-320} shall be added.
A task shall be added in each, each named the same, \lstinline{task_MT-DBI-320}.
A move of the task from projectA to projectB shall be tried, the operation
returning false.
The tasks for each project shall be queried, one item shall be found in each.
Finally, the task and the two projects shall be removed.

\textit{Tests: } MR-DBI-230 Task project edit, MR-DBI-060 Unique task names
per project.

\paragraph{MT-DBI-330 -- Task project edit nonexistent}
Two parent projects, named \lstinline{projectA_MT-DBI-330} and
\lstinline{projectB_MT-DBI-330} shall be added.
A task named \lstinline{task_MT-DBI-330} shall be added to projectA.
The move of the task from projectA to a project named
\lstinline{nonexistent_project} shall be tried, the operation returning false.
The list of tasks of projectA shall be queried, and found to contain
one item with name equal to the input.
Finally, the task and the two projects shall be removed.

\textit{Tests: } MR-DBI-230 Task project edit.

\paragraph{MT-DBI-340 -- Entry staging empty at first}
The initial entry staging shall be queried.
Every field is checked and must be empty.

\textit{Tests: } MR-DBI-430 Entry staging query.

\paragraph{MT-DBI-350 -- Entries empty at first}
The initial entries shall be queried in the date range
from UNIX UTC timestamp \lstinline{1745817889}
to UNIX UTC timestamp \lstinline{1746249889}. The result must
be empty.

\textit{Tests: } MR-DBI-500 Entries query

\paragraph{MT-DBI-360 -- Entry staging project name}
A project named \lstinline{project_MT-DBI-360} shall be added.
A task named \lstinline{task_MT-DBI-360} shall be added to it.
The entry staging project name shall be set to the above project name.
The entry staging shall be queried, the project name field shall
be found to contain a value equal to the project name.
The project shall not be removed, as it will be reused by subsequent tests.

\textit{Tests: } MR-DBI-370 Entry staging project name.

\paragraph{MT-DBI-370 -- Entry staging task auto-fill}
The entry staging shall be queried. The task field shall be found
to contain a value equal to the task name of the previous test.

\textit{Tests: } MR-DBI-370 Entry staging project name.

\paragraph{MT-DBI-380 -- Entry staging non-existent project name}
The entry staging project name shall be edited to
\lstinline{nonexistent_project}, the operation returning true.
The entry staging shall be queried, the project and task fields
must be empty.

\textit{Tests: } MR-DBI-370 Entry staging project name.

\paragraph{MT-DBI-390 -- Entry staging active projects}
A project named \lstinline{project_MT-DBI-390} shall be added.
A task named \lstinline{task_MT-DBI-390} shall be added to it.
The project shall be set to inactive.
The entrystaging project shall be set to this project name,
returning true.
The entrystaging shall be queried, the project and task names
must not hold values.
Finally, the project is deleted from the DB.

\textit{Tests: } MR-DBI-370 Entry staging project name.

\paragraph{MT-DBI-400 -- Entry staging task name without project}
The entry staging project name shall be checked, it must be empty.
The entry staging task name shall be set to the existing
\lstinline{task_MT-DBI-360}, the operation returning true.
The entry staging shall be queried, the project and task names
must be empty.

\textit{Tests: } MR-DBI-380 Entry staging task name.

\paragraph{MT-DBI-410 -- Entry staging task name}
A second task named \lstinline{task_MT-DBI-410}
shall be added to the existing project \lstinline{project_MT-DBI-360}.
The entry staging project name shall be set to this project.
Then, the entry staging task name shall be set to the new task
without error, the operation returning true.
The entry staging shall be queried, and the project and task
names must be found equal to the ones mentioned above.

\textit{Tests: } MR-DBI-380 Entry staging task name.

\paragraph{MT-DBI-420 -- Entry staging task name active}
A third task named \lstinline{task_MT-DBI-420}
shall be added to the existing project \lstinline{project_MT-DBI-360}.
This task shall be set to inactive.
The entry staging project name shall be set to the above project.
The entry staging task name shall be set to the new inactive task,
the operation returning true.
The entry staging shall be queried, the project and task names
must be found empty.

\textit{Tests: } MR-DBI-380 Entry staging task name.

\paragraph{MT-DBI-430 -- Entry staging task name non-existent}
The entry staging project name shall be set to the existing
project \lstinline{project_MT-DBI-360}.
The entry staging task name shall be set to the non-existent task
name \lstinline{nonexistent_task}, the operation returning true.
The entry staging shall be queried, the project and task names
must be empty.

\textit{Tests: } MR-DBI-380 Entry staging task name.

\paragraph{MT-DBI-440 -- Entry staging location name}
A location named \lstinline{location_MT-DBI-440} shall be added.
The entry staging location name shall be set to this location
name without error, the operation returning true.
The entry staging shall be queried, the location name must
hold a value and be equal to the input name.
The location shall be kept for subsequent tests.

\textit{Tests: } MR-DBI-410 Entry staging location name.

\paragraph{MT-DBI-450 -- Entry staging location name active}
A location named \lstinline{location_MT-DBI-450} shall
be added and set to inactive.
The entry staging location name shall be set to this location
name, the operation returning true.
The entry staging shall be queried, the location name must be empty.
The location shall be kept for subsequent tests.

\textit{Tests: } MR-DBI-410 Entry staging location name.

\paragraph{MT-DBI-460 -- Entry staging location name non-existent}
The entry staging location name shall be set to
the non-existent location name \lstinline{nonexistent_location},
the operation returning true.
The entry staging shall be queried, the location name must be empty.

\textit{Tests: } MR-DBI-410 Entry staging location name.

\paragraph{MT-DBI-470 -- Entry staging start date}
Given the start date initialized to UTC UNIX timestamp
\lstinline{1745818889}, the entry staging start date shall be
set to this date without error, the operation returning true.
The entry staging shall be queried, the start date must hold
a value equal to the input.

\textit{Tests: } MR-DBI-390 Entry staging start.

\paragraph{MT-DBI-480 -- Entry staging stop date}
Given the stop date initialized to UTC UNIX timestamp
\lstinline{1745819889}, the entry staging stop date shall be
set to this date without error, the operation returning true.
The entry staging shall be queried, the stop date must hold
a value equal to the input.

\textit{Tests: } MR-DBI-400 Entry staging stop.

\paragraph{MT-DBI-490 -- Entry staging project id}
The existing project named \lstinline{project_MT-DBI-360} shall
be set in the entry staging.
The entry staging project id shall be queried.
For comparison, the list of projects in the DB shall be queried
separately and the id for the relevant project found.
The two id must match.

\textit{Tests: } MR-DBI-440 Entry staging project id.

\paragraph{MT-DBI-500 -- Entry staging project id no project}
The entry staging project and task fields shall be emptied
by trying to set the project name to \lstinline{nonexistent_project}.
The entry staging shall be queried and the project and task fields found
to be empty.
The entry staging project id shall be queried. The id value must be
empty.

\textit{Tests: } MR-DBI-440 Entry staging project id.

\paragraph{MT-DBI-510 -- Entry staging set project which has no tasks}
A project named \lstinline{project_MT-DBI-510} shall be added.
The entry staging project name shall be set to this name, the operation
returning true.
The entry staging shall be queried, the project and task fields must
be empty.

\textit{Tests: } MR-DBI-370 Entry staging project name.

\paragraph{MT-DBI-520 -- Committing wrong date order}
The entry staging project shall be set to the existing project named
\lstinline{project_MT-DBI-360}. The entry staging task shall be
set to the existing task named \lstinline{task_MT-DBI-410}.
The entry staging location shall be set to the existing
location named \lstinline{location_MT-DBI-440}.
The entry staging start date shall be set to UNIX UTC timestamp
\lstinline{1745819989} and the stop date to
\lstinline{1745819889}, which is in the wrong order.
The entry staging commit shall be tried, the operation returning
false.
The entries over the date range from UNIX UTC timestamp \lstinline{1745817889}
to UNIX UTC timestamp \lstinline{1746249889} shall be queried. The result must
be empty.

\textit{Tests: } MR-DBI-120 Entries start stop ordering.

\paragraph{MT-DBI-530 -- Committing zero duration entry}
Keeping the entry staging settings from MT-DBI-520,
the entry staging stop date shall be set to the same value as
the entry staging start date.
The entry staging commit shall be tried, the operation returning false.
The entries over the date range from UNIX UTC timestamp \lstinline{1745817889}
to UNIX UTC timestamp \lstinline{1746249889} shall be queried. The result must
be empty.

\textit{Tests: } MR-DBI-120 Entries start stop ordering.

\paragraph{MT-DBI-540 -- Committing missing task}
Keeping the previous settings,
the entry staging start date shall be set to UNIX UTC timestamp
\lstinline{1745819889} and the stop date to
\lstinline{1745819989}, which is in the right order.
The entry staging task shall be emptied by setting it to
the name \lstinline{nonexistent_task}.
The entry staging commit shall be tried, the operation returning false.
The entries over the date range from UNIX UTC timestamp \lstinline{1745817889}
to UNIX UTC timestamp \lstinline{1746249889} shall be queried. The result must
be empty. Finally, the entry staging project and task are set back to the valid
\lstinline{project_MT-DBI-360} and \lstinline{task_MT-DBI-410}.

\textit{Tests: } MR-DBI-420 Entry staging commit.

\paragraph{MT-DBI-550 -- Committing missing location}
Keeping the previous settings,
the entry staging location shall be emptied by setting it to the name
\lstinline{nonexistent_location}.
The entry staging commit shall be tried, the operation returning false.
The entries over the date range from UNIX UTC timestamp \lstinline{1745817889}
to UNIX UTC timestamp \lstinline{1746249889} shall be queried. The result must
be empty. Finally, the entry staging location shall be set back to the valid
\lstinline{location_MT-DBI-440}.

\textit{Tests: } MR-DBI-420 Entry staging commit.

\paragraph{MT-DBI-560 -- Committing valid entry}
Keeping the previous valid parameters,
the entry staging commit shall be executed, the operation returning true.
The entries over the date range from UNIX UTC timestamp \lstinline{1745817889}
to UNIX UTC timestamp \lstinline{1746249889}. The result must contain
exactly one item.
The entry fields shall all be checked and must be found equal to the
values inputted in entry staging.

\textit{Tests: } MR-DBI-420 Entry staging commit.

\paragraph{MT-DBI-570 -- Overlapping entry with stop}
Keeping the previous parameters, the entry staging start and stop dates
shall be set to UTC UNIX timestamps \lstinline{1745819789} and
\lstinline{1745819890} respectively, meaning the entry starts before
the existing entry and its stop date impinges on the existing period.
The entry staging commit shall be executed, the operation returning false.
The entries over the date range from UNIX UTC timestamp \lstinline{1745817889}
to UNIX UTC timestamp \lstinline{1746249889} shall be queried. The result must
contain exactly one item.

\textit{Tests: } MR-DBI-130 Entries non-overlapping dates.

\paragraph{MT-DBI-580 -- Overlapping entry with start}
Keeping the previous parameters, the entry staging start and stop dates
shall be set to UTC UNIX timestamps \lstinline{1745819890} and
\lstinline{1745819990} respectively, meaning the entry starts after
the existing entry and its start date impinges on the existing period.
The entry staging commit shall be executed, the operation returning false.
The entries over the date range from UNIX UTC timestamp \lstinline{1745817889}
to UNIX UTC timestamp \lstinline{1746249889} shall be queried. The result must
contain exactly one item.

\textit{Tests: } MR-DBI-130 Entries non-overlapping dates.

\paragraph{MT-DBI-590 -- Overlapping entry, both dates inside}
Keeping the previous parameters, the entry staging start and stop dates shall be
set to UTC UNIX timestamps \lstinline{1745819890} and \lstinline{1745819895}
respectively, meaning the entry is contained inside the existing entry. The
entry staging commit shall be executed, the operation returning false. The
entries over the date range from UNIX UTC timestamp \lstinline{1745817889} to
UNIX UTC timestamp \lstinline{1746249889} shall be queried. The result must
contain exactly one item.

\textit{Tests: } MR-DBI-130 Entries non-overlapping dates.

\paragraph{MT-DBI-600 -- Overlapping entry, both dates on either side}
Keeping the previous parameters, the entry staging start and stop dates shall be
set to UTC UNIX timestamps \lstinline{1745819800} and \lstinline{1745819999}
respectively, meaning the existing entry is contained inside the candidate
entry. The entry staging commit shall be executed, the operation returning
false. The entries over the date range from UNIX UTC timestamp
\lstinline{1745817889} to UNIX UTC timestamp \lstinline{1746249889}
shall be queried. The result must contain exactly one item.

\textit{Tests: } MR-DBI-130 Entries non-overlapping dates.

\paragraph{MT-DBI-610 -- Valid second entry immediately following}
Keeping the previous parameters, the entry staging start and stop dates
shall be set to UTC UNIX timestamps \lstinline{1745819989} and
\lstinline{1745819999}. This new entry is placed just after the existing entry.
The entry staging commit shall be executed, the operation returning true.
The entries over the date range from UNIX UTC timestamp
\lstinline{1745817889} to UNIX UTC timestamp \lstinline{1746249889}
shall be queried. The result must contain exactly two items.

\textit{Tests: } MR-DBI-420 Entry staging commit.

\paragraph{MT-DBI-620 -- Entry project id query}
The id of the project currently held in entry staging shall be queried for
reference.
The entries over the date range from UNIX UTC timestamp
\lstinline{1745817889} to UNIX UTC timestamp \lstinline{1746249889}
shall be queried. The id of the first entry shall be used
to query the corresponding project id through the function under
test.
This id and the reference id taken from the entry staging must match.

\textit{Tests: } MR-DBI-510 Entry project id query.

\paragraph{MT-DBI-630 -- Entries duration over date range}
The duration of entries over the date range from UNIX UTC timestamp
\lstinline{1745817889} to UNIX UTC timestamp \lstinline{1746249889}
shall be queried without error.
The result must be equal to a Duration equivalent to 110 seconds
exactly.

\textit{Tests: } MR-DBI-530 Entries duration over date range.

\paragraph{MT-DBI-640 -- Entries duration over date range, zero duration}
The duration of entries over the date range from UNIX UTC timestamp
\lstinline{1735817889} to UNIX UTC timestamp \lstinline{1736249889}
shall be queried without error.
The result must be equal to a Duration equivalent to 0 seconds
exactly.

\textit{Tests: } MR-DBI-530 Entries duration over date range.

\paragraph{MT-DBI-645 -- Entries query ordering}
The entries over the date range from UNIX UTC timestamp
\lstinline{1745817889} to UNIX UTC timestamp \lstinline{1746249889}
shall be queried. The result must contain exactly two items.
The start dates of the first and second entries shall be compared,
the first returned entry must have a start date anterior to
the start date of the second returned entry.

\textit{Tests: } MR-DBI-500 Entries query.

\paragraph{MT-DBI-650 -- Delete entry}
The entries over the date range from UNIX UTC timestamp
\lstinline{1745817889} to UNIX UTC timestamp \lstinline{1746249889}
shall be queried. The id of the last entry in the list shall
be obtained. This entry shall be deleted without error, the operation
returning true. The list of entries over the above date range
shall be queried again, the result must contain exactly one item.
Finally, the entry is added back.

\textit{Tests: } MR-DBI-520 Delete entry.

\paragraph{MT-DBI-660 -- Entry project edit}
The entries over the date range from UNIX UTC timestamp
\lstinline{1745817889} to UNIX UTC timestamp \lstinline{1746249889}
shall be queried. The first entry is taken.
Its project shall be edited to the existing project
named \lstinline{project_MT-DBI-390}, without error, the operation
returning true.
The entries shall be re-queried, and the project name found
equal to the input.

\textit{Tests: } MR-DBI-450 Entry project edit.

\paragraph{MT-DBI-665 -- Entry project edit auto-fill}
Given the newly modified entry from the above test MT-DBI-660,
the task name shall be checked and found equal to
\lstinline{task_MT-DBI-390}.

\textit{Tests: } MR-DBI-450 Entry project edit.

\paragraph{MT-DBI-670 -- Entry project edit no tasks}
The entries over the date range from UNIX UTC timestamp
\lstinline{1745817889} to UNIX UTC timestamp \lstinline{1746249889}
shall be queried. The first entry is taken.
Its project shall be edited to the existing project
named \lstinline{project_MT-DBI-510}, which has no tasks,
the operation returning false.
The entries shall be re-queried, and the project name shall be
found equal to what it was at the beginning of the test.

\textit{Tests: } MR-DBI-450 Entry project edit.

\paragraph{MT-DBI-680 -- Entry project edit non-existent}
The entries over the date range from UNIX UTC timestamp
\lstinline{1745817889} to UNIX UTC timestamp \lstinline{1746249889}
shall be queried. The first entry is taken.
Its project shall be edited to the project name \lstinline{nonexistent_project},
the operation returning false.
The entries shall be re-queried, and the project name shall be
found equal to what it was at the beginning of the test.

\textit{Tests: } MR-DBI-450 Entry project edit.

\paragraph{MT-DBI-690 -- Entry task edit}
The entries over the date range from UNIX UTC timestamp
\lstinline{1745817889} to UNIX UTC timestamp \lstinline{1746249889}
shall be queried. The first entry is taken.
The project name shall be edited to the existing project name
\lstinline{project_MT-DBI-360}.
The task name shall then be edited to \lstinline{task_MT-DBI-410}
without error, the operation returning true.
The entries shall be re-queried, and the task name shall be found
equal to the input.

\textit{Tests: } MR-DBI-460 Entry task edit.

\paragraph{MT-DBI-700 -- Entry task edit non-existent}
The entries over the date range from UNIX UTC timestamp
\lstinline{1745817889} to UNIX UTC timestamp \lstinline{1746249889}
shall be queried. The first entry is taken.
The task name shall be edited to \lstinline{nonexistent_task},
the operation returning false.
The entries shall be re-queried, and the task name shall be
found equal to what it was at the beginning of the test.

\textit{Tests: } MR-DBI-460 Entry task edit.

\paragraph{MT-DBI-710 -- Entry location edit}
The entries over the date range from UNIX UTC timestamp
\lstinline{1745817889} to UNIX UTC timestamp \lstinline{1746249889}
shall be queried. The first entry is taken.
The location name shall be edited to the existing
\lstinline{location_MT-DBI-450}, the operation returning true.
The entries shall be re-queried, and the location name shall be
found equal to the input.

\textit{Tests: } MR-DBI-490 Entry location edit.

\paragraph{MT-DBI-720 -- Entry location edit non-existent}
The entries over the date range from UNIX UTC timestamp
\lstinline{1745817889} to UNIX UTC timestamp \lstinline{1746249889}
shall be queried. The first entry is taken.
The location name shall be edited to \lstinline{nonexistent_location},
the operation returning false.
The entries shall be re-queried, and the location name shall be
found equal to what it was at the beginning of the test.

\textit{Tests: } MR-DBI-490 Entry location edit.

\paragraph{MT-DBI-730 -- Entry start edit}
The entries over the date range from UNIX UTC timestamp
\lstinline{1745817889} to UNIX UTC timestamp \lstinline{1746249889}
shall be queried. The first entry is taken.
The entry's start date shall be modified to \lstinline{1745819888}
without error, the operation returning true.
The entries shall be re-queried, and the start date shall be
found equal to the input.

\textit{Tests: } MR-DBI-470 Entry start edit.

\paragraph{MT-DBI-740 -- Entry start edit, wrong order}
The entries over the date range from UNIX UTC timestamp
\lstinline{1745817889} to UNIX UTC timestamp \lstinline{1746249889}
shall be queried. The \emph{second} entry is taken.
The entry's start date shall be modified to \lstinline{1755819999},
the operation returning false.
The entries shall be re-queried, and the start date shall be
found equal to what it was at the beginning of the test.

\textit{Tests: } MR-DBI-470 Entry start edit.

\paragraph{MT-DBI-750 -- Entry start edit, overlapping}
The entries over the date range from UNIX UTC timestamp
\lstinline{1745817889} to UNIX UTC timestamp \lstinline{1746249889}
shall be queried. The second entry is taken.
The entry's start date shall be modified to \lstinline{1745819988},
the operation returning false.
The entries shall be re-queried, and the start date shall be
found equal to what it was at the beginning of the test.

\textit{Tests: } MR-DBI-470 Entry start edit.

\paragraph{MT-DBI-760 -- Entry stop edit}
The entries over the date range from UNIX UTC timestamp
\lstinline{1745817889} to UNIX UTC timestamp \lstinline{1746249889}
shall be queried. The second entry is taken.
The entry's stop date shall be modified to \lstinline{1745819998},
the operation returning true.
The entries shall be re-queried, and the stop date shall be
found equal to the input.

\textit{Tests: } MR-DBI-480 Entry stop edit.

\paragraph{MT-DBI-770 -- Entry stop edit, wrong order}
The entries over the date range from UNIX UTC timestamp
\lstinline{1745817889} to UNIX UTC timestamp \lstinline{1746249889}
shall be queried. The \emph{first} entry is taken.
The entry's stop date shall be modified to \lstinline{1744817889},
the operation returning false.
The entries shall be re-queried, and the stop date shall be
found equal to what it was at the beginning of the test.

\textit{Tests: } MR-DBI-480 Entry stop edit.

\paragraph{MT-DBI-780 -- Entry stop edit, overlapping}
The entries over the date range from UNIX UTC timestamp
\lstinline{1745817889} to UNIX UTC timestamp \lstinline{1746249889}
shall be queried. The first entry is taken.
The entry's stop date shall be modified to \lstinline{1746819889},
the operation returning false.
The entries shall be re-queried, and the stop date shall be
found equal to what it was at the beginning of the test.

\textit{Tests: } MR-DBI-480 Entry stop edit.

\paragraph{MT-DBI-790 -- Locked project removal}
The entries over the date range from UNIX UTC timestamp
\lstinline{1745817889} to UNIX UTC timestamp \lstinline{1746249889}
shall be queried. The first entry is taken.
The project id of the first entry shall be retrieved.
The number of DB projects shall be saved for reference.
The removal of this project from the DB shall be tried, the operation
returning false. The DB projects shall be re-queried and found to still
contain the same number of projects.

\textit{Tests: } MR-DBI-110 Entries locking hierarchy items removal.

\paragraph{MT-DBI-800 -- Locked task removal}
The entries over the date range from UNIX UTC timestamp
\lstinline{1745817889} to UNIX UTC timestamp \lstinline{1746249889}
shall be queried. The first entry is taken.
The task name of the first entry shall be retrieved.
Then, the list of DB tasks for the entry's project shall be retrieved, and the
task id shall be obtained.
The removal of this task from the DB shall be tried, the operation
returning false. The DB tasks shall be re-queried and found to still
contain the task in question.

\textit{Tests: } MR-DBI-110 Entries locking hierarchy items removal.

\paragraph{MT-DBI-810 -- Locked location removal}
The entries over the date range from UNIX UTC timestamp
\lstinline{1745817889} to UNIX UTC timestamp \lstinline{1746249889}
shall be queried. The first entry is taken.
The location name of the first entry shall be retrieved.
Then, the list of DB locations shall be retrieved, and the location
id of the location with the name from the entry shall be obtained.
The number of locations in the DB shall be saved for reference.
The removal of this location from the DB shall be tried, the operation
returning false. The DB locations shall be re-queried and found to still
contain the same number of locations.

\textit{Tests: } MR-DBI-110 Entries locking hierarchy items removal.

\paragraph{MT-DBI-820 -- Entries export}
The entries in the date range from UNIX UTC timestamp \lstinline{1745817889} to
UNIX UTC timestamp \lstinline{1746249889} shall be exported to export rows. The
number of rows must be exactly two.
The entries over the same date range shall be queried for reference. The row
fields for entry id, project name, task name, location name, start date and stop
date must be found equal between the export rows and the entries list.

\textit{Tests: } MR-DBI-540 Entries export.

\paragraph{MT-DBI-825 -- Entries export ordering}
The entries export rows obtained in MT-DBI-820 shall be found to
be in increasing order of start date.

\textit{Tests: } MR-DBI-540 Entries export.

\paragraph{MT-DBI-830 -- Entries export, empty}
The entries in the date range from UNIX UTC timestamp
\lstinline{1735817889} to UNIX UTC timestamp \lstinline{1736249889}
shall be exported to export rows without error. The set of rows must be empty.

\textit{Tests: } MR-DBI-540 Entries export.

\paragraph{MT-DBI-840 -- Project total report}
The project total in the date range from UNIX UTC timestamp
\lstinline{1745817889} to UNIX UTC timestamp \lstinline{1746249889} shall be
generated without error.
The entries shall be queried over the same date range for reference.
The project names, task names, and durations must match.

\textit{Tests: } MR-DBI-550 Project total

\paragraph{MT-DBI-850 -- Project total report, empty}
The project total in the date range from UNIX UTC timestamp
\lstinline{1735817889} to UNIX UTC timestamp \lstinline{1736249889}
shall be generated without error. The result must be empty.

\textit{Tests: } MR-DBI-550 Project total

\paragraph{MT-DBI-860 -- Weekly totals report}
The entries in the date range from UNIX UTC timestamp \lstinline{1745817889} to
UNIX UTC timestamp \lstinline{1746249889} shall be retrieved.
A \emph{week} shall be initialized from the start date of the first entry.
A weekly totals report shall be generated without error.
The value of every field in the report shall be checked, based on the data
present in entries.

\textit{Tests: } MR-DBI-560 Weekly totals

\paragraph{MT-DBI-870 -- Weekly totals report, empty}
A week shall be initialized from the UNIX UTC timestamp
\lstinline{1735817889}. A weekly totals report shall be generated without error.
The result must be empty.

\textit{Tests: } MR-DBI-560 Weekly totals

\subsection{Exporter}
The exporter module is tested by the following test cases.
We mainly test the adherence to [AD3].
The DB module is not used for testing, as the exporter module
is decoupled from it.
We refer to a \emph{default dataset}, which corresponds to
the export rows set, date range and timezone from the example
file in [AD3].

\paragraph{MT-EXP-010 -- Export file directory exception}
Using the \emph{default dataset}, and an export filepath target
set to a bare temporary directory, the export operation
must throw an exception.

\textit{Tests: } MR-EXP-020 Export file directory exception.

\paragraph{MT-EXP-020 -- Export file exists exception}
Using the \emph{default dataset}, and an export filepath target
set to an existing empty text file \lstinline{empty.csv} in a temporary
directory, the export operation must throw an exception.

\textit{Tests: } MR-EXP-030 Export file exists exception.

\paragraph{MT-EXP-030 -- Export file nonexistent path exception}
Using the \emph{default dataset}, and an export filepath target
set to a non-existent path in a temporary directory
\lstinline{tempdir/nonexistent/export.csv} in a temporary
directory, the export operation must throw an exception.

\textit{Tests: } MR-EXP-040 Export file non-existent path exception.

\paragraph{MT-EXP-040 -- Export file csv extension exception}
Using the \emph{default dataset}, and an export filepath target
set to the file \lstinline{tempdir/export.txt} in a valid
temporary directory, the export operation must throw an exception.

\textit{Tests: } IRS1-GEN-020 File extension

\paragraph{MT-EXP-050 -- Export file}
Using the \emph{default dataset}, and an export filepath
target set to the file \lstinline{tempdir/export.csv} in a valid
temporary directory, the export operation must be executed
without error. The file must exist as a result. The file must
be of non-zero size as a result.
This valid file is refered to as \emph{the file} in subsequent tests,
and is assumed to exist.

\textit{Tests: } MR-EXP-010 Export file.

\paragraph{MT-EXP-060 -- Export file UTF-8}
Using the created \emph{file}, load the first three characters \lstinline{# E}
as bytes. The bytes must be equal to
\begin{enumerate}
\item \lstinline{0x23},
\item \lstinline{0x20},
\item \lstinline{0x45}.
\end{enumerate}

\textit{Test: } IRS1-GEN-030 File encoding, IRS1-GEN-010 File type.

\paragraph{MT-EXP-070 -- Line endings}
The \emph{file} shall be loaded as a set of characters.
No carriage return character must be found.
At least one newline character must be found.

\textit{Tests: } IRS1-GEN-040 Line endings.

\paragraph{MT-EXP-080 -- File ending}
The \emph{file} shall be loaded as a set of characters.
The last character must be a newline character.

\textit{Tests: } IRS1-GEN-050 File ending.

\paragraph{MT-EXP-090 -- File structure}
The \emph{file} shall be loaded as a set of lines.
Exactly ten lines must be found.
The first six lines must begin with a hash character \lstinline{#}.
The last 4 lines must not begin with a hash character.

\textit{Tests: } IRS1-GEN-060 File structure, IRS1-GEN-070 Header location,
IRS1-GEN-080 Body location, IRS1-HED-010 Header format.

\paragraph{MT-EXP-100 -- Export date line}
The first line in \emph{the file} shall be prefixed exactly with
\lstinline{# Export date: }.

\textit{Tests: } IRS1-HED-100 Export date format, IRS1-HED-150 Header ordering

\paragraph{MT-EXP-110 -- Export date value}
The string after the prefix (MT-EXP-100) shall be parsed as a \emph{full date}.
This date must be equal to the current \emph{date} to within 1 minute error.

\textit{Tests: } IRS1-HED-020 Header export date

\paragraph{MT-EXP-120 -- Export date timezone}
The last 5 characters of the export date header line must match
with a \emph{plus} or \emph{minus} sign followed by four digits.

\textit{Tests: } IRS1-HED-050 Header dates timezone

\paragraph{MT-EXP-130 -- Period start date format}
The second line in \emph{the file} shall be prefixed exactly with
\lstinline{# Export start date: }.

\textit{Tests: } IRS1-HED-110 Period start date format, IRS1-HED-150 Header
ordering.

\paragraph{MT-EXP-140 -- Period start date value}
The string after the prefix (MT-EXP-130) must match exactly
\lstinline{01Jan2024 00:00:00 +0100}.

\textit{Tests: } IRS1-HED-030 Period start date, IRS1-HED-050 Header dates
timezone, IRS1-HED-090 Header date format.

\paragraph{MT-EXP-150 -- Period stop date format}
The third line in \emph{the file} shall be prefixed exactly with
\lstinline{# Export stop date: }.

\textit{Tests: } IRS1-HED-120 Period stop date format, IRS1-HED-150 Header
ordering.

\paragraph{MT-EXP-160 -- Period stop date value}
The string after the prefix (MT-EXP-150) must match exactly
\lstinline{31Dec2024 23:59:59 +0100}.

\textit{Tests: } IRS1-HED-040 Period stop date, IRS1-HED-050 Header dates
timezone, IRS1-HED-090 Header date format.

\paragraph{MT-EXP-170 -- Header timezone format}
The fourth line in \emph{the file} shall be prefixed exactly with
\lstinline{# Header timezone: }.

\textit{Tests: } IRS1-HED-130 Header timezone format, IRS1-HED-150 Header
ordering.

\paragraph{MT-EXP-180 -- Header timezone value}
The string after the prefix (MT-EXP-170) must match exactly
\lstinline{Europe/Paris}.

\textit{Tests: } IRS1-HED-060 Header timezone

\paragraph{MT-EXP-190 -- Program version format}
The fifth line in \emph{the file} shall be prefixed exactly with
\lstinline{# timesheeting version: }.

\textit{Tests: } IRS1-HED-140 Program version format, IRS1-HED-150 Header
ordering.

\paragraph{MT-EXP-200 -- Program version value}
The string after the prefix (MT-EXP-190) must match the program version
string indicated by the \emph{version} module.

\textit{Tests: } IRS1-HED-070 Header program version, IRS1-HED-080 Program
version string.

\paragraph{MT-EXP-210 -- DB version format}
The sixth line in \emph{the file} shall be prefixed exactly with
\lstinline{# timesheeting DB version: }.

\textit{Tests: } IRS1-HED-145 Database version format, IRS1-HED-150 Header
ordering.

\paragraph{MT-EXP-220 -- DB version value}
The string after the prefix (MT-EXP-210) must match with the \gls{DB}
version number indicated by the \emph{version} module, converted
from integer to a string.

\textit{Tests: } IRS1-HED-075 Header database version, IRS1-HED-085 Database
version string.

\paragraph{MT-EXP-230 -- CSV format}
The last 4 lines of \emph{the file} must all contain at least one comma.
The lines must not contain any comma not followed by a whitespace.

\textit{Tests: } IRS1-BDY-010 CSV format, IRS1-BDY-020 CSV delimiter.

\paragraph{MT-EXP-240 -- Column list}
The first line of the body of \emph{the file} shall match exactly
\lstinline{Entry ID, Project ID, Project name, Task ID, Task name, Location ID, Location name, Start date, Stop date}.

\textit{Tests: } IRS1-BDY-030 Body structure, IRS1-BDY-040 Column list.

\paragraph{MT-EXP-250 -- No empty fields}
None of the body lines of \emph{the file} shall contain
two consecutive comma separated only by eventual whitespace.

\textit{Tests: } IRS1-BDY-050 No empty fields.

\paragraph{MT-EXP-260 -- Entries lines}
The last three lines of \emph{the file} shall match exactly,

\begin{lstlisting}[numbers=none]
4, 1, Project1, 3, Task3, 1, Location1, 1729952454, 1729953654
8, 1, Project1, 11, Task11, 1, Location1, 1729953659, 1729953789
9, 15, Project15, 5, Task5, 3, Location3, 1729953888, 1729953988
\end{lstlisting}

\textit{Tests: } IRS1-BDY-060 Id format, IRS1-BDY-070 Timesheet date format,
IRS1-BDY-080 Timesheet name format, IRS1-BDY-090 Timesheet entries ordering,
IRS1-BDY-100 Timesheet entries period.
