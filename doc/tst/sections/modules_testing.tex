\section{Modules testing}
\subsection{version}
The version module is tested using automated unit tests.

\paragraph{MT-VER-010 -- Program version number}
We get the current program version number as a \emph{string}.
We check that its format is compatible with the following \emph{regular
  expression}: \lstinline{[0123456789]+\.[0123456789]+(?:dev)?}.

\textit{Tests: } MR-VER-010 -- Program version number

\paragraph{MT-VER-020 -- DB version number}
We get the current DB version as an integer. We test it is greater than zero.

\textit{Tests: } MR-VER-020 -- DB version number

\subsection{config}
The following automated unit tests are used for the config module.

\begin{minipage}{\linewidth}
  \begin{lstlisting}[caption={timesheeting nominal configuration file},
                     label={lst:config_file}]
[db]
file = "/tmp/timesheeting.db"

[time]
timezone = "Europe/Paris"
hours_per_workday = 7.7

[log]
file = "/tmp/timesheeting.log"
active_levels = [ "debug", "error", "info" ]
max_log_age = 604800

[keys]
  [keys.navigation]
  up = ["e", "UP"]
  down = ["n", "DOWN"]
  left = ["h", "LEFT"]
  right = ["i", "RIGHT"]
  subtabs = ["TAB"]
  previous = [","]
  next = ["."]
  duration_display = ["d"]
  entries_screen = ["1"]
  projects_screen = ["2"]
  locations_screen = ["3"]
  project_report_screen = ["4"]
  weekly_report_screen = ["5"]
  active_visibility = ["v"]
  quit = ["q"]
  [keys.actions]
  commit_entry = ["ENTER"]
  set_now = ["SPACE"]
  add = ["a"]
  rename = ["r"]
  remove = ["x"]
  active_toggle = ["b"]
  task_project_change = ["p"]
  [keys.edit_mode]
  validate = ["ENTER"]
  cancel = ["ESCAPE"]
  select_suggestion = ["TAB"]
\end{lstlisting} \end{minipage}

\begin{minipage}{\linewidth}
  \begin{lstlisting}[caption={timesheeting configuration file
                              with duplicate binding},
                     label={lst:config_file_duplicate}]
[db]
file = "/tmp/timesheeting.db"

[time]
timezone = "Europe/Paris"
hours_per_workday = 7.7

[log]
file = "/tmp/timesheeting.log"
active_levels = [ "debug", "error", "info" ]
max_log_age = 604800

[keys]
  [keys.navigation]
  up = ["e", "UP"]
  down = ["n", "DOWN"]
  left = ["h", "LEFT"]
  right = ["i", "RIGHT"]
  subtabs = ["TAB"]
  previous = [","]
  next = ["."]
  duration_display = ["d"]
  entries_screen = ["1"]
  projects_screen = ["2"]
  locations_screen = ["3"]
  project_report_screen = ["4"]
  weekly_report_screen = ["5"]
  active_visibility = ["v"]
  quit = ["q"]
  [keys.actions]
  commit_entry = ["ENTER"]
  set_now = ["SPACE"]
  add = ["a", "v"]
  rename = ["r"]
  remove = ["x"]
  active_toggle = ["b"]
  task_project_change = ["p"]
  [keys.edit_mode]
  validate = ["ENTER"]
  cancel = ["ESCAPE"]
  select_suggestion = ["TAB"]
\end{lstlisting} \end{minipage}

\begin{minipage}{\linewidth}
  \begin{lstlisting}[caption={timesheeting configuration file with invalid
                              special key},
                     label={lst:config_file_invalidstr}]
[db]
file = "/tmp/timesheeting.db"

[time]
timezone = "Europe/Paris"
hours_per_workday = 7.7

[log]
file = "/tmp/timesheeting.log"
active_levels = [ "debug", "error", "info" ]
max_log_age = 604800

[keys]
  [keys.navigation]
  up = ["e", "UP", "GOOFY"]
  down = ["n", "DOWN"]
  left = ["h", "LEFT"]
  right = ["i", "RIGHT"]
  subtabs = ["TAB"]
  previous = [","]
  next = ["."]
  duration_display = ["d"]
  entries_screen = ["1"]
  projects_screen = ["2"]
  locations_screen = ["3"]
  project_report_screen = ["4"]
  weekly_report_screen = ["5"]
  active_visibility = ["v"]
  quit = ["q"]
  [keys.actions]
  commit_entry = ["ENTER"]
  set_now = ["SPACE"]
  add = ["a"]
  rename = ["r"]
  remove = ["x"]
  active_toggle = ["b"]
  task_project_change = ["p"]
  [keys.edit_mode]
  validate = ["ENTER"]
  cancel = ["ESCAPE"]
  select_suggestion = ["TAB"]
\end{lstlisting} \end{minipage}

\paragraph{MT-CON-010 -- Config file loading}
Given the filepath to a file containing \cref{lst:config_file},
the \emph{config} module loader shall return the internal configuration
parameters representation without error.

\textit{Tests: } MR-CON-010 -- Load from path.

\paragraph{MT-CON-020 -- Config file non-existent}
Given the filepath \lstinline{/dev/null/nonexistent} as the path
to the configuration file, the \emph{config} module loader shall
emit an exception.

\textit{Tests: } MR-CON-020 -- Non-existent path.

\paragraph{MT-CON-030 -- Log filepath}
The configuration parameters loaded from \cref{lst:config_file}
shall contain a filepath \lstinline{log_filepath}.

\textit{Tests: } MR-CON-050 -- Log filepath

\paragraph{MT-CON-040 -- Log levels}
The configuration parameters loaded from \cref{lst:config_file}
shall contain a vector of strings \lstinline{log_levels_to_activate}.

\textit{Tests: } MR-CON-060 -- Log levels

\paragraph{MT-CON-050 -- DB filepath}
The configuration parameters loaded from \cref{lst:config_file}
shall contain a filepath \lstinline{db_filepath}.

\textit{Tests: } MR-CON-070 -- DB filepath

\paragraph{MT-CON-060 -- Timezone string}
The configuration parameters loaded from \cref{lst:config_file}
shall contain a string \lstinline{timezone}.

\textit{Tests: } MR-CON-080 -- Timezone string

\paragraph{MT-CON-070 -- Duration display days}
The configuration parameters loaded from \cref{lst:config_file}
shall contain a float \lstinline{hours_per_workday}.

\textit{Tests: } MR-CON-090 -- Duration display days

\paragraph{MT-CON-080 -- Bindings loading}
The configuration parameters at the node \lstinline{keys} loaded from
in \cref{lst:config_file} shall be tested for one to one mapping between
key and action.

\textit{Tests: }
\begin{itemize}
\item MR-CON-100 -- Binding list
\item MR-CON-110 -- Regular keys
\item MR-CON-120 -- Special keys
\item MR-CON-130 -- Multiple keys for an action
\end{itemize}

\paragraph{MT-CON-090 -- Binding duplicate}
Loading the configuration file \cref{lst:config_file_duplicate} shall
result in an exception being emitted.

\textit{Tests: } MR-CON-140 -- Protection against duplicates

\paragraph{MT-CON-100 -- Backspace mapping}
The configuration parameters loaded from \cref{lst:config_file}
shall include a binding for the backspace key to the \emph{backspace}
in the \emph{edit map}.

\textit{Tests: } MR-CON-150 -- Backspace mapping

\paragraph{MT-CON-110 -- Unbound mappings}
From the configuration parameters loaded from \cref{lst:config_file},
querying the action corresponding to key \lstinline{u} shall return
\lstinline{unbound} for both the \emph{normal map} and the \emph{edit map}.

\textit{Tests: } MR-CON-160 -- Unbound keys

\paragraph{MT-CON-120 -- Invalid special key}
Loading the configuration file \cref{lst:config_file_invalidstr} shall
result in an exception being emitted.

\textit{Tests: } MR-CON-170 -- Invalid key strings

\paragraph{MT-CON-130 -- Maximum log age}
The configuration parameters loaded from \cref{lst:config_file}
shall contain an unsigned integer \lstinline{max_log_age}.

\textit{Tests: } MR-CON-065 -- Maximum log age

\subsection{core}
\paragraph{MT-COR-010 -- Project generic item}
A generic item of type \lstinline{Project} shall be instantiated without
error with id 4, name \lstinline{project1}, and active flag set to true.
The id, name and active flag shall be retrieved and be found equal to
their input values.

\textit{Tests: } MR-COR-010 -- Generic item

\paragraph{MT-COR-020 -- Task generic item}
A generic item of type \lstinline{Task} shall be instantiated without
error with id 2, name \lstinline{task1}, and active flag set to false.
The id, name and active flag shall be retrieved and be found equal to
their input values.

\textit{Tests: } MR-COR-010 -- Generic item

\paragraph{MT-COR-030 -- Location generic item}
A generic item of type \lstinline{Location} shall be instantiated without
error with id 3, name \lstinline{location1}, and active flag set to true.
The id, name and active flag shall be retrieved and be found equal to
their input values.

\textit{Tests: } MR-COR-010 -- Generic item

\paragraph{MT-COR-040 -- Vector of generic item names}
A vector of 3 generic items of type \lstinline{Location} is created,
with the following id, name and activity flags:
\lstinline{1, location1, true}, \lstinline{2, location2, true},
\lstinline{3, location3, true}.
It shall be converted to a vector of the item names.
The result vector must be equal to \lstinline{location1, location2, location3}.

\textit{Tests: } MR-COR-020 -- Vector of generic item names

\paragraph{MT-COR-050 -- Entry object}
An Entry with id 5, project name \lstinline{project1}, task name
\lstinline{task1}, start date instantiated from UNIX timestamp
\lstinline{1745837288}, stop date from UNIX timestamp
\lstinline{1745838288} and location name \lstinline{location1}
shall be instantiated without error.
Each attribute shall be queried and found equal to the input.
The dates are compared using a conversion to UNIX timestamp.

\textit{Tests: } MR-COR-030 -- Entry

\paragraph{MT-COR-060 -- Entry to strings}
The entry from MT-COR-050 shall be converted to a vector of strings.
This vector must be equal to
\lstinline{project1, task1, start date display string in long format, stop date display string in long format, location1},
with the date strings equal to what is returned by the method for long
display format.

\textit{Tests: } MR-COR-040 -- Entry to strings

\paragraph{MT-COR-070 -- Entry to short strings}
The entry from MT-COR-060 shall be converted to a vector of short strings.
The result shall be equal to that in MT-COR-070, except the date display
strings must be in short format.

\textit{Tests: } MR-COR-050 -- Entry to short strings

\paragraph{MT-COR-080 -- Entry staging with values}
An entry staging object with project name \lstinline{project1},
task name \lstinline{task1}, start date instantiated from UNIX timestamp
\lstinline{1745837288}, stop date from UNIX timestamp
\lstinline{1745838288} and location name \lstinline{location1}
shall be instantiated without error.
Each attribute shall be queried and found equal to the input.
The dates are compared using a conversion to UNIX timestamp.

\textit{Tests: } MR-COR-060 -- Entry staging

\paragraph{MT-COR-090 -- Entry staging without values}
An entry staging with every attribute as a nullopt shall be instantiated
without error.

\textit{Tests: } MR-COR-070 -- Entry staging

\paragraph{MT-COR-100 -- Entry staging with values to strings}
The entry staging object from MT-COR-080 shall be converted to
a vector of string representations. The resulting vector shall be
found equal to the inputs, with dates in long display format.

\textit{Tests: } MR-COR-070 -- Entry staging to strings

\paragraph{MT-COR-110 -- Entry staging without values to strings}
The entry staging object from MT-COR-090 shall be converted to
a vector of string representations. The result vector shall
contain 5 strings with a single whitespace.

\textit{Tests: } MR-COR-070 -- Entry staging to strings

\paragraph{MT-COR-120 -- Entry staging with values to short strings}
The entry staging object from MT-COR-080 shall be converted to
a vector of short string representations. The resulting vector shall be
found equal to the inputs, with dates in short display format.

\textit{Tests: } MR-COR-080 -- Entry staging to short strings

\paragraph{MT-COR-130 -- Entry staging without values to short strings}
The entry staging object from MT-COR-090 shall be converted to
a vector of short string representations. The result vector shall
contain 5 strings with a single whitespace.

\textit{Tests: } MR-COR-080 -- Entry staging to short strings

\paragraph{MT-COR-140 -- Export row}
An export row shall be instantiated without error with the following
attributes: entry id 4, project id 15, project name \lstinline{project15},
task id 1, task name \lstinline{task33}, location id 4, location name
\lstinline{location44}, start date initialized to UNIX timestamp
\lstinline{1745837288}, stop date initialized to UNIX timestamp
\lstinline{1745838288}.
Each attribute will be queried and be found equal to its input value.

\textit{Tests: } MR-COR-090 -- Export row

\paragraph{MT-COR-150 -- Export row csv string}
The export row object from MT-COR-140 shall be converted to csv string
format. This string shall be equal to
\lstinline{4, 15, project15, 1, task33, 4, location44, 1745837288, 1745838288}.

\textit{Tests: } MR-COR-100 -- Export csv string

\paragraph{MR-COR-160 -- Project total instantiation}
The project total object shown as example in MR-COR-120 shall
be instantiated without error.

\textit{Tests: } MR-COR-110 Project total

\paragraph{MR-COR-170 -- Project total to menu items}
The project total object shown as example in MR-COR-120 shall
be converted to menu items without error.
The number of menu items shall be 8.
Their cell string and display strings shall be found equal
to either the project and task names, or the duration in
and string format. The face of the first two items shall
be bold, the face of the 6 next items shall be normal.

\textit{Tests: } MR-COR-120 Project total to menu items.

\paragraph{MR-COR-180 -- Weekly totals instantiation}
The weekly totals shown as example in MR-COR-140 shall be
instantiated without error.

\textit{Tests: } MR-COR-130 Weekly totals

\paragraph{MR-COR-190 -- Weekly totals to menu items}
The weekly totals shown as example in MR-COR-140 shall be converted to menu
items without error. The number of menu items shall be 72.

\textit{Tests: } MR-COR-140 Weekly totals to menu items.
