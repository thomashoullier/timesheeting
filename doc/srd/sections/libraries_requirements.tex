\section{Libraries requirements}
\subsection{config\textunderscore lib}
\subsubsection{Utilities}
\paragraph{LR-CON-010 -- Tilde expansion utility}
The config\textunderscore lib shall provide a utility function for tilde
expansion of filepaths. It replaces eventual leading \lstinline{~/}
with \lstinline{$HOME} in filepaths. It does nothing on filepaths
without tilde.

\textit{Uplink: } TODO.

\textit{Tested by: } LT-CON-010 Tilde expansion nominal, LT-CON-020 Tilde
expansion none.

\paragraph{LR-CON-020 -- First existing file}
The config\textunderscore lib shall provide a utility function, which,
given a filepath suffix and a list of folders, returns the first found
existing filepath by trying the suffix over the successive folder
candidates. It returns nothing if no possible file exists.

\textit{Uplink: } TODO.

\textit{Tested by: } LT-CON-030 First existing file found, LT-CON-040 First
existing file not found.

\subsubsection{Configuration loader}
\paragraph{LR-CON-030 -- Configuration file format}
The config\textunderscore lib file loader shall use the TOML file format [AD3]
for the configuration file.

\textit{Uplink: } TODO.

\textit{Tested by: } LT-CON-050 Configuration file loading.

\paragraph{LR-CON-040 -- String loading}
The configuration file loader shall allow reading parameters of type
\emph{string}.

\textit{Uplink: } TODO.

\textit{Tested by: } LT-CON-070 String parameter reading.

\paragraph{LR-CON-050 -- String empty case}
The configuration file loader shall emit an exception if a loaded
\emph{string} is empty (\textit{ie} \lstinline{""}).

\textit{Uplink: } TODO.

\textit{Tested by: } LT-CON-080 String parameter empty.

\paragraph{LR-CON-060 -- Filepath loading}
The configuration file loader shall allow reading parameters of type
\emph{filepath} with automatic tilde expansion.

\textit{Uplink: } TODO.

\textit{Tested by: } LT-CON-090 Filepath reading.

\paragraph{LR-CON-070 -- Filepath parent non-existent}
The configuration file loader shall emit an exception if the loaded
\emph{filepath} direct parent does not exist.

\textit{Uplink: } TODO.

\textit{Tested by: } LT-CON-100 Filepath non-existent.

\paragraph{LR-CON-080 -- Float loading}
The configuration file loader shall allow reading parameters of type
\emph{float}.

\textit{Uplink: } TODO.

\textit{Tested by: } LT-CON-110 Float reading.

\paragraph{LR-CON-090 -- Parameter empty case}
The configuration file loader shall emit an exception if any parameter
is empty, \textit{ie} \lstinline{parameter = }.

\textit{Uplink: } TODO.

\textit{Tested by: } LT-CON-120 Empty parameter case.

\paragraph{LR-CON-100 -- Vector of strings loading}
The configuration file loader shall allow reading parameters of type
\emph{vector of strings}. The order of strings in the vector is preserved.

\textit{Uplink: } TODO.

\textit{Tested by: } LT-CON-130 Vector of strings reading.

\paragraph{LR-CON-110 -- Vector of non-strings case}
The configuration file loader shall emit an exception if a loaded
\emph{vector of strings} does not in fact contain strings.

\textit{Uplink: } TODO.

\textit{Tested by: } LT-CON-140 Vector of non-strings case.

\paragraph{LR-CON-120 -- Configuration file nonexistent}
The configuration file loader shall emit an exception if it is provided
a configuration filepath which does not exist.

\textit{Uplink: } TODO.

\paragraph{LR-CON-130 -- Integer loading}
The configuration file loader shall allow reading parameters of type
\emph{integer}.

\textit{Uplink: } TODO.

\textit{Tested by: } LT-CON-150 Unsigned integer case.

\subsection{time\textunderscore lib}
\subsubsection{time\textunderscore zone}
The time\textunderscore zone object is a singleton which provides the
timezone information to the rest of the program.

\paragraph{LR-TMZ-010 -- Time zone initialization}
The timezone object shall be initialized with a valid TZ identifier string,
as defined in [AD4].

TODO: uplink

\textit{Tested by: } LT-TMZ-020 Time zone initialization.

\paragraph{LR-TMZ-020 -- Time zone singleton}
The timezone object shall be a singleton. It is initialized once and the same
instance is retrieved through a \lstinline{get} method subsequently.

TODO: uplink

\textit{Tested by: } LT-TMZ-030 Time zone name, LT-TMZ-040 std
time\textunderscore zone.

\paragraph{LR-TMZ-030 -- Invalid time zone}
The timezone object shall emit an exception if it is initialized with
an invalid TZ string (\textit{ie} a string outside of those defined in [AD4]).

TODO: uplink

\textit{Tested by: } LT-TMZ-010 Invalid time zone.

\paragraph{LR-TMZ-040 -- Time zone name}
The timezone class shall provide a method to retrieve its TZ identifier string.

TODO: uplink

\textit{Tested by: } LT-TMZ-030 Time zone name.

\paragraph{LR-TMZ-050 -- std time\textunderscore zone}
The timezone class shall provide a method to retrieve the
\lstinline{std::chrono::time_zone} representation corresponding to the set
time zone.

TODO: uplink

\textit{Tested by: } LT-TMZ-040 std time\textunderscore zone.

\subsubsection{date}
A Date is a representation of a time point in UTC using the system clock
as time reference.

\paragraph{LR-DAT-010 -- Current time initialization}
The Date shall allow initialization at the current time.

TODO: uplink

\textit{Tested by: } LT-DAT-030 Current time initialization.

\paragraph{LR-DAT-020 -- std::time\textunderscore point initialization}
The Date shall allow initialization from a \lstinline{std::time_point}.

TODO: uplink

\textit{Tested by: } LT-DAT-010 std::time\textunderscore point initialization.

\paragraph{LR-DAT-030 -- Beginning of year}
The Date shall allow initialization at the beginning of the current year
in the current time zone (as set in the TimeZone singleton).

TODO: uplink

\textit{Tested by: } LT-DAT-150 Beginning of year.

\paragraph{LR-DAT-040 -- UNIX timestamp initialization}
The Date shall allow initialization from a UNIX timestamp in seconds.

TODO: uplink

\textit{Tested by: } LT-DAT-040 UNIX timestamp initialization.

\paragraph{LR-DAT-050 -- Date string initialization}
The Date shall allow initialization from a string in the format
\lstinline{%d%b%Y %H:%M:%S}, with format specifiers as defined in [AD5].
The date initialization uses time zone currently set
in the TimeZone singleton.

An example of a valid initialization string is \lstinline {19Jan2025 09:41:34}.

TODO: uplink

\textit{Tested by: } LT-DAT-060 Date string initialization,
LT-DAT-190 Date parsing before 1970,
LT-DAT-200 Date parsing beyond 2038,
LT-DAT-210 Dates beyond four digits clipped,
LT-DAT-220 Dates with minus sign exception.

\paragraph{LR-DAT-060 -- Date string shortcuts}
The Date shall allow initialization from shortened strings
with formats,
\begin{itemize}
\item \lstinline{%d%b%Y %H:%M}
\item \lstinline{%d%b%Y %H}
\item \lstinline{%d%b%Y}
\end{itemize}
(with the same set of format specifiers as in LR-DAT-050),
replacing the omitted items from the full format defined in LR-DAT-050
by zero values.

Examples of shortened initialization strings and their full equivalent are
given in \cref{tab:date_shortened}.

\begin{table}
  \caption{\label{tab:date_shortened}
    Shortened date initialization string examples}
  \begin{tabular}{| c | c |} \hline
    \textbf{Shortened} & \textbf{Full} \\ \hline
    \texttt{19Jan2025 09:41} & \texttt{19Jan2025 09:41:00} \\ \hline
    \texttt{19Jan2025 09} & \texttt{19Jan2025 09:00:00} \\ \hline
    \texttt{19Jan2025} & \texttt{19Jan2025 00:00:00} \\ \hline
  \end{tabular}\end{table}

TODO: uplink

\textit{Tested by: } LT-DAT-070 Date string shortcuts 1,
LT-DAT-080 Date string shortcuts 2,
LT-DAT-090 Date string shortcuts 3.

\paragraph{LR-DAT-065 -- Date string shortcut with prefix}
The initializer specified in LR-DAT-060 shall allow a mode where
a prefix string is added to the date string to parse.
The initial date string is first tried through all the formats
of LR-DAT-060, and if all fail, the parsing is retried with
the prefix string present.

\textit{Rationale:} This allows parsing user strings like \lstinline{17:20}
with an automatically provided prefix day string like \lstinline{30May2025},
while still accepting inputs such as \lstinline{30May2025 17:20} directly.

\textit{Tested by: } LT-DAT-095 Date string shortcut with prefix.

\paragraph{LR-DAT-070 -- Date string invalid}
During Date initialization with a date string, an exception shall be
thrown if an invalid date string is used (\textit{ie} outside of the formats
specified).

TODO: uplink

\textit{Tested by: } LT-DAT-100 Date string invalid.

\paragraph{LR-DAT-075 -- Date std::time\textunderscore point access}
The Date shall allow read access to its internal
\lstinline{std::chrono::time_point} representation.

TODO: uplink

\textit{Tested by: } LT-DAT-020 Date std::time\textunderscore point access.

\paragraph{LR-DAT-080 -- Date output string}
The Date shall be convertible to a string in format
\lstinline{%d%b%Y %H:%M:%S} (see [AD5]) in the current time zone.

TODO: uplink

\textit{Tested by: } LT-DAT-110 Date output string.

\paragraph{LR-DAT-090 -- Date output hours/minutes}
The Date shall be convertible to a string in format
\lstinline{%H:%M} (see [AD5]) in the current time zone.

TODO: uplink

\textit{Tested by: } LT-DAT-120 Date output hours/minutes.

\paragraph{LR-DAT-100 -- Date output unambiguous string}
The Date shall be convertible to a string in format
\lstinline{%d%b%Y %H:%M:%S %z} (see [AD5]) in the current time zone.

TODO: uplink

\textit{Tested by: } LT-DAT-130 Date output unambiguous string.

\paragraph{LR-DAT-110 -- Date output UNIX timestamp}
The Date shall be convertible to a UNIX timestamp in seconds in \gls{UTC}.

TODO: uplink

\textit{Tested by: } LT-DAT-050 UNIX timestamp output.

\paragraph{LR-DAT-120 -- Date output day/month/year}
The Date shall be convertible to a string in format
\lstinline{%d%b%Y} (see [AD5]) in the current time zone.

TODO: uplink

\textit{Tested by: } LT-DAT-140 Date output day/month/year.

\paragraph{LR-DAT-130 -- Second resolution}
The Date shall represent time points with a resolution of at most 1 second.

TODO: uplink

\textit{Tested by: } LT-DAT-020 Date std::time\textunderscore point access.

\paragraph{LR-DAT-140 -- Date comparison}
The Date class shall provide \emph{lesser than}, \emph{greater than},
\emph{lesser than or equal}, and \emph{greater than or equal} comparison
operators.

TODO: uplink

\textit{Tested by: } LT-DAT-160 Date comparison.

\paragraph{LR-DAT-150 -- Date ago initialization}
The Date shall allow initialization at a number of seconds ago from
\emph{now}.

TODO: uplink

\textit{Tested by: } LT-DAT-170 Date ago initialization.

\paragraph{LR-DAT-160 -- Date unambiguous string initialization}
The Date shall allow initialization from a string in the
same format as the one outputted in LR-DAT-100.

TODO: uplink

\textit{Tested by: } LT-DAT-180 Date unambiguous string initialization.

\subsubsection{DateRange}
A DateRange represents a range between a start Date and a stop Date.

\paragraph{LR-DTR-010 -- DateRange initialization}
The DateRange shall be initialized using a start Date and a stop Date.

TODO: uplink

\textit{Tested by: } LT-DTR-010 DateRange initialization.

\paragraph{LR-DTR-020 -- DateRange ordering}
The DateRange initialization shall emit an exception if the
start Date is \emph{greater than} the stop Date.

TODO: uplink

\textit{Tested by: } LT-DTR-020 DateRange ordering.

\paragraph{LR-DTR-030 -- Dates read access}
The DateRange shall allow read access to the start and stop Date.

TODO: uplink

\textit{Tested by: } LT-DTR-030 Dates read access.

\paragraph{LR-DTR-040 -- DateRange to strings}
The DateRange shall be convertible to a vector of two Date strings
as defined in LR-DAT-080.

TODO: uplink

\textit{Tested by: } LT-DTR-040 DateRange to strings.

\paragraph{LR-DTR-050 -- DateRange to day strings}
The DateRange shall be convertible to a vector of two Date day strings
as defined in LR-DAT-120.

TODO: uplink

\textit{Tested by: } LT-DTR-050 DateRange to day strings.

\paragraph{LR-DTR-060 -- DateRange contains Date}
The DateRange shall provide a predicate which, given a Date,
tests whether the Date is included in the Date range.
Note the inclusion is in the sense \lstinline{start <= date <= stop}.

\textit{Tested by: } LT-DTR-060 DateRange contains Date.

\subsubsection{Day}
A Day corresponds to a DateRange covering a single calendar day in
the current TimeZone. A given day starts at midnight and ends at the next
midnight.

\paragraph{LR-DAY-010 -- Now initialization}
The default Day initialization shall be to the current calendar day,
as indicated by the system clock, in the time zone currently set
in TimeZone.

TODO: uplink

\textit{Tested by: } LT-DAY-050 Now initialization.

\paragraph{LR-DAY-020 -- year/month/day initialization}
Day shall allow initialization from a \lstinline{std::chrono::year_month_day}
object, which represents a calendar day.

TODO: uplink

\textit{Tested by: } LT-DAY-010 Day ymd initialization.

\paragraph{LR-DAY-030 -- DateRange representation}
The Day shall be convertible to a corresponding DateRange.

TODO: uplink

\textit{Tested by: } LT-DAY-020 DateRange representation.

\paragraph{LR-DAY-040 -- DateRange start and stop Date}
The DateRange obtained from a Day shall have a start Date set
to \lstinline{00:00:00} and stop Date set to \lstinline{00:00:00} of the
following day, in the currently set time zone and for the currently set calendar
day.

TODO: uplink

\textit{Tested by: } LT-DAY-030 DateRange start and stop Date.

\paragraph{LR-DAY-050 -- String representation}
The Day shall be convertible to a string representation in format
\lstinline{%d%b%Y %a} (see [AD5]).

For instance, \lstinline{21Jan2025 Tue} is a valid string representation
for a Day.

TODO: uplink

\textit{Tested by: } LT-DAY-040 String representation.

\paragraph{LR-DAY-060 -- Next and previous}
Day shall include methods to select the \emph{next} and \emph{previous} days
from the currently set calendar day.

For instance, if the current Day is set to \lstinline{21Jan2025}, calling
\emph{previous} must change the Day to \lstinline{20Jan2025}.

TODO: uplink

\textit{Tested by: } LT-DAY-060 Next, LT-DAY-070 Previous.

\paragraph{LR-DAY-070 -- Day/Month/Year string representation}
The Day shall be convertible to a string representation in the format
\lstinline{%d%b%Y}.

\textit{Tested by: } LT-DAY-080 Day/Month/Year string representation.

\paragraph{LR-DAY-080 -- Day contains Date}
The Day shall provide a method testing whether a given Date is contained
within the Day understood as a DateRange.

\textit{Tested by: } LT-DAY-090 Day contains Date.

\subsubsection{Week}
A \emph{Week} represents a DateRange from a monday on midnight to midnight of
the next monday, for a particular calendar week.
The dates are as currently defined in the TimeZone.

\paragraph{LR-WEK-010 -- Now initialization}
The Week shall allow initialization at the current calendar week, as defined by
the system clock.

TODO: uplink

\textit{Tested by: } LT-WEK-030 Now initialization.

\paragraph{LR-WEK-020 -- Date initialization}
The Week shall allow initialization to a week containing a given Date.

TODO: uplink

\textit{Tested by: } LT-WEK-010 Date initialization,
LT-WEK-020 DateRange conversion.

\paragraph{LR-WEK-030 -- DateRange conversion}
The Week shall be convertible to a DateRange. The start Date of the DateRange
is set to midnight of the monday of the week currently pointed to, and the
stop Date is at midnight on the next monday.

\textit{Tested by: } LT-WEK-020 DateRange conversion.

\paragraph{LR-WEK-040 -- String representation}
The Week shall be convertible to a string representation in the format
\lstinline{%d%b%Y W%V} (see [AD5]), with the day string \lstinline{%d%b%Y}
set to the monday of the week.

For instance, \lstinline{20Jan2025 W04} is a valid string representation.

\textit{Tested by: } LT-WEK-040 String representation.

\paragraph{LR-WEK-050 -- Array of days}
The Week shall be convertible to an array of all the Days contained
in the week, in order from monday to sunday.

\textit{Tested by: } LT-WEK-050 Array of days.

\paragraph{LR-WEK-060 -- Next and previous}
The Week shall include methods to select the \emph{next} and \emph{previous}
weeks from the currently set calendar week.

\textit{Tested by: } LT-WEK-060 Next, LT-WEK-070 Previous.

\subsubsection{Duration}
A Duration is a length of time. It does not have a particular location
in terms of date.

\paragraph{LR-DUR-010 -- Zero initialization}
The Duration shall default initialize to a duration of zero.

\textit{Tested by: } LT-DUR-010 Zero initialization pass,
LT-DUR-020 Zero initialization seconds string.

\paragraph{LR-DUR-020 -- Seconds initialization}
The Duration shall allow initialization from a count of seconds.

TODO: uplink

\textit{Tested by: } LT-DUR-030 Seconds initialization pass,
LT-DUR-040 Seconds initialization value.

\paragraph{LR-DUR-030 -- Duration resolution}
The Duration shall have a resolution of at most one second.

TODO: uplink

\textit{Tested by: } LT-DUR-050 Duration resolution.

\paragraph{LR-DUR-040 -- Seconds string}
The Duration shall be convertible to a string in format
\lstinline{"x seconds"}, with \lstinline{x} the seconds count
for the duration.

For instance \lstinline{"104592 seconds"} is a valid string.

TODO: uplink

\textit{Tested by: } LT-DUR-020 Zero initialization seconds string,
LT-DUR-040 Seconds initialization value.

\paragraph{LR-DUR-050 -- Hours string}
The Duration shall be convertible to a string in format
\lstinline{"x hours"}, with \lstinline{x} the hours count
for the duration at three decimal places of precision.

For instance \lstinline{"29.053 hours"} is a valid string.

\textit{Tested by: } LT-DUR-060 Hours string.

\paragraph{LR-DUR-060 -- Days string}
The Duration shall be convertible to a string in format
\lstinline{"x days"}, with \lstinline{x} the days count
for the duration at three decimal places of precision.
The number of hours in a day must be provided to the
computation externally.

For instance \lstinline{"3.773 days"} is a valid string.

\textit{Tested by: } LT-DUR-070 Days string.

\paragraph{LR-DUR-070 -- Seconds short string}
The Duration shall be convertible to a shortened string which
is the one in LR-DUR-040 without the \lstinline{" seconds"}
part.

For instance, \lstinline{"104592"} is a valid string.

\textit{Tested by: } LT-DUR-080 Seconds short string.

\paragraph{LR-DUR-080 -- Hours short string}
The Duration shall be convertible to a shortened string
which is the one from LR-DUR-050 without the \lstinline{" hours"}
part.

For instance \lstinline{"29.053"} is a valid string.

\textit{Tested by: } LT-DUR-090 Hours short string.

\paragraph{LR-DUR-090 -- Days short string}
The Duration shall be convertible to a shortened string
which is the one from LR-DUR-060 without the \lstinline{" days"}
part.

For instance \lstinline{"3.773"} is a valid string.

\textit{Tested by: } LT-DUR-100 Days short string.

\paragraph{LR-DUR-100 -- Zero second short string}
A Duration with a value of 0 shall return a single whitespace
\lstinline{" "} as a second short string.

\textit{Tested by: } LT-DUR-110 Zero second short string.

\paragraph{LR-DUR-110 -- Zero hour short string}
A Duration with a value of 0 shall return a single whitespace
\lstinline{" "} as a hour short string.

\textit{Tested by: } LT-DUR-120 Zero hour short string.

\paragraph{LR-DUR-120 -- Zero day short string}
A Duration with a value of 0 shall return a single whitespace
\lstinline{" "} as a day short string.

\textit{Tested by: } LT-DUR-130 Zero day short string.

\paragraph{LR-DUR-130 -- Day string invalid hours}
When given a count of hours in a day which is lesser than or equal to zero,
the day string method shall throw an exception.

\textit{Tested by: } LT-DUR-140 Day string invalid hours.

\paragraph{LR-DUR-140 -- Day short string invalid hours}
When given a count of hours in a day which is lesser than or equal to zero,
the day short string method shall throw an exception.

\textit{Tested by: } LT-DUR-150 Day short string invalid hours.

\subsubsection{Duration displayer}
The DurationDisplayer is a singleton for displaying a Duration according
to a currently selected format.

\paragraph{LR-DRD-010 -- DurationDisplayer initialization pass}
The DurationDisplayer shall be initialized with a number of work hours
in a day.

\textit{Tested by: } LT-DRD-010 DurationDisplayer initialization pass.

\paragraph{LR-DRD-020 -- DurationDisplayer singleton}
The DurationDisplayer shall be a singleton. It is initialized onec and
the same instance is retrieved through a \lstinline{get} method
subsequently.

\textit{Tested by: } LT-DRD-020 Hours string display,
LT-DRD-030 Hours short string display,
LT-DRD-040 Days string display,
LT-DRD-050 Days short string display,
LT-DRD-060 Seconds string display,
LT-DRD-070 Seconds short string display.

\paragraph{LR-DRD-030 -- Display formats}
The DurationDisplayer shall allow selecting the formats: \emph{hours},
\emph{days}, \emph{seconds}.
These refer to the second, hour and day strings for Duration.

\textit{Tested by: } LT-DRD-020 Hours string display,
LT-DRD-030 Hours short string display,
LT-DRD-040 Days string display,
LT-DRD-050 Days short string display,
LT-DRD-060 Seconds string display,
LT-DRD-070 Seconds short string display.

\paragraph{LR-DRD-040 -- Default format}
The default format for displaying a Duration upon initialization shall
be \emph{hours}.

\textit{Tested by: } LT-DRD-020 Hours string display,
LT-DRD-030 Hours short string display.

\paragraph{LR-DRD-050 -- Format cycling}
The currently selected format of the DurationDisplayer shall be selected
through a \emph{cycle} method.

The cycle between formats is as follows:
\begin{itemize}
\item hours $\rightarrow$ days
\item days $\rightarrow$ seconds
\item seconds $\rightarrow$ hours
\end{itemize}

\textit{Tested by: } LT-DRD-040 Days string display,
LT-DRD-060 Seconds string display,
LT-DRD-080 Cycling back to hours.

\paragraph{LR-DRD-060 -- String display}
Given a Duration, the DurationDisplayer shall output a string depending
on the currently selected format, as follows,
\begin{itemize}
\item hours: string from LR-DUR-050,
\item days: string from LR-DUR-060 with number of hours as initialized in the
            DurationDisplayer,
\item seconds: string from LR-DUR-040.
\end{itemize}

\textit{Tested by: } LT-DRD-020 Hours string display,
LT-DRD-040 Days string display,
LT-DRD-060 Seconds string display.

\paragraph{LR-DRD-070 -- Short string display}
Given a Duration, the DurationDisplayer shall output a short string depending
on the currently selected format, as follows,
\begin{itemize}
\item hours: string from LR-DUR-080,
\item days: string from LR-DUR-090 with number of hours as initialized in the
            DurationDisplayer,
\item seconds: string from LR-DUR-070.
\end{itemize}

\textit{Tested by: } LT-DRD-030 Hours short string display,
LT-DRD-050 Days short string display,
LT-DRD-070 Seconds short string display.

\subsection{log\textunderscore lib}
The log\textunderscore lib provides a logging interface to the program.
It logs messages of various \emph{log levels} to a file. The interface
is done through a \emph{logger} singleton.

\paragraph{LR-LOG-010 -- Logging}
The logger shall allow logging messages to a log file.

\textit{Tested by: } LT-LOG-060 Logging.

\paragraph{LR-LOG-020 -- Log levels}
The logger shall log only the messages with a log level which is
activated in the logger. The messages with a log level which is
not activated in the logger are not logged.

The log levels are,
\begin{enumerate}
\item \lstinline{info},
\item \lstinline{debug},
\item \lstinline{error}.
\end{enumerate}

The default log level for messages is \lstinline{info}.

\textit{Tested by: } LT-LOG-080 Logging custom active level,
LT-LOG-090 Logging custom inactive level.

\paragraph{LR-LOG-030 -- Log cleanup}
The logger shall clean existing log files of old entries on initialization.
The entries are considered old based on a maximal age threshold.

\textit{Tested by: } LT-LOG-050 Log cleanup.

\paragraph{LR-LOG-040 -- Log file}
The logger initialization shall be given a filepath to the log file
to use. The file may or may not already exist.

\textit{Tested by: } LT-LOG-040 Log file initialization.

\paragraph{LR-LOG-050 -- Log file exception}
The logger initialization shall throw an exception if the provided
log filepath is in a non-existent directory.

\textit{Tested by: } LT-LOG-010 Opening log file in non-existent folder.

\paragraph{LR-LOG-060 -- Log file invalid}
The logger initialization shall throw an exception if the provided
log file is filled with invalid data (not log entries).

\textit{Tested by: } LT-LOG-030 Log file invalid.

\paragraph{LR-LOG-070 -- Log levels initialization}
The logger initialization shall be given the set of log levels
to activate, by string.

\textit{Tested by: } LT-LOG-040 Log file initialization.

\paragraph{LR-LOG-080 -- Log maximum age initialization}
The logger initialization shall be given the maximal age threshold
in seconds for entries cleanup.

\textit{Tested by: } LT-LOG-040 Log file initialization.

\paragraph{LR-LOG-090 -- Log entry format}
The log entries shall be written in the following format,
on a single line,
\begin{enumerate}
\item The full current date string (LR-DAT-100 -- Date output unambiguous
  string),
\item followed by a space,
\item followed by the log level string in square brackets \lstinline{[]},
\item followed by a space, a semicolon \lstinline{:}, and another
  space,
\item followed by the log message.
\end{enumerate}

For instance, the following log entry is valid,
\begin{lstlisting}[numbers=none]
17May2025 09:16:36 +0200 [debug] : timesheeting UI initialized.
\end{lstlisting}

\textit{Tested by: } LT-LOG-070 Log entry format.

\paragraph{LR-LOG-100 -- Log chronometer}
The logger shall include a chronometer feature controlled by
a \lstinline{tick}, \lstinline{tock} mechanism.
The elapsed duration between the \lstinline{tick} and \lstinline{tock}
calls is logged with a \lstinline{debug} log level.

\textit{Tested by: } LT-LOG-100 Log chronometer.

\paragraph{LR-LOG-110 -- Log chronometer message}
The user of the logger may provide a message to the \lstinline{tock}
call of the chronometer. If no message is provided then the default
is the empty string \lstinline{""}.

\textit{Tested by: } LT-LOG-110 Log chronometer message.

\paragraph{LR-LOG-120 -- Log chronometer entry format}
The log chronometer entry format follows the usual log entry format,
the log message shall be in the following format.
\begin{enumerate}
\item The user message,
\item followed by the logged duration expressed in milliseconds
  with three decimal places, followed by a space,
\item followed by the suffix \lstinline{ms.}
\end{enumerate}

\textit{Tested by: } LT-LOG-120 Log chronometer entry format.

\paragraph{LR-LOG-130 -- Log message multiline exception}
The logger shall throw an exception if a multiline message is provided
for logging.

\textit{Tested by: } LT-LOG-130 Log message multiline exception,
LT-LOG-140 Chronometer message multiline exception.

\paragraph{LR-LOG-140 -- Unknown log levels exception}
In case an unknown log level string is provided to the logger initialization,
the logger shall throw an exception.

\textit{Tested by: } LT-LOG-020 Unknown log levels exception.

\subsection{db\textunderscore lib}
The db\textunderscore lib contains a handle to a \gls{DB}, and a statement
wrapper.

\subsubsection{DB handle}
\paragraph{LR-DBL-010 DB handle constructor}
The DB handle shall initialize a connection to a DB when provided
with a filepath to either,
\begin{enumerate}
\item a new file in an existing directory,
\item an existing DB file.
\end{enumerate}

\textit{Tested by: } LT-DBL-010 DB handle constructor,
LT-DBL-030 Saving the DB,
LT-DBL-040 Opening existing DB.

\paragraph{LR-DBL-020 DB handle non-existing folder}
The DB handle constructor shall throw an exception when provided with a filepath
to a non-existent folder.

\textit{Tested by: } LT-DBL-080 Opening DB in non-existent folder.

\paragraph{LR-DBL-030 DB handle invalid file}
The DB handle constructor shall throw an exception when provided with a
file which is not a valid DB file, eg a completely unrelated text file.

\textit{Tested by: } LT-DBL-090 Opening an invalid DB file.

\paragraph{LR-DBL-040 check DB user version}
The DB handle shall provide a method to check the user version of the DB,
\begin{enumerate}
\item an exception is thrown in case of version mismatch,
\item the version is initialized to the provided number if it was not
      initialized.
\end{enumerate}

\textit{Tested by: } LT-DBL-020 Set user version,
LT-DBL-050 User version check,
LT-DBL-060 User version check exception.

\paragraph{LR-DBL-050 get DB user version}
The DB handle shall provide a method to get the user version of the DB.

\textit{Tested by: } LT-DBL-070 User version get.

\paragraph{LR-DBL-060 execute statement}
The DB handle shall provide a method to execute a raw statement, provided
as a string, on an open DB.

\textit{Tested by: } LT-DBL-100 Executing a valid statement.

\paragraph{LR-DBL-070 execute statement exceptions}
The DB handle shall throw an exception in case a non-nominal statement
execution return code is encountered when executing a statement (LR-DBL-060).

\textit{Tested by: } LT-DBL-110 Executing an invalid statement.

\paragraph{LR-DBL-080 prepare statement}
The DB handle shall provide a method to prepare a statement, provided
as a string, on an open DB.

\textit{Tested by: } LT-DBL-120 Preparing a valid statement.

\paragraph{LR-DBL-090 prepare statement exceptions}
The DB handle shall throw an exception in case an error is encountered when
preparing a statement (LR-DBL-080).

\textit{Tested by: } LT-DBL-130 Preparing an invalid statement.

\subsubsection{Statement wrapper}
\paragraph{LR-DBL-100 Execute statement}
A statement shall allow simple execution (single step statement). If needed, the
statement must allow repeated execution.

\textit{Tested by: } LT-DBL-140 Executing statement,
LT-DBL-160 Repeat statement execution.

\paragraph{LR-DBL-110 Execute invalid statement}
Executing an invalid statement shall raise an exception.

\textit{Tested by: } LT-DBL-150 Executing invalid statement.

\paragraph{LR-DBL-120 Statement execution return value}
The statement execution shall return true on successful execution, and
false if the execution would have violated a database constraint, and thus
failed.

\textit{Tested by: } LT-DBL-140 Executing statement,
LT-DBL-160 Repeat statement execution,
LT-DBL-170 Statement execution constraint violation.

\paragraph{LR-DBL-130 Statement step}
A statement shall allow a step execution (single step in a multi-step
statement).

\textit{Tested by: } LT-DBL-180 Statement stepping,
LT-DBL-190 Statement stepping second,
LT-DBL-200 Statement stepping no more rows.

\paragraph{LR-DBL-140 Statement step invalid}
Stepping an invalid statement shall raise an exception.

\textit{Tested by: } LT-DBL-220 Statement stepping invalid.

\paragraph{LR-DBL-150 Statement step return value}
The stepping of a statement shall return true as long as results
are available for retrieval in the statement, and false when results
are no longer available.

\textit{Tested by: } LT-DBL-180 Statement stepping,
LT-DBL-190 Statement stepping second,
LT-DBL-200 Statement stepping no more rows.

\paragraph{LR-DBL-160 Statement step reset}
The stepping of a statement shall perform an automatic reset when
the end of available results is reached.

\textit{Tested by: } LT-DBL-210 Statement stepping reset.

\paragraph{LR-DBL-170 Statement parameters binding}
The statements shall allow binding positional parameters of type
integer and string.

\textit{Tested by: } LT-DBL-230 Statement binding integer,
LT-DBL-240 Statement binding string,
LT-DBL-410 Store and query negative integers,
LT-DBL-420 Store and query integers larger than 32 bits.

\paragraph{LR-DBL-180 Statement column return}
The statements shall allow retrieval of result columns from rows.
The results may be of type boolean, integer or string.

\textit{Tested by: } LT-DBL-260 Statement return boolean,
LT-DBL-270 Statement return integer,
LT-DBL-280 Statement return string,
LT-DBL-410 Store and query negative integers,
LT-DBL-420 Store and query integers larger than 32 bits.

\paragraph{LR-DBL-190 Statement return shortcut}
Statements shall allow a shortcut operation which, in order, steps
the statement, retrieves all the result columns of the first returned
row, and then reset the statement.

\textit{Tested by: } LT-DBL-290 Statement return shortcut,
LT-DBL-300 Statement return shortcut reset.

\paragraph{LR-DBL-200 Statement binding too many parameters}
Statement parameter binding shall throw an exception if
too many parameters are given.

\textit{Tested by: } LT-DBL-250 Binding too many parameters.

\paragraph{LR-DBL-210 Statement column return wrong number}
The statement column retrieval shall throw an exception if too many
columns are asked for.

\textit{Tested by: } LT-DBL-310 Statement column return wrong number of strings,
LT-DBL-320 Statement column return wrong number of booleans,
LT-DBL-330 Statement column return wrong number of integers.

\paragraph{LR-DBL-220 Statement column return wrong type}
The statement column retrieval shall throw an exception if the
queried type does not match the returned type.

\textit{Tested by: } LT-DBL-340 Statement column return wrong type integer,
LT-DBL-350 Statement column return wrong type boolean,
LT-DBL-360 Statement column return wrong type string.

\paragraph{LR-DBL-230 Statement column return no rows}
The statement column retrieval shall throw an exception if
there are no rows to give as result.

\textit{Tested by: } LT-DBL-370 Statement column return no rows.

\paragraph{LR-DBL-240 Statement column return null}
The statement column retrieval shall return the following
default values for each queried type in case the DB returns
NULL as a result.
\begin{enumerate}
\item Boolean: false,
\item Integer: 0,
\item String: the empty string \lstinline{""}.
\end{enumerate}

\textit{Tested by: } LT-DBL-380 Statement column return null boolean,
LT-DBL-390 Statement column return null integer,
LT-DBL-400 Statement column return null string.

\subsection{ncurses\textunderscore lib}
\paragraph{LR-NCU-010}
TODO: uplink

\subsection{suggestion\textunderscore lib}
The suggestion library provides a heuristic for selecting a suggestion
based on a set of unique strings and a query string. The suggestion
must feel natural to the user, this is achieved by combining three
elementary heuristics in a cascade:
\begin{enumerate}
\item Substring matching,
\item Substring matching ignoring case,
\item Fuzzy matching.
\end{enumerate}

\paragraph{LR-SUG-010 -- Overall heuristic}
Given a set of unique strings to pick from, and a query string,
the suggestion matcher shall return the string chosen from the set
by the following heuristics,
\begin{enumerate}
\item The string returned by the substring matcher, if any,
\item Otherwise, the string returned by the substring matcher
      with case ignored in all strings, if any,
\item Otherwise, the string returned by the fuzzy matcher,
\item Otherwise, the empty string is returned.
\end{enumerate}

\textit{Tested by: } LT-SUG-040 Prefix match without case,
LT-SUG-110 No match.

\paragraph{LR-SUG-020 -- Empty choice set}
Given an empty set of strings to pick from, and a query string,
the suggestion matcher shall return the empty string.

\textit{Tested by: } LT-SUG-010 Empty set of choice strings.

\paragraph{LR-SUG-030 -- Empty query string}
Given an empty query string and any set of strings to pick from,
the suggestion matcher shall return the empty string.

\textit{Tested by: } LT-SUG-020 Empty query string.

\paragraph{LR-SUG-040 -- Minimal character set}
The suggestion matcher shall work nominally for strings containing
any of the characters in the C++ \emph{basic character set} as
defined in [AD6].

\textit{Tested by: } LT-SUG-120 Non-letters.

\paragraph{LR-SUG-050 -- Ignoring case behavior}
Ignoring the case when using the substring matcher shall apply
to alphabet letters only. The other characters are treated
in the same fashion whether case is ignored or not.

\textit{Tested by: } LT-SUG-120 Non-letters.

\subsubsection{Substring matcher}
\paragraph{LR-SUG-060 -- Substring matcher heuristic}
Given a set of unique strings to pick from, and a query string,
the substring matcher shall return a string containing the
exact query string as a substring.

For instance, the substring \lstinline{ana} is found in
the string \lstinline{banana}.

\textit{Tested by: } LT-SUG-030 Prefix match with case,
LT-SUG-050 Base substring case.

\paragraph{LR-SUG-070 -- Earliest substring occurence}
In case multiple strings are found with the method described
in LR-SUG-060, a string containing the earliest occurence
of the substring shall be returned.

For instance, given the set \lstinline{bbanana, banana}, and the
query substring \lstinline{ana}, the string \lstinline{banana}
must be returned since it contains the substring at the earliest
position.

\textit{Tested by: } LT-SUG-060 Earliest substring.

\paragraph{LR-SUG-080 -- Shortest substring match}
In case multiple strings are found with the method described
in LR-SUG-070, a string among the shortest shall be returned.

For instance, given the set \lstinline{banana, bananana}, and
the query substring \lstinline{ana}, the string \lstinline{banana}
must be returned.

\textit{Tested by: } LT-SUG-070 Shortest substring.

\paragraph{LR-SUG-090 -- Substring results ordering}
In case multiple strings are found with the method described
in LR-SUG-080, the string which comes first in alphabetical order
(extended to an arbitrary ordering for characters outside of letters)
shall be returned. Note that since the items in the string set are unique,
this guarantees only at most one string matches.

For instance, given the set \lstinline{banana, banane}, and the query
substring \lstinline{ana}, the string \lstinline{banana} must be
returned.

\textit{Tested by: } LT-SUG-080 First alphabetical substring.

\subsubsection{Fuzzy matcher}
The fuzzy matcher is less strictly specified than the other matchers. However,
some important properties must be met.

\paragraph{LR-SUG-100 -- Fuzzy matcher heuristic}
The fuzzy matcher shall ensure out of order queries are still matched.

For instance, given the set \lstinline{"The brown fox", "jumps over the fence"},
and the query \lstinline{"fence over jumps"}, the fuzzy matcher must
return \lstinline{"jumps over the fence"}.

\textit{Tested by: } LT-SUG-090 Out of order.

\paragraph{LR-SUG-110 -- Fuzzy matching small typo}
The fuzzy matcher shall exhibit the fuzzy property. It must match
items despite small typos and change of case
which forbid exact matches.

For instance, given the set \lstinline{Jupiter, Saturn}, and the
query string \lstinline{juppiter}, it must return \lstinline{Jupiter}.

\textit{Tested by: } LT-SUG-100 Small typo.

\paragraph{LR-SUG-120 -- Fuzzy matching no characters in common}
The fuzzy matcher shall not return a match if the query string
has no character in common with any of the strings in the search
set.

For instance, given the set \lstinline{banana, fox} and the query
string \lstinline{yyy}, the fuzzy matcher must match nothing.

\textit{Tested by: } LT-SUG-110 No match.
