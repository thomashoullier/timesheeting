\section{Libraries requirements}
\subsection{config\textunderscore lib}
\subsubsection{Utilities}
\paragraph{LR-CON-010 -- Tilde expansion utility}
The config\textunderscore lib shall provide a utility function for tilde
expansion of filepaths. It replaces eventual leading \lstinline{~/}
with \lstinline{$HOME} in filepaths. It does nothing on filepaths
without tilde.

\textit{Uplink: } TODO.

\paragraph{LR-CON-020 -- First existing file}
The config\textunderscore lib shall provide a utility function, which,
given a filepath suffix and a list of folders, returns the first found
existing filepath by trying the suffix over the successive folder
candidates. It returns nothing if no possible file exists.

\textit{Uplink: } TODO.

\subsubsection{Configuration loader}
\paragraph{LR-CON-030 -- Configuration file format}
The config\textunderscore lib file loader shall use the TOML file format [AD3]
for the configuration file.

\textit{Uplink: } TODO.

\paragraph{LR-CON-040 -- String loading}
The configuration file loader shall allow reading parameters of type
\emph{string}.

\textit{Uplink: } TODO.

\paragraph{LR-CON-050 -- String empty case}
The configuration file loader shall emit an exception if a loaded
\emph{string} is empty (\textit{ie} \lstinline{""}).

\textit{Uplink: } TODO.

\paragraph{LR-CON-060 -- Filepath loading}
The configuration file loader shall allow reading parameters of type
\emph{filepath} with automatic tilde expansion.

\textit{Uplink: } TODO.

\paragraph{LR-CON-070 -- Filepath parent non-existent}
The configuration file loader shall emit an exception if the loaded
\emph{filepath} direct parent does not exist.

\textit{Uplink: } TODO.

\paragraph{LR-CON-080 -- Float loading}
The configuration file loader shall allow reading parameters of type
\emph{float}.

\textit{Uplink: } TODO.

\paragraph{LR-CON-090 -- Parameter empty case}
The configuration file loader shall emit an exception if any parameter
is empty, \textit{ie} \lstinline{parameter = }.

\textit{Uplink: } TODO.

\paragraph{LR-CON-100 -- Vector of strings loading}
The configuration file loader shall allow reading parameters of type
\emph{vector of strings}. The order of strings in the vector is preserved.

\textit{Uplink: } TODO.

\paragraph{LR-CON-110 -- Vector of non-strings case}
The configuration file loader shall emit an exception if a loaded
\emph{vector of strings} does not in fact contain strings.

\textit{Uplink: } TODO.

\paragraph{LR-CON-120 -- Configuration file nonexistent}
The configuration file loader shall emit an exception if it is provided
a configuration filepath which does not exist.

\textit{Uplink: } TODO.

\subsection{time\textunderscore lib}
\subsubsection{time\textunderscore zone}
The time\textunderscore zone object is a singleton which provides the
timezone information to the rest of the program.

\paragraph{LR-TMZ-010 -- Time zone initialization}
The timezone object shall be initialized with a valid TZ identifier string,
as defined in [AD4].

TODO: uplink
\paragraph{LR-TMZ-020 -- Time zone singleton}
The timezone object shall be a singleton. It is initialized once and the same
instance is retrieved through a \lstinline{get} method subsequently.

TODO: uplink
\paragraph{LR-TMZ-030 -- Invalid time zone}
The timezone object shall emit an exception if it is initialized with
an invalid TZ string (\textit{ie} a string outside of those defined in [AD4]).

TODO: uplink
\paragraph{LR-TMZ-040 -- Time zone name}
The timezone class shall provide a method to retrieve its TZ identifier string.

TODO: uplink

\paragraph{LR-TMZ-050 -- std time\textunderscore zone}
The timezone class shall provide a method to retrieve the
\lstinline{std::chrono::time_zone} representation corresponding to the set
time zone.

TODO: uplink

\subsubsection{date}
A Date is a representation of a time point in UTC using the system clock
as time reference.

\paragraph{LR-DAT-010 -- Current time initialization}
The Date shall allow initialization at the current time.

TODO: uplink

\paragraph{LR-DAT-020 -- std::time\textunderscore point initialization}
The Date shall allow initialization from a \lstinline{std::time_point}.

TODO: uplink

\paragraph{LR-DAT-030 -- Beginning of year}
The Date shall allow initialization at the beginning of the current year
in the current time zone (as set in the TimeZone singleton).

TODO: uplink

\paragraph{LR-DAT-040 -- UNIX timestamp initialization}
The Date shall allow initialization from a UNIX timestamp in seconds.

TODO: uplink

\paragraph{LR-DAT-050 -- Date string initialization}
The Date shall allow initialization from a string in the format
\lstinline{%d%b%Y %H:%M:%S}, with format specifiers as defined in [AD5].
The date initialization uses time zone currently set
in the TimeZone singleton.

An example of a valid initialization string is \lstinline {19Jan2025 09:41:34}.

TODO: uplink

\paragraph{LR-DAT-060 -- Date string shortcuts}
The Date shall allow initialization from shortened strings
with formats,
\begin{itemize}
\item \lstinline{%d%b%Y %H:%M}
\item \lstinline{%d%b%Y %H}
\item \lstinline{%d%b%Y}
\end{itemize}
(with the same set of format specifiers as in LR-DAT-050),
replacing the omitted items from the full format defined in LR-DAT-050
by zero values.

Examples of shortened initialization strings and their full equivalent are
given in \cref{tab:date_shortened}.

\begin{table}
  \caption{\label{tab:date_shortened}
    Shortened date initialization string examples}
  \begin{tabular}{| c | c |} \hline
    \textbf{Shortened} & \textbf{Full} \\ \hline
    \texttt{19Jan2025 09:41} & \texttt{19Jan2025 09:41:00} \\ \hline
    \texttt{19Jan2025 09} & \texttt{19Jan2025 09:00:00} \\ \hline
    \texttt{19Jan2025} & \texttt{19Jan2025 00:00:00} \\ \hline
  \end{tabular}\end{table}

TODO: uplink

\paragraph{LR-DAT-070 -- Date string invalid}
During Date initialization with a date string, an exception shall be
thrown if an invalid date string is used (\textit{ie} outside of the formats
specified).

TODO: uplink

\paragraph{LR-DAT-075 -- Date std::time\textunderscore point access}
The Date shall allow read access to its internal
\lstinline{std::chrono::time_point} representation.

TODO: uplink

\paragraph{LR-DAT-080 -- Date output string}
The Date shall be convertible to a string in format
\lstinline{%d%b%Y %H:%M:%S} (see [AD5]) in the current time zone.

TODO: uplink
\paragraph{LR-DAT-090 -- Date output hours/minutes}
The Date shall be convertible to a string in format
\lstinline{%H:%M} (see [AD5]) in the current time zone.

TODO: uplink
\paragraph{LR-DAT-100 -- Date output unambiguous string}
The Date shall be convertible to a string in format
\lstinline{%d%b%Y %H:%M:%S %z} (see [AD5]) in the current time zone.

TODO: uplink
\paragraph{LR-DAT-110 -- Date output UNIX timestamp}
The Date shall be convertible to a UNIX timestamp in seconds in \gls{UTC}.

TODO: uplink
\paragraph{LR-DAT-120 -- Date output day/month/year}
The Date shall be convertible to a string in format
\lstinline{%d%b%Y} (see [AD5]) in the current time zone.

TODO: uplink
\paragraph{LR-DAT-130 -- Second resolution}
The Date shall represent time points with a resolution of at most 1 second.

TODO: uplink

\paragraph{LR-DAT-140 -- Date comparison}
The Date class shall provide \emph{lesser than} and \emph{greater than}
comparison operators.

TODO: uplink

\subsubsection{DateRange}
A DateRange represents a range between a start Date and a stop Date.

\paragraph{LR-DTR-010 -- DateRange initialization}
The DateRange shall be initialized using a start Date and a stop Date.

TODO: uplink

\paragraph{LR-DTR-020 -- DateRange ordering}
The DateRange initialization shall emit an exception if the
start Date is \emph{greater than} the stop Date.

TODO: uplink

\paragraph{LR-DTR-030 -- Dates read access}
The DateRange shall allow read access to the start and stop Date.

TODO: uplink
\paragraph{LR-DTR-040 -- DateRange to strings}
The DateRange shall be convertible to a vector of two Date strings
as defined in LR-DAT-080.

TODO: uplink
\paragraph{LR-DTR-050 -- DateRange to day strings}
The DateRange shall be convertible to a vector of two Date day strings
as defined in LR-DAT-120.

TODO: uplink

\subsection{log\textunderscore lib}
\paragraph{LR-LOG-010}
TODO: uplink

\subsection{db\textunderscore lib}
\paragraph{LR-DBL-010}
TODO: uplink

\subsection{ncurses\textunderscore lib}
\paragraph{LR-NCU-010}
TODO: uplink

\subsection{suggestion\textunderscore lib}
\paragraph{LR-SUG-010}
TODO: uplink
