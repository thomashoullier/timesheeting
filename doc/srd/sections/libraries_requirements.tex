\section{Libraries requirements}
\subsection{config\textunderscore lib}
\subsubsection{Utilities}
\paragraph{LR-CON-010 -- Tilde expansion utility}
The config\textunderscore lib shall provide a utility function for tilde
expansion of filepaths. It replaces eventual leading \lstinline{~/}
with \lstinline{$HOME} in filepaths. It does nothing on filepaths
without tilde.

\textit{Uplink: } TODO.

\paragraph{LR-CON-020 -- First existing file}
The config\textunderscore lib shall provide a utility function, which,
given a filepath suffix and a list of folders, returns the first found
existing filepath by trying the suffix over the successive folder
candidates. It returns nothing if no possible file exists.

\textit{Uplink: } TODO.

\subsubsection{Configuration loader}
\paragraph{LR-CON-030 -- Configuration file format}
The config\textunderscore lib file loader shall use the TOML file format [AD3]
for the configuration file.

\textit{Uplink: } TODO.

\paragraph{LR-CON-040 -- String loading}
The configuration file loader shall allow reading parameters of type
\emph{string}.

\textit{Uplink: } TODO.

\paragraph{LR-CON-050 -- String empty case}
The configuration file loader shall emit an exception if a loaded
\emph{string} is empty (\textit{ie} \lstinline{""}).

\textit{Uplink: } TODO.

\paragraph{LR-CON-060 -- Filepath loading}
The configuration file loader shall allow reading parameters of type
\emph{filepath} with automatic tilde expansion.

\textit{Uplink: } TODO.

\paragraph{LR-CON-070 -- Filepath parent non-existent}
The configuration file loader shall emit an exception if the loaded
\emph{filepath} direct parent does not exist.

\textit{Uplink: } TODO.

\paragraph{LR-CON-080 -- Float loading}
The configuration file loader shall allow reading parameters of type
\emph{float}.

\textit{Uplink: } TODO.

\paragraph{LR-CON-090 -- Parameter empty case}
The configuration file loader shall emit an exception if any parameter
is empty, \textit{ie} \lstinline{parameter = }.

\textit{Uplink: } TODO.

\paragraph{LR-CON-100 -- Vector of strings loading}
The configuration file loader shall allow reading parameters of type
\emph{vector of strings}. The order of strings in the vector is preserved.

\textit{Uplink: } TODO.

\paragraph{LR-CON-110 -- Vector of non-strings case}
The configuration file loader shall emit an exception if a loaded
\emph{vector of strings} does not in fact contain strings.

\textit{Uplink: } TODO.

\paragraph{LR-CON-120 -- Configuration file nonexistent}
The configuration file loader shall emit an exception if it is provided
a configuration filepath which does not exist.

\textit{Uplink: } TODO.

\subsection{time\textunderscore lib}
\subsubsection{time\textunderscore zone}
The time\textunderscore zone object is a singleton which provides the
timezone information to the rest of the program.

\paragraph{LR-TMZ-010 -- Time zone initialization}
The timezone object shall be initialized with a valid TZ identifier string,
as defined in [AD4].

TODO: uplink
\paragraph{LR-TMZ-020 -- Time zone singleton}
The timezone object shall be a singleton. It is initialized once and the same
instance is retrieved through a \lstinline{get} method subsequently.

TODO: uplink
\paragraph{LR-TMZ-030 -- Invalid time zone}
The timezone object shall emit an exception if it is initialized with
an invalid TZ string (\textit{ie} a string outside of those defined in [AD4]).

TODO: uplink
\paragraph{LR-TMZ-040 -- Time zone name}
The timezone class shall provide a method to retrieve its TZ identifier string.

TODO: uplink

\paragraph{LR-TMZ-050 -- std time\textunderscore zone}
The timezone class shall provide a method to retrieve the
\lstinline{std::chrono::time_zone} representation corresponding to the set
time zone.

TODO: uplink

\subsubsection{date}
A Date is a representation of a time point in UTC using the system clock
as time reference.

\paragraph{LR-DAT-010 -- Current time initialization}
The Date shall allow initialization at the current time.

TODO: uplink

\paragraph{LR-DAT-020 -- std::time\textunderscore point initialization}
The Date shall allow initialization from a \lstinline{std::time_point}.

TODO: uplink

\paragraph{LR-DAT-030 -- Beginning of year}
The Date shall allow initialization at the beginning of the current year
in the current time zone (as set in the TimeZone singleton).

TODO: uplink

\paragraph{LR-DAT-040 -- UNIX timestamp initialization}
The Date shall allow initialization from a UNIX timestamp in seconds.

TODO: uplink

\paragraph{LR-DAT-050 -- Date string initialization}
The Date shall allow initialization from a string in the format
\lstinline{%d%b%Y %H:%M:%S}, with format specifiers as defined in [AD5].
The date initialization uses time zone currently set
in the TimeZone singleton.

An example of a valid initialization string is \lstinline {19Jan2025 09:41:34}.

TODO: uplink

\paragraph{LR-DAT-060 -- Date string shortcuts}
The Date shall allow initialization from shortened strings
with formats,
\begin{itemize}
\item \lstinline{%d%b%Y %H:%M}
\item \lstinline{%d%b%Y %H}
\item \lstinline{%d%b%Y}
\end{itemize}
(with the same set of format specifiers as in LR-DAT-050),
replacing the omitted items from the full format defined in LR-DAT-050
by zero values.

Examples of shortened initialization strings and their full equivalent are
given in \cref{tab:date_shortened}.

\begin{table}
  \caption{\label{tab:date_shortened}
    Shortened date initialization string examples}
  \begin{tabular}{| c | c |} \hline
    \textbf{Shortened} & \textbf{Full} \\ \hline
    \texttt{19Jan2025 09:41} & \texttt{19Jan2025 09:41:00} \\ \hline
    \texttt{19Jan2025 09} & \texttt{19Jan2025 09:00:00} \\ \hline
    \texttt{19Jan2025} & \texttt{19Jan2025 00:00:00} \\ \hline
  \end{tabular}\end{table}

TODO: uplink

\paragraph{LR-DAT-070 -- Date string invalid}
During Date initialization with a date string, an exception shall be
thrown if an invalid date string is used (\textit{ie} outside of the formats
specified).

TODO: uplink

\paragraph{LR-DAT-075 -- Date std::time\textunderscore point access}
The Date shall allow read access to its internal
\lstinline{std::chrono::time_point} representation.

TODO: uplink

\paragraph{LR-DAT-080 -- Date output string}
The Date shall be convertible to a string in format
\lstinline{%d%b%Y %H:%M:%S} (see [AD5]) in the current time zone.

TODO: uplink
\paragraph{LR-DAT-090 -- Date output hours/minutes}
The Date shall be convertible to a string in format
\lstinline{%H:%M} (see [AD5]) in the current time zone.

TODO: uplink
\paragraph{LR-DAT-100 -- Date output unambiguous string}
The Date shall be convertible to a string in format
\lstinline{%d%b%Y %H:%M:%S %z} (see [AD5]) in the current time zone.

TODO: uplink
\paragraph{LR-DAT-110 -- Date output UNIX timestamp}
The Date shall be convertible to a UNIX timestamp in seconds in \gls{UTC}.

TODO: uplink
\paragraph{LR-DAT-120 -- Date output day/month/year}
The Date shall be convertible to a string in format
\lstinline{%d%b%Y} (see [AD5]) in the current time zone.

TODO: uplink
\paragraph{LR-DAT-130 -- Second resolution}
The Date shall represent time points with a resolution of at most 1 second.

TODO: uplink

\paragraph{LR-DAT-140 -- Date comparison}
The Date class shall provide \emph{lesser than} and \emph{greater than}
comparison operators.

TODO: uplink

\subsubsection{DateRange}
A DateRange represents a range between a start Date and a stop Date.

\paragraph{LR-DTR-010 -- DateRange initialization}
The DateRange shall be initialized using a start Date and a stop Date.

TODO: uplink

\paragraph{LR-DTR-020 -- DateRange ordering}
The DateRange initialization shall emit an exception if the
start Date is \emph{greater than} the stop Date.

TODO: uplink

\paragraph{LR-DTR-030 -- Dates read access}
The DateRange shall allow read access to the start and stop Date.

TODO: uplink
\paragraph{LR-DTR-040 -- DateRange to strings}
The DateRange shall be convertible to a vector of two Date strings
as defined in LR-DAT-080.

TODO: uplink
\paragraph{LR-DTR-050 -- DateRange to day strings}
The DateRange shall be convertible to a vector of two Date day strings
as defined in LR-DAT-120.

TODO: uplink

\subsubsection{Day}
A Day corresponds to a DateRange covering a single calendar day in
the current TimeZone. A given day starts at midnight and ends at the next
midnight.

\paragraph{LR-DAY-010 -- Now initialization}
The default Day initialization shall be to the current calendar day,
as indicated by the system clock, in the time zone currently set
in TimeZone.

TODO: uplink

\paragraph{LR-DAY-020 -- year/month/day initialization}
Day shall allow initialization from a \lstinline{std::chrono::year_month_day}
object, which represents a calendar day.

TODO: uplink

\paragraph{LR-DAY-030 -- DateRange representation}
The Day shall be convertible to a corresponding DateRange.

TODO: uplink

\paragraph{LR-DAY-040 -- DateRange start and stop Date}
The DateRange obtained from a Day shall have a start Date set
to \lstinline{00:00:00} and stop Date set to \lstinline{00:00:00} of the
following day, in the currently set time zone and for the currently set calendar
day.

TODO: uplink

\paragraph{LR-DAY-050 -- String representation}
The Day shall be convertible to a string representation in format
\lstinline{%d%b%Y %a} (see [AD5]).

For instance, \lstinline{21Jan2025 Tue} is a valid string representation
for a Day.

TODO: uplink

\paragraph{LR-DAY-060 -- Next and previous}
Day shall include methods to select the \emph{next} and \emph{previous} days
from the currently set calendar day.

For instance, if the current Day is set to \lstinline{21Jan2025}, calling
\emph{previous} must change the Day to \lstinline{20Jan2025}.

TODO: uplink

\subsubsection{Week}
A \emph{Week} represents a DateRange from a monday on midnight to midnight of
the next monday, for a particular calendar week.
The dates are as currently defined in the TimeZone.

\paragraph{LR-WEK-010 -- Now initialization}
The Week shall allow initialization at the current calendar week, as defined by
the system clock.

TODO: uplink

\paragraph{LR-WEK-020 -- Date initialization}
The Week shall allow initialization to a week containing a given Date.

TODO: uplink

\paragraph{LR-WEK-030 -- DateRange conversion}
The Week shall be convertible to a DateRange. The start Date of the DateRange
is set to midnight of the monday of the week currently pointed to, and the
stop Date is at midnight on the next monday.

TODO: uplink

\paragraph{LR-WEK-040 -- String representation}
The Week shall be convertible to a string representation in the format
\lstinline{%d%b%Y W%V} (see [AD5]), with the day string \lstinline{%d%b%Y}
set to the monday of the week.

For instance, \lstinline{20Jan2025 W04} is a valid string representation.

TODO: uplink

\paragraph{LR-WEK-050 -- Array of days}
The Week shall be convertible to an array of all the Days contained
in the week, in order from monday to sunday.

TODO: uplink
  
\paragraph{LR-WEK-060 -- Next and previous}
The Week shall include methods to select the \emph{next} and \emph{previous}
weeks from the currently set calendar week.

TODO: uplink

\subsubsection{Duration}
A Duration is a length of time. It does not have a particular location
in terms of date.

\paragraph{LR-DUR-010 -- Zero initialization}
The Duration shall default initialize to a duration of zero.

TODO: uplink
\paragraph{LR-DUR-020 -- Seconds initialization}
The Duration shall allow initialization from a count of seconds.

TODO: uplink
\paragraph{LR-DUR-030 -- Duration resolution}
The Duration shall have a resolution of at most one second.

TODO: uplink
\paragraph{LR-DUR-040 -- Seconds string}
The Duration shall be convertible to a string in format
\lstinline{"x seconds"}, with \lstinline{x} the seconds count
for the duration.

For instance \lstinline{"104592 seconds"} is a valid string.

TODO: uplink
\paragraph{LR-DUR-050 -- Hours string}
The Duration shall be convertible to a string in format
\lstinline{"x hours"}, with \lstinline{x} the hours count
for the duration at three decimal places of precision.

For instance \lstinline{"29.053 hours"} is a valid string.

TODO: uplink
\paragraph{LR-DUR-060 -- Days string}
The Duration shall be convertible to a string in format
\lstinline{"x days"}, with \lstinline{x} the days count
for the duration at three decimal places of precision.
The number of hours in a day must be provided to the
computation externally.

For instance \lstinline{"3.773 days"} is a valid string.

TODO: uplink
\paragraph{LR-DUR-070 -- Seconds short string}
The Duration shall be convertible to a shortened string which
is the one in LR-DUR-040 without the \lstinline{" seconds"}
part.

For instance, \lstinline{"104592"} is a valid string.

TODO: uplink
\paragraph{LR-DUR-080 -- Hours short string}
The Duration shall be convertible to a shortened string
which is the one from LR-DUR-050 without the \lstinline{" hours"}
part.

For instance \lstinline{"29.053"} is a valid string.

TODO: uplink
\paragraph{LR-DUR-090 -- Days short string}
The Duration shall be convertible to a shortened string
which is the one from LR-DUR-060 without the \lstinline{" days"}
part.

For instance \lstinline{"3.773"} is a valid string.

TODO: uplink
\paragraph{LR-DUR-100 -- Zero second short string}
A Duration with a value of 0 shall return a single whitespace
\lstinline{" "} as a second short string.

TODO: uplink
\paragraph{LR-DUR-110 -- Zero hour short string}
A Duration with a value of 0 shall return a single whitespace
\lstinline{" "} as a hour short string.

TODO: uplink
\paragraph{LR-DUR-120 -- Zero day short string}
A Duration with a value of 0 shall return a single whitespace
\lstinline{" "} as a day short string.

TODO: uplink
\paragraph{LR-DUR-130 -- Day string invalid hours}
When given a count of hours in a day which is lesser than or equal to zero,
the day string method shall throw an exception.

TODO: uplink
\paragraph{LR-DUR-140 -- Day short string invalid hours}
When given a count of hours in a day which is lesser than or equal to zero,
the day short string method shall throw an exception.

TODO: uplink
\subsubsection{Duration displayer}
The DurationDisplayer is a singleton for displaying a Duration according
to a currently selected format.

\paragraph{LR-DRD-010 -- DurationDisplayer initialization pass}
The DurationDisplayer shall be initialized with a number of work hours
in a day.

TODO:uplink
\paragraph{LR-DRD-020 -- DurationDisplayer singleton}
The DurationDisplayer shall be a singleton. It is initialized onec and
the same instance is retrieved through a \lstinline{get} method
subsequently.

TODO:uplink
\paragraph{LR-DRD-030 -- Display formats}
The DurationDisplayer shall allow selecting the formats: \emph{hours},
\emph{days}, \emph{seconds}.
These refer to the second, hour and day strings for Duration.

TODO:uplink
\paragraph{LR-DRD-040 -- Default format}
The default format for displaying a Duration upon initialization shall
be \emph{hours}.

TODO:uplink
\paragraph{LR-DRD-050 -- Format cycling}
The currently selected format of the DurationDisplayer shall be selected
through a \emph{cycle} method.

The cycle between formats is as follows:
\begin{itemize}
\item hours $\rightarrow$ days
\item days $\rightarrow$ seconds
\item seconds $\rightarrow$ hours
\end{itemize}

TODO:uplink
\paragraph{LR-DRD-060 -- String display}
Given a Duration, the DurationDisplayer shall output a string depending
on the currently selected format, as follows,
\begin{itemize}
\item hours: string from LR-DUR-050,
\item days: string from LR-DUR-060 with number of hours as initialized in the
            DurationDisplayer,
\item seconds: string from LR-DUR-040.
\end{itemize}

TODO: uplink
\paragraph{LR-DRD-070 -- Short string display}
Given a Duration, the DurationDisplayer shall output a short string depending
on the currently selected format, as follows,
\begin{itemize}
\item hours: string from LR-DUR-080,
\item days: string from LR-DUR-090 with number of hours as initialized in the
            DurationDisplayer,
\item seconds: string from LR-DUR-070.
\end{itemize}

TODO:uplink
\subsection{log\textunderscore lib}
\paragraph{LR-LOG-010}
TODO: uplink

\subsection{db\textunderscore lib}
\paragraph{LR-DBL-010}
TODO: uplink

\subsection{ncurses\textunderscore lib}
\paragraph{LR-NCU-010}
TODO: uplink

\subsection{suggestion\textunderscore lib}
\paragraph{LR-SUG-010}
TODO: uplink
