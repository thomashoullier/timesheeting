\section{Modules requirements}
\subsection{version}
The \emph{version} module configures the versions of program components,
to ensure compatibility between different executables and \gls{DB}.

\paragraph{MR-VER-010 -- Program version number}
The \emph{version} module shall provide a program version number
as a string in the following format: \lstinline{x.y},
with \lstinline{x} the major revision number and \lstinline{y} the
minor revision number. The major version number and minor version
number are both integers. Zero is an acceptable version number.

The program version number may be suffixed with \lstinline{dev} to
indicate an internal development version.

Valid program version number examples are:
\begin{itemize}
\item \lstinline{0.5}: for major version 0, minor version 5.
\item \lstinline{3.1}: for major version 3, minor version 1.
\item \lstinline{3.1dev}: for a development version after \lstinline{3.1}.
\end{itemize}

\textit{uplink: } R-REL-010 -- Software version format

\paragraph{MR-VER-020 -- DB version number}
The \emph{version} module shall provide a \gls{DB} version number as
a positive integer.

\textit{uplink: } TODO DB module.

\subsection{cli}
\paragraph{MR-CLI-010}
TODO: uplink

\subsection{core}
\paragraph{MR-COR-010}
TODO: uplink

\subsection{db}
\paragraph{MR-DBI-010}
TODO: uplink

\subsection{keys}
\paragraph{MR-KEY-010}
TODO: uplink

\subsection{tui}
\paragraph{MR-TUI-010}
TODO: uplink

\subsection{exporter}
\paragraph{MR-EXP-010}
TODO: uplink
