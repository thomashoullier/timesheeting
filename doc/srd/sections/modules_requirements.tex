\section{Modules requirements}
\subsection{version}
The \emph{version} module configures the versions of program components,
to ensure compatibility between different executables and \gls{DB}.

\paragraph{MR-VER-010 -- Program version number}
The \emph{version} module shall provide a program version number
as a string in the following format: \lstinline{x.y},
with \lstinline{x} the major revision number and \lstinline{y} the
minor revision number. The major version number and minor version
number are both integers. Zero is an acceptable version number.

The program version number may be suffixed with \lstinline{dev} to
indicate an internal development version.

Valid program version number examples are:
\begin{itemize}
\item \lstinline{0.5}: for major version 0, minor version 5.
\item \lstinline{3.1}: for major version 3, minor version 1.
\item \lstinline{3.1dev}: for a development version after \lstinline{3.1}.
\end{itemize}

\textit{uplink: } R-REL-010 -- Software version format

\paragraph{MR-VER-020 -- DB version number}
The \emph{version} module shall provide a \gls{DB} version number as
a positive integer.

\textit{uplink: } TODO DB module.

\subsection{config}
The \emph{config} module loads the configuration file, translates the parameters
into an internal representation and transmits them to the rest of the program.

\subsubsection{Configuration file loading}

\paragraph{MR-CON-010 -- Load from path}
The \emph{config} module shall load any compliant user configuration file when
provided with its path.

\textit{uplink: } TODO.

\paragraph{MR-CON-020 -- Non-existent path}
The \emph{config} module shall emit an exception when the path provided for the
configuration file corresponds to a file which does not exist.

\textit{uplink: } TODO.

\paragraph{MR-CON-030 -- Default configuration file locations}
The \emph{config} module shall provide a default load method which looks
for the configuration file \lstinline{timesheeting/timesheeting.toml}
in the following default locations, in order of preference,
\begin{enumerate}
  \item \lstinline{$XDG_CONFIG_HOME}
  \item \lstinline{$HOME/.config/}
  \item \lstinline{/etc/}
\end{enumerate}

\textit{uplink: } TODO.

\paragraph{MR-CON-040 -- Default configuration file not found}
The \emph{config} module default load method shall emit an exception if
no configuration file was found in any of the default locations.

\textit{uplink: } TODO.

\subsubsection{Logger configuration}
\paragraph{MR-CON-050 -- Log filepath}
The \emph{config} module shall load from the configuration file a filepath
for the log file. This parameter is at the node \lstinline{log.file}.

\textit{uplink: } TODO.

\paragraph{MR-CON-060 -- Log levels}
The \emph{config} module shall load from the configuration file a vector of
strings defining the active log levels. This parameter is at the node
\lstinline{log.active_levels}.

\textit{uplink: } TODO.

\subsubsection{DB configuration}
\paragraph{MR-CON-070 -- DB filepath}
The \emph{config} module shall load from the configuration file a filepath
for the DB file. This parameter is at the node \lstinline{db.file}.

\textit{uplink: } TODO.

\subsubsection{Timezone configuration}
\paragraph{MR-CON-080 -- Timezone string}
The \emph{config} module shall load from the configuration file a string
representing the timezone to use. This parameter is at the node
\lstinline{time.timezone}.

\textit{uplink: } TODO.

\subsubsection{Duration display configuration}
\paragraph{MR-CON-090 -- Duration display days}
THe \emph{config} module shall load from the configuration file a float
representing the number of hours in a workday for duration display.
This parameter is at the node \lstinline{time.hours_per_workday}.

\textit{uplink: } TODO.

\subsubsection{Key bindings configuration}
TODO:
* Parsing of each key binding (maybe do a table)
* Parsing primary and secondary
* Parsing special keys

\subsection{cli}
\paragraph{MR-CLI-010}
TODO: uplink

\subsection{core}
\paragraph{MR-COR-010}
TODO: uplink

\subsection{db}
\paragraph{MR-DBI-010}
TODO: uplink

\subsection{keys}
\paragraph{MR-KEY-010}
TODO: uplink

\subsection{tui}
\paragraph{MR-TUI-010}
TODO: uplink

\subsection{exporter}
\paragraph{MR-EXP-010}
TODO: uplink
