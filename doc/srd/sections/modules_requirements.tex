\section{Modules requirements}
\subsection{version}
The \emph{version} module configures the versions of program components,
to ensure compatibility between different executables and \gls{DB}.

\paragraph{MR-VER-010 -- Program version number}
The \emph{version} module shall provide a program version number
as a string in the following format: \lstinline{x.y},
with \lstinline{x} the major revision number and \lstinline{y} the
minor revision number. The major version number and minor version
number are both integers. Zero is an acceptable version number.

The program version number may be suffixed with \lstinline{dev} to
indicate an internal development version.

Valid program version number examples are:
\begin{itemize}
\item \lstinline{0.5}: for major version 0, minor version 5.
\item \lstinline{3.1}: for major version 3, minor version 1.
\item \lstinline{3.1dev}: for a development version after \lstinline{3.1}.
\end{itemize}

\textit{uplink: } R-REL-010 -- Software version format

\paragraph{MR-VER-020 -- DB version number}
The \emph{version} module shall provide a \gls{DB} version number as
a positive integer.

\textit{uplink: } TODO DB module.

\subsection{config}
The \emph{config} module loads the configuration file, translates the parameters
into an internal representation and transmits them to the rest of the program.

\subsubsection{Configuration file loading}

\paragraph{MR-CON-010 -- Load from path}
The \emph{config} module shall load any compliant user configuration file when
provided with its path.

\textit{uplink: } TODO.

\paragraph{MR-CON-020 -- Non-existent path}
The \emph{config} module shall emit an exception when the path provided for the
configuration file corresponds to a file which does not exist.

\textit{uplink: } TODO.

\paragraph{MR-CON-030 -- Default configuration file locations}
The \emph{config} module shall provide a default load method which looks
for the configuration file \lstinline{timesheeting/timesheeting.toml}
in the following default locations, in order of preference,
\begin{enumerate}
  \item \lstinline{$XDG_CONFIG_HOME}
  \item \lstinline{$HOME/.config/}
  \item \lstinline{/etc/}
\end{enumerate}

\textit{uplink: } TODO.

\paragraph{MR-CON-040 -- Default configuration file not found}
The \emph{config} module default load method shall emit an exception if
no configuration file was found in any of the default locations.

\textit{uplink: } TODO.

\subsubsection{Logger configuration}
\paragraph{MR-CON-050 -- Log filepath}
The \emph{config} module shall load from the configuration file a filepath
for the log file. This parameter is at the node \lstinline{log.file}.

\textit{uplink: } TODO.

\paragraph{MR-CON-060 -- Log levels}
The \emph{config} module shall load from the configuration file a vector of
strings defining the active log levels. This parameter is at the node
\lstinline{log.active_levels}.

\textit{uplink: } TODO.

\paragraph{MR-CON-065 -- Maximum log age}
The \emph{config} module shall load from the configuration file an
unsigned integer defining the maximum age of logs (for log rotation)
in seconds. This parameter is at the node \lstinline{log.max_log_age}.

\textit{uplink: } TODO.

\subsubsection{DB configuration}
\paragraph{MR-CON-070 -- DB filepath}
The \emph{config} module shall load from the configuration file a filepath
for the DB file. This parameter is at the node \lstinline{db.file}.

\textit{uplink: } TODO.

\subsubsection{Timezone configuration}
\paragraph{MR-CON-080 -- Timezone string}
The \emph{config} module shall load from the configuration file a string
representing the timezone to use. This parameter is at the node
\lstinline{time.timezone}.

\textit{uplink: } TODO.

\subsubsection{Duration display configuration}
\paragraph{MR-CON-090 -- Duration display days}
The \emph{config} module shall load from the configuration file a float
representing the number of hours in a workday for duration display.
This parameter is at the node \lstinline{time.hours_per_workday}.

\textit{uplink: } TODO.

\subsubsection{Key bindings configuration}
The key bindings are loaded into maps from key to action. The maps are,
\begin{itemize}
\item The \emph{normal map}, which is the binding map for the normal mode,
\item The \emph{edit map}, which is the binding map for the edit mode.
\end{itemize}

The normal mode is the regular mode of operation of the program, entered
upon program launch. The edit mode is used when typing text.
For the purpose of organization of the user-facing configuration file,
the normal mode bindings are separated into the sections \lstinline{navigation}
and \lstinline{actions}, while the edit mode bindings are under the section
\lstinline{edit\_mode}.

\paragraph{MR-CON-100 -- Binding list}
The \emph{config} module shall load bindings from the following nodes,
organized hierarchically. The top level node is \lstinline{keys}.

The \emph{normal map} bindings are loaded from,
\begin{itemize}
\item navigation
  \begin{itemize}
  \item up
  \item down
  \item left
  \item right
  \item subtabs
  \item previous
  \item next
  \item duration\_display
  \item entries\_screen
  \item projects\_screen
  \item locations\_screen
  \item project\_report\_screen
  \item weekly\_report\_screen
  \item active\_visibility
  \item quit
  \end{itemize}
\item actions
  \begin{itemize}
  \item commit\_entry
  \item set\_now
  \item add
  \item rename
  \item remove
  \item active\_toggle
  \item task\_project\_change
  \end{itemize}
\end{itemize}

The \emph{edit map} bindings are loaded from,
\begin{itemize}
\item edit\_mode
  \begin{itemize}
  \item validate
  \item cancel
  \item select\_suggestion
  \end{itemize}
\end{itemize}

\textit{uplink: } TODO.

\paragraph{MR-CON-110 -- Regular keys}
The \emph{config} module shall allow binding any regular (\textit{ie}
non-escaped and non-special) character key to any binding action.
The regular keys are designated by their character in the configuration
file, \textit{ie} the key \lstinline{e} is designated by the character
\lstinline{"e"} in the configuration file.

\textit{uplink: } TODO.

\paragraph{MR-CON-120 -- Special keys}
The \emph{config} module shall allow binding the following special
keys to any binding action. The special keys are designated by strings.
\begin{itemize}
\item \lstinline{ESCAPE}: the escape key.
\item \lstinline{ENTER}: the enter or \emph{return} key.
\item \lstinline{SPACE}: the spacebar.
\item \lstinline{TAB}: the tabulation key.
\item \lstinline{UP}: the up arrow key.
\item \lstinline{DOWN}: the down arrow key.
\item \lstinline{LEFT}: the left arrow key.
\item \lstinline{RIGHT}: the right arrow key.
\end{itemize}

\textit{uplink: } TODO.

\paragraph{MR-CON-130 -- Multiple keys for an action}
The \emph{config} module shall allow binding multiple keys to the
same action. The keys are specified as a list of strings attached
to the corresponding action node.

For instance, for binding the keys \lstinline{e} and \lstinline{a}
to the action with node \lstinline{rename}, the configuration file
line is: \lstinline{rename = ["e", "a"]}.

\textit{uplink: } TODO.

\paragraph{MR-CON-140 -- Protection against duplicates}
The \emph{config} module shall emit an exception when the configuration
file contains duplicate keys bound in the same map (\textit{ie} either
\emph{normal} or \emph{edit}).

\textit{uplink: } TODO.

\paragraph{MR-CON-150 -- Backspace mapping}
The \emph{config} module shall map the backspace key to a backspace
action in the \emph{edit} map. This is done outside of the configuration
file, it is an implicit binding.

\textit{uplink: } TODO.

\paragraph{MR-CON-160 -- Unbound keys}
The maps in the \emph{config} module shall return an \emph{unbound}
action when queried with a key which was not bound during configuration.

\textit{uplink: } TODO.

\paragraph{MR-CON-170 -- Invalid key strings}
The \emph{config} module shall emit an exception when a key
string in the configuration file does not map to any regular or \emph{special}
characters.

\subsection{cli}
\paragraph{MR-CLI-010}
TODO: uplink

\subsection{core}
The \emph{core} module contains several objects used throughout the program.

\paragraph{MR-COR-010 -- Generic item}
The items Project, Task and Location shall all be represented as objects with
the following attributes,
\begin{enumerate}
\item id,
\item name,
\item active (boolean flag to indicate whether the item is active or not).
\end{enumerate}
TODO: uplink

\paragraph{MR-COR-020 -- Vector of generic item names}
A vector of generic items shall be convertible to a vector of their names,
in order.

\paragraph{MR-COR-030 -- Entry}
Entries shall be represented as objects with the following attributes,
\begin{enumerate}
\item id,
\item project name,
\item task name,
\item start date,
\item stop date,
\item location name.
\end{enumerate}

\paragraph{MR-COR-040 -- Entry to strings}
Entries shall be convertible to a vector of strings for display.
These strings are, in order,
\begin{enumerate}
\item project name,
\item task name,
\item start date display string (long format),
\item stop date display string (long format),
\item location name.
\end{enumerate}

\paragraph{MR-COR-050 -- Entry to short strings}
Entries shall be convertible to a vector of strings with the same composition
as in MR-COR-040, except the dates are outputted in short format.

\paragraph{MR-COR-060 -- Entry staging}
The entry staging state shall be represented as an object with the following
attributes,
\begin{enumerate}
\item project name,
\item task name,
\item start date,
\item stop date,
\item location name.
\end{enumerate}

These attributes are optional, meaning they may hold a value or not.

\paragraph{MR-COR-070 -- Entry staging to strings}
The entry staging object shall be convertible to a vector of string
representations of its attributes, in order, with the dates in
long format. Attributes which do not hold values are converted to a
single whitespace character.

\paragraph{MR-COR-080 -- Entry staging to short strings}
The entry staging object shall be convertible to the same vector
of strings as specified in MR-COR-070 except with the dates in short
format.

\paragraph{MR-COR-090 -- Export row}
Export rows (for CSV file export) shall be represented as an object
with the following attributes,
\begin{enumerate}
\item entry id,
\item project id,
\item project name,
\item task id,
\item task name,
\item location id,
\item location name,
\item start date,
\item stop date.
\end{enumerate}

\paragraph{MR-COR-100 -- Export csv string}
Export rows shall be convertible to a string representation with
comma-separated attributes. The dates are converted to unix timestamps.

\paragraph{MR-COR-110 -- Project total}
Project total (report of time worked per task) shall be represented
as objects with the following attributes,
\begin{enumerate}
\item project name,
\item total (Duration),
\item vector of task totals.
\end{enumerate}

Each task total has the following attributes,
\begin{enumerate}
\item task name,
\item total (Duration).
\end{enumerate}

\paragraph{MR-COR-120 -- Project total to menu items}
Project total objects shall be convertible to menu items for the TUI,
indicating,
\begin{enumerate}
\item The string to display in the cell,
\item The string to display in the status bar,
\item The string face to use (bold, italics etc.).
\end{enumerate}

The menu items are ordered in a nested fashion, with tasks grouped per
project, and the overall project total displayed first in bold.
The menu items are in lexicographic order.

An example of menu items as displayed is shown in
\cref{tab:project_total_menu_items}.

\begin{table} \caption{\label{tab:project_total_menu_items} Menu items for
    project total.}
  \begin{tabular}{| c | c |} \hline
    \textbf{Project1} & \textbf{13.202 hours} \\ \hline
    Task1 & 1.202 hours \\ \hline
    Task2 & 3.000 hours \\ \hline
    Task3 & 9.000 hours \\ \hline
  \end{tabular}
\end{table}

\paragraph{MR-COR-130 -- Weekly totals}
Weekly totals (report of time worked per task in a week) shall be represented
as objects with the following attributes,
\begin{enumerate}
\item total duration worked in the week,
\item total duration worked per day in the week,
\item vector of per project totals.
\end{enumerate}

Note a week has seven days and starts on monday.

Each per project total has the following attributes,
\begin{enumerate}
\item project name,
\item total duration worked on the project in the week,
\item total duration worked on the project per day,
\item vector of per task totals.
\end{enumerate}

Each per task total has the following attributes,
\begin{enumerate}
\item task name,
\item total duration worked on the task in the week,
\item total duration worked on the task per day.
\end{enumerate}

\paragraph{MR-COR-140 -- Weekly totals to menu items}
Weekly total objects shall be convertible to menu items for the TUI,
indicating,
\begin{enumerate}
\item The string to display in the cell,
\item The string to display in the status bar,
\item The string face to use (bold, italics etc.).
\end{enumerate}

The menu items are in lexicographic order. A column header
precedes the actual data. The menu items are, in order,
\begin{enumerate}
\item A header line indicating the columns
  \lstinline{Task, Mon, Tue, Wed, Thu, Fri, Sat, Sun, TOTAL} in normal face.
\item An all-tasks line, indicating the weekly total and daily breakdown of
  the total.
\item Per-project breakdown of weekly totals, with the project total first
  in bold face, following by all individual tasks in normal face.
\end{enumerate}

An example of menu items as displayed is shown in
\cref{tab:weekly_total_menu_items}.

\begin{table} \caption{\label{tab:weekly_total_menu_items} Menu items for
    weekly total.}
  \begin{tabular}{| c | c | c | c | c | c | c | c | c |} \hline
    Task & Mon & Tue & Wed & Thu & Fri & Sat & Sun & TOTAL \\ \hline
    ALL & 13.108 & 7.138 & 4.114 & 9.807
        & 12.498 & 12.386 & 12.039 & 71.090 \\ \hline
    \textbf{Project1} & \textbf{13.108} & \textbf{2.138} & \textbf{2.000}
                      & \textbf{4.807} & \textbf{6.000} & \textbf{7.386}
                      & \textbf{7.039} & \textbf{42.478} \\ \hline
    Task1 & 3.108 & 1.138 & 1.000
          & 2.807 & 3.000 & 2.386
          & 2.039 & 15.478 \\ \hline
    Task2 & 5.000 & 0.500 & 0.400
          & 0.500 & 2.000 & 2.500
          & 3.000 & 13.900 \\ \hline
    Task3 & 5.000 & 0.500 & 0.600
          & 1.500 & 1.000 & 2.500
          & 2.000 & 13.100 \\ \hline
    \textbf{Project2} & & \textbf{5.000} & \textbf{2.114}
                      & \textbf{5.000} & \textbf{6.498} & \textbf{5.000}
                      & \textbf{5.000} & \textbf{28.612} \\ \hline
    Task4 & & 5.000 & 2.114
          & 5.000 & 6.498 & 5.000
          & 5.000 & 28.612 \\ \hline
  \end{tabular}
\end{table}
\subsection{db}
The DB module contains a singleton which interacts with the \gls{DB}.
It is the unique point of direct interaction with the DB in the program,
either in read or write.

\paragraph{MR-DBI-010 -- DB loading}
The DB object shall be loaded using a filepath to the \gls{DB} file.
TODO: uplink

\paragraph{MR-DBI-020 -- DB get user version}
The DB object shall allow retrieving the user version number of the
loaded DB.

\subsubsection{DB tables}
\paragraph{MR-DBI-030 -- Projects table}
The \gls{DB} shall contain a \emph{projects} table with the following columns,
\begin{enumerate}
\item id,
\item name,
\item active (boolean flag).
\end{enumerate}

\paragraph{MR-DBI-040 -- Unique project names}
The \emph{projects} table shall only allow unique names.

\paragraph{MR-DBI-050 -- Tasks table}
The \gls{DB} shall contain a \emph{tasks} table with the following columns,
\begin{enumerate}
\item id,
\item name,
\item project\textunderscore id (id from the \emph{projects} table),
\item active (boolean flag).
\end{enumerate}

\paragraph{MR-DBI-060 -- Unique task names per project}
The \emph{tasks} table shall only allow unique names per project.
Two tasks are allowed to have the same name if they are in different
projects.

\paragraph{MR-DBI-070 -- Project's tasks deletion}
Upon deletion of a project, all its corresponding tasks must also
be deleted.
Note deletion is only possible if no task in the project is present
in entries.

\paragraph{MR-DBI-080 -- Locations table}
The \gls{DB} shall contain a \emph{locations} table with the following columns,
\begin{enumerate}
\item id,
\item name,
\item active (boolean flag).
\end{enumerate}

\paragraph{MR-DBI-090 -- Unique location names}
The \emph{locations} table shall only allow unique names.

\paragraph{MR-DBI-100 -- Entries table}
The \gls{DB} shall contain an \emph{entries} table with the following columns,
\begin{enumerate}
\item id,
\item task\textunderscore id (id from the \emph{tasks} table),
\item start (start date UTC UNIX timestamp),
\item stop (stop date UTC UNIX timestamp),
\item location\textunderscore id (id from the \emph{locations} table).
\end{enumerate}

\paragraph{MR-DBI-110 -- Entries locking hierarchy items removal}
The \emph{locations} and \emph{tasks} referenced to in \emph{entries} shall be
locked for removal, along with their corresponding project. \emph{Locations},
\emph{tasks} and \emph{projects} may only be removed once no part of
\emph{entries} references them.

\paragraph{MR-DBI-120 -- Entries start stop ordering}
The \emph{entries} table shall only allow \emph{start} and \emph{stop}
values for which $start < stop$.

\paragraph{MR-DBI-130 -- Entries non-overlapping dates}
The \emph{entries} table shall only allow items with non-overlapping
\emph{start} and \emph{stop} dates.

\paragraph{MR-DBI-140 -- Entry staging table}
The \gls{DB} shall contain an \emph{entrystaging} table with the following
columns,
\begin{enumerate}
\item task\textunderscore id (id from the \emph{tasks} table),
\item start,
\item stop,
\item location\textunderscore id (id from the \emph{locations} table).
\end{enumerate}

Note the \emph{entrystaging} table only ever has one row.

\subsubsection{Hierarchy items}
\paragraph{MR-DBI-150 -- Inserting projects}
The \gls{DB} shall allow inserting new \emph{projects}, identified by a name
string. The operation must return true for success and false in case
a constraint was violated.

\paragraph{MR-DBI-160 -- Projects active default}
The newly inserted projects shall be active by default.

\paragraph{MR-DBI-170 -- Inserting tasks}
The \gls{DB} shall allow inserting new \emph{tasks}, identified by the parent
project id, and a name string. The operation must return true for success
and false in case a constraint was violated.

\paragraph{MR-DBI-180 -- Tasks active default}
The newly inserted tasks shall be active by default.

\paragraph{MR-DBI-190 -- Inserting locations}
The \gls{DB} shall allow inserting new \emph{locations}, identified by a name
string. The operation must return true for success and false in case a
constraint was violated.

\paragraph{MR-DBI-200 -- Locations active default}
The newly inserted locations shall be active by default.

\paragraph{MR-DBI-210 -- Project name edit}
The \gls{DB} shall allow editing the name of existing \emph{projects}.
The projects are identified by their id. The operation must return true for
success and false in case a constraint was violated.

\paragraph{MR-DBI-220 -- Task name edit}
The \gls{DB} shall allow editing the name of existing \emph{tasks}.
The tasks are identified by their id. The operation must return true for success
and false in case a constraint was violated.

\paragraph{MR-DBI-230 -- Task project edit}
The \gls{DB} shall allow editing the project id of existing \emph{tasks}.
The tasks are identified by their id, and projects by their name.
The operation must return true for success and false in case a
constraint was violated.

\paragraph{MR-DBI-240 -- Location name edit}
The \gls{DB} shall allow editing the name of existing \emph{locations}.
The locations are identified by their id.
The operation must return true for success and false in case a
constraint was violated.

\paragraph{MR-DBI-250 -- Toggle project active}
The \gls{DB} shall allow toggling the active status of existing \emph{projects}.
The projects are identified by their id.

\paragraph{MR-DBI-260 -- Toggle task active}
The \gls{DB} shall allow toggling the active status of existing \emph{tasks}.
The tasks are identified by their id.

\paragraph{MR-DBI-270 -- Toggle location active}
The \gls{DB} shall allow toggling the active status of existing \emph{locations}.
The locations are identified by their id.

\paragraph{MR-DBI-280 -- Query projects}
The \gls{DB} shall allow querying the list of existing \emph{projects}.
The result is in alphabetical order.

\paragraph{MR-DBI-290 -- Query active projects}
The \gls{DB} shall allow querying the list of active \emph{projects}.
The result is in alphabetical order.

\paragraph{MR-DBI-300 -- Query tasks of project}
The \gls{DB} shall allow querying the list of existing \emph{tasks}
for a given project id.
The result is in alphabetical order.

\paragraph{MR-DBI-310 -- Query active tasks of project}
The \gls{DB} shall allow querying the list of active \emph{tasks}
for a given project id.
The result is in alphabetical order.

\paragraph{MR-DBI-320 -- Query locations}
The \gls{DB} shall allow querying the list of existing \emph{locations}.
The result is in alphabetical order.

\paragraph{MR-DBI-330 -- Query active locations}
The \gls{DB} shall allow querying the list of active \emph{locations}.
The result is in alphabetical order.

\paragraph{MR-DBI-340 -- Delete projects}
The \gls{DB} shall allow the deletion of existing \emph{projects},
identified by their id.
The operation must return true for success and false in case a
constraint was violated.

\paragraph{MR-DBI-350 -- Delete tasks}
The \gls{DB} shall allow the deletion of existing \emph{tasks},
identified by their id.
The operation must return true for success and false in case a
constraint was violated.

\paragraph{MR-DBI-360 -- Delete locations}
The \gls{DB} shall allow the deletion of existing \emph{locations},
identified by their id.
The operation must return true for success and false in case a
constraint was violated.

\subsubsection{Entry staging}
\paragraph{MR-DBI-370 -- Entry staging project name}
The \gls{DB} shall allow editing the \emph{entrystaging} project name.
Only active projects may be set.
The operation must return true for success and false in case a
constraint was violated.

\paragraph{MR-DBI-380 -- Entry staging task name}
The \gls{DB} shall allow editing the \emph{entrystaging} task name.
Only active tasks may be set.
The operation must return true for success and false in case a
constraint was violated.

\paragraph{MR-DBI-390 -- Entry staging start}
The \gls{DB} shall allow editing the \emph{entrystaging} start \emph{date}.
The operation must return true for success and false in case a
constraint was violated.

\paragraph{MR-DBI-400 -- Entry staging stop}
The \gls{DB} shall allow editing the \emph{entrystaping} stop \emph{date}.
The operation must return true for success and false in case a
constraint was violated.

\paragraph{MR-DBI-410 -- Entry staging location}
The \gls{DB} shall allow editing the \emph{entrystaging} location name.
Only active locations may be set.
The operation must return true for success and false in case a
constraint was violated.

\paragraph{MR-DBI-420 -- Entry staging commit}
The \gls{DB} shall allow committing the \emph{entrystaging} to an
\emph{entries} item.
The operation must return true for success and false in case a
constraint was violated.

\paragraph{MR-DBI-430 -- Entry staging query}
The \gls{DB} shall allow the retrieval of the current \emph{entrystaging} item.

\paragraph{MR-DBI-440 -- Entry staging project id}
The \gls{DB} shall allow the retrieval of the current \emph{entrystaging}
project id. The return value is optional, as there is no value to
return in the case where no project/task is set in the entry staging.

\subsubsection{Entries}
\paragraph{MR-DBI-450 -- Entry project edit}
The \gls{DB} shall allow the editing of existing \emph{entries} project
by name.
The operation must return true for success and false in case a
constraint was violated.

\paragraph{MR-DBI-460 -- Entry task edit}
The \gls{DB} shall allow the editing of existing \emph{entries} task
by name.
The operation must return true for success and false in case a
constraint was violated.

\paragraph{MR-DBI-470 -- Entry start edit}
The \gls{DB} shall allow the editing of existing \emph{entries} start
\emph{date}.
The operation must return true for success and false in case a
constraint was violated.

\paragraph{MR-DBI-480 -- Entry stop edit}
The \gls{DB} shall allow the editing of existing \emph{entries} stop
\emph{date}.
The operation must return true for success and false in case a
constraint was violated.

\paragraph{MR-DBI-490 -- Entry location edit}
The \gls{DB} shall allow the editing of existing \emph{entries} location
by name.
The operation must return true for success and false in case a
constraint was violated.

\paragraph{MR-DBI-500 -- Entries query}
The \gls{DB} shall allow the retrieval of the list of existing \emph{entries}
in a given \emph{date range}. The entries shall be returned in
increasing order of start date.

\paragraph{MR-DBI-510 -- Entry project id query}
The \gls{DB} shall allow the retrieval of the project id associated to
a given item in \emph{entries}, identified by id.

\paragraph{MR-DBI-520 -- Delete entry}
The \gls{DB} shall allow the deletion of existing \emph{entries}, by id.
The operation must return true for success and false in case a
constraint was violated.

\paragraph{MR-DBI-530 -- Entries duration over date range}
The \gls{DB} shall allow the retrieval of the total \emph{duration} worked
on \emph{entries} in a given \emph{date range}.

\subsubsection{Reports}
\paragraph{MR-DBI-540 -- Entries export}
The \gls{DB} shall allow the export of entries to \emph{export row} format,
over a given \emph{date range}. The exported rows must be ordered by
increasing entry start date.

\paragraph{MR-DBI-550 -- Project total}
The \gls{DB} shall allow the generation of a \emph{project total} report
over a given \emph{date range}. The projects must be ordered alphabetically,
and the same is required for the set of tasks inside each project.

\paragraph{MR-DBI-560 -- Weekly totals}
The \gls{DB} shall allow the generation of a \emph{weekly totals} report
over a given \emph{week}.

\subsection{keys}
\paragraph{MR-KEY-010}
TODO: uplink

\subsection{tui}
\paragraph{MR-TUI-010}
TODO: uplink

\subsection{exporter}
\paragraph{MR-EXP-010 -- Export file}
The exporter module shall produce an export file compliant with [AD7],
given a set of entry \emph{export rows} in a given \emph{date range},
and the filepath to the target export file.

\paragraph{MR-EXP-020 -- Export file directory exception}
If a directory is provided to the exporter module instead of a filepath,
then an exception shall be thrown.

\paragraph{MR-EXP-030 -- Export file exists exception}
If the filepath provided to the exporter module corresponds to
a file which already exists, then an exception shall be thrown.
