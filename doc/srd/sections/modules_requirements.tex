\section{Modules requirements}
\subsection{version}
The \emph{version} module configures the versions of program components,
to ensure compatibility between different executables and \gls{DB}.

\paragraph{MR-VER-010 -- Program version number}
The \emph{version} module shall provide a program version number
as a string in the following format: \lstinline{x.y},
with \lstinline{x} the major revision number and \lstinline{y} the
minor revision number. The major version number and minor version
number are both integers. Zero is an acceptable version number.

The program version number may be suffixed with \lstinline{dev} to
indicate an internal development version.

Valid program version number examples are:
\begin{itemize}
\item \lstinline{0.5}: for major version 0, minor version 5.
\item \lstinline{3.1}: for major version 3, minor version 1.
\item \lstinline{3.1dev}: for a development version after \lstinline{3.1}.
\end{itemize}

\textit{uplink: } R-REL-010 -- Software version format

\paragraph{MR-VER-020 -- DB version number}
The \emph{version} module shall provide a \gls{DB} version number as
a positive integer.

\textit{uplink: } TODO DB module.

\subsection{config}
The \emph{config} module loads the configuration file, translates the parameters
into an internal representation and transmits them to the rest of the program.

\subsubsection{Configuration file loading}

\paragraph{MR-CON-010 -- Load from path}
The \emph{config} module shall load any compliant user configuration file when
provided with its path.

\textit{uplink: } TODO.

\paragraph{MR-CON-020 -- Non-existent path}
The \emph{config} module shall emit an exception when the path provided for the
configuration file corresponds to a file which does not exist.

\textit{uplink: } TODO.

\paragraph{MR-CON-030 -- Default configuration file locations}
The \emph{config} module shall provide a default load method which looks
for the configuration file \lstinline{timesheeting/timesheeting.toml}
in the following default locations, in order of preference,
\begin{enumerate}
  \item \lstinline{$XDG_CONFIG_HOME}
  \item \lstinline{$HOME/.config/}
  \item \lstinline{/etc/}
\end{enumerate}

\textit{uplink: } TODO.

\paragraph{MR-CON-040 -- Default configuration file not found}
The \emph{config} module default load method shall emit an exception if
no configuration file was found in any of the default locations.

\textit{uplink: } TODO.

\subsubsection{Logger configuration}
\paragraph{MR-CON-050 -- Log filepath}
The \emph{config} module shall load from the configuration file a filepath
for the log file. This parameter is at the node \lstinline{log.file}.

\textit{uplink: } TODO.

\paragraph{MR-CON-060 -- Log levels}
The \emph{config} module shall load from the configuration file a vector of
strings defining the active log levels. This parameter is at the node
\lstinline{log.active_levels}.

\textit{uplink: } TODO.

\subsubsection{DB configuration}
\paragraph{MR-CON-070 -- DB filepath}
The \emph{config} module shall load from the configuration file a filepath
for the DB file. This parameter is at the node \lstinline{db.file}.

\textit{uplink: } TODO.

\subsubsection{Timezone configuration}
\paragraph{MR-CON-080 -- Timezone string}
The \emph{config} module shall load from the configuration file a string
representing the timezone to use. This parameter is at the node
\lstinline{time.timezone}.

\textit{uplink: } TODO.

\subsubsection{Duration display configuration}
\paragraph{MR-CON-090 -- Duration display days}
The \emph{config} module shall load from the configuration file a float
representing the number of hours in a workday for duration display.
This parameter is at the node \lstinline{time.hours_per_workday}.

\textit{uplink: } TODO.

\subsubsection{Key bindings configuration}
The key bindings are loaded into maps from key to action. The maps are,
\begin{itemize}
\item The \emph{normal map}, which is the binding map for the normal mode,
\item The \emph{edit map}, which is the binding map for the edit mode.
\end{itemize}

The normal mode is the regular mode of operation of the program, entered
upon program launch. The edit mode is used when typing text.
For the purpose of organization of the user-facing configuration file,
the normal mode bindings are separated into the sections \lstinline{navigation}
and \lstinline{actions}, while the edit mode bindings are under the section
\lstinline{edit\_mode}.

\paragraph{MR-CON-100 -- Binding list}
The \emph{config} module shall load bindings from the following nodes,
organized hierarchically. The top level node is \lstinline{keys}.

The \emph{normal map} bindings are loaded from,
\begin{itemize}
\item navigation
  \begin{itemize}
  \item up
  \item down
  \item left
  \item right
  \item subtabs
  \item previous
  \item next
  \item duration\_display
  \item entries\_screen
  \item projects\_screen
  \item locations\_screen
  \item project\_report\_screen
  \item weekly\_report\_screen
  \item active\_visibility
  \item quit
  \end{itemize}
\item actions
  \begin{itemize}
  \item commit\_entry
  \item set\_now
  \item add
  \item rename
  \item remove
  \item active\_toggle
  \item task\_project\_change
  \end{itemize}
\end{itemize}

The \emph{edit map} bindings are loaded from,
\begin{itemize}
\item edit\_mode
  \begin{itemize}
  \item validate
  \item cancel
  \item select\_suggestion
  \end{itemize}
\end{itemize}

\textit{uplink: } TODO.

\paragraph{MR-CON-110 -- Regular keys}
The \emph{config} module shall allow binding any regular (\textit{ie}
non-escaped and non-special) character key to any binding action.
The regular keys are designated by their character in the configuration
file, \textit{ie} the key \lstinline{e} is designated by the character
\lstinline{"e"} in the configuration file.

\textit{uplink: } TODO.

\paragraph{MR-CON-120 -- Special keys}
The \emph{config} module shall allow binding the following special
keys to any binding action. The special keys are designated by strings.
\begin{itemize}
\item \lstinline{ESCAPE}: the escape key.
\item \lstinline{ENTER}: the enter or \emph{return} key.
\item \lstinline{SPACE}: the spacebar.
\item \lstinline{TAB}: the tabulation key.
\item \lstinline{UP}: the up arrow key.
\item \lstinline{DOWN}: the down arrow key.
\item \lstinline{LEFT}: the left arrow key.
\item \lstinline{RIGHT}: the right arrow key.
\end{itemize}

\textit{uplink: } TODO.

\paragraph{MR-CON-130 -- Multiple keys for an action}
The \emph{config} module shall allow binding multiple keys to the
same action. The keys are specified as a list of strings attached
to the corresponding action node.

For instance, for binding the keys \lstinline{e} and \lstinline{a}
to the action with node \lstinline{rename}, the configuration file
line is: \lstinline{rename = ["e", "a"]}.

\textit{uplink: } TODO.

\paragraph{MR-CON-140 -- Protection against duplicates}
The \emph{config} module shall emit an exception when the configuration
file contains duplicate keys bound in the same map (\textit{ie} either
\emph{normal} or \emph{edit}).

\textit{uplink: } TODO.

\paragraph{MR-CON-150 -- Backspace mapping}
The \emph{config} module shall map the backspace key to a backspace
action in the \emph{edit} map. This is done outside of the configuration
file, it is an implicit binding.

\textit{uplink: } TODO.

\paragraph{MR-CON-160 -- Unbound keys}
The maps in the \emph{config} module shall return an \emph{unbound}
action when queried with a key which was not bound during configuration.

\textit{uplink: } TODO.

\paragraph{MR-CON-170 -- Invalid key strings}
The \emph{config} module shall emit an exception when a key
string in the configuration file does not map to any regular or \emph{special}
characters.

\subsection{cli}
\paragraph{MR-CLI-010}
TODO: uplink

\subsection{core}
\paragraph{MR-COR-010}
TODO: uplink

\subsection{db}
\paragraph{MR-DBI-010}
TODO: uplink

\subsection{keys}
\paragraph{MR-KEY-010}
TODO: uplink

\subsection{tui}
\paragraph{MR-TUI-010}
TODO: uplink

\subsection{exporter}
\paragraph{MR-EXP-010}
TODO: uplink
