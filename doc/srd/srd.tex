%% Document-wide settings.
\documentclass[letterpaper]{article}
\title{timesheeting system requirements document}
\author{Thomas HOULLIER \href{mailto:pro@houllier.net}
         {\texttt{\textlangle pro@houllier.net\textrangle}}}

\usepackage[colorlinks=true, allcolors=blue,
            hyperfootnotes=false,
            pdfauthor={Thomas HOULLIER},
            pdftitle={timesheeting system requirements document}]
            {hyperref} % Links for ref/cite.

%% Loading packages
\usepackage{amsmath} % For cases in equations.
\usepackage{amsfonts} % For maths sets.
\usepackage{physics} % For \abs{} and \norm{}.
\usepackage[inkscapelatex=false]{svg} %svg graphics
\usepackage{siunitx} % units formatting

\usepackage[backend=biber,style=numeric,citestyle=numeric-comp,maxcitenames=99,dateabbrev=false]{biblatex}
\addbibresource{biblio.bib}
\usepackage{setspace} % Bibliography spacings
\DeclareSourcemap{
  \maps[datatype=bibtex]{
    \map[overwrite]{
      \step[fieldsource=doi, final]
      \step[fieldset=url, null]
      \step[fieldset=eprint, null] }}}
\setcounter{biburllcpenalty}{7000} % break long url in bibliography
\setcounter{biburlucpenalty}{8000}
\renewcommand*{\bibfont}{\footnotesize} % bibliography font size
% Format of biblatex urldate in the bibliography.
\DeclareFieldFormat{urldate}{%
  Visited on \thefield{urlday}\addspace%
  \mkbibmonth{\thefield{urlmonth}}\addspace%
  \thefield{urlyear}\isdot}
\usepackage[ruled,vlined]{algorithm2e} % Algorithms.
\DontPrintSemicolon
\SetKwInOut{Input}{Input}\SetKwInOut{Output}{Output}
\usepackage{mathtools} % Ceiling function.
\usepackage{outlines} % Nest lists.
\usepackage{interval} % Writing intervals.
\usepackage[font={footnotesize,sf}]{caption} %Caption for figures in minipages.
\usepackage{floatrow}
% Figure captions always below. Figures always centered.
\floatsetup[figure]{capposition=bottom,objectset=centering}
\usepackage{wrapfig} %Wrapping figure with text.
\usepackage{stmaryrd} % Double brackets for integers interval.
\usepackage{doi} % Hyperlink DOI
\usepackage{etoolbox} %Ragged right bibliography.
\usepackage{color, colortbl} % Coloring rows in tables.
\usepackage{subcaption} % Subfigures.
\usepackage{pdfpages} % Include PDF pages.
\usepackage{epigraph} % Quotations at beginning of chapters.
\setlength\epigraphwidth{.8\textwidth}
\usepackage[acronym,nonumberlist,nogroupskip,nopostdot]{glossaries} % Glossary for acronyms.
\renewcommand*{\glstextformat}[1]{\textcolor{black}{#1}} % No color on links for abbrev.

\DeclarePairedDelimiter{\ceil}{\lceil}{\rceil} % Ceiling function.
\DeclarePairedDelimiter{\floor}{\lfloor}{\rfloor} % Floor function.

\DeclareMathOperator*{\argmin}{argmin}

\setcounter{tocdepth}{3} % Table of content depth
\setcounter{secnumdepth}{3} % Section numbering depth

% Non-breaking around footnotes.
\makeatletter
\let\Footnote\footnote
\def\pst@@killglue{\unskip\ifdim\lastskip>\z@\expandafter\pst@@killglue\fi}
\def\footnote{\pst@@killglue\Footnote}
\makeatother

% More space below equations
\appto\normalsize{\belowdisplayshortskip=\belowdisplayskip}

% Rewrite month codes in bibliography
\DeclareSourcemap{
  \maps[datatype=bibtex]{
    \map[overwrite]{
      \step[fieldsource=month, match=\regexp{\A(j|J)an(uary)?\Z}, replace=1]
      \step[fieldsource=month, match=\regexp{\A(f|F)eb(ruary)?\Z}, replace=2]
      \step[fieldsource=month, match=\regexp{\A(m|M)ar(ch)?\Z}, replace=3]
      \step[fieldsource=month, match=\regexp{\A(a|A)pr(il)?\Z}, replace=4]
      \step[fieldsource=month, match=\regexp{\A(m|M)ay\Z}, replace=5]
      \step[fieldsource=month, match=\regexp{\A(j|J)un(e)?\Z}, replace=6]
      \step[fieldsource=month, match=\regexp{\A(j|J)ul(y)?\Z}, replace=7]
      \step[fieldsource=month, match=\regexp{\A(a|A)ug(ust)?\Z}, replace=8]
      \step[fieldsource=month, match=\regexp{\A(s|S)ep(tember)?\Z}, replace=9]
      \step[fieldsource=month, match=\regexp{\A(o|O)ct(ober)?\Z}, replace=10]
      \step[fieldsource=month, match=\regexp{\A(n|N)ov(ember)?\Z}, replace=11]
      \step[fieldsource=month, match=\regexp{\A(d|D)ec(ember)?\Z}, replace=12]}}}

% Footnotes marker color
\renewcommand\thefootnote{\textcolor{blue}{\arabic{footnote}}}

\pdfsuppresswarningpagegroup=1 % Silence warnings about pagegroups for figures.
\pdfminorversion=6 % PDF version 1.6 since we include articles in 1.6.

% Allow an extra pass to fix overfull hboxes by allowing more whitespace.
\emergencystretch=1em

% Page numbering and copyright notice.
\usepackage{fancyhdr}
\usepackage{lastpage}

\fancypagestyle{FirstPage}{
\fancyhf{} % Clear footer.
\rfoot{\thepage \hspace{1pt} of \pageref*{LastPage}}
\renewcommand{\headrulewidth}{0pt} % Remove rule at top of page
\lfoot{\href{https://creativecommons.org/licenses/by/4.0/}
       {\includesvg[inkscapelatex=false,height=14pt]{images/ccby.svg}}}}

\fancypagestyle{plain}{
\fancyhf{} % Clear footer.
\rfoot{\thepage \hspace{1pt} of \pageref*{LastPage}}
\renewcommand{\headrulewidth}{0pt} % Remove rule at top of page
}

% Version history
\usepackage{vhistory}

% Keywords
\providecommand{\keywords}[1]{\textbf{Keywords --} #1}

% Glossary
\makeglossaries
\loadglsentries{../glossary/glossary.tex}

\usepackage{fontawesome} %inline icons
\usepackage{xcolor}
\usepackage{listings} % Code listings
\definecolor{codeback}{rgb}{0.99,0.99,0.98}
\definecolor{codecomment}{HTML}{0588fc}
\definecolor{codekeyword}{HTML}{af5f00}
\definecolor{codestring}{HTML}{ffa07a}
\lstdefinestyle{mystyle}{
  backgroundcolor=\color{codeback},
  commentstyle=\color{codecomment},
  keywordstyle=\color{codekeyword},
  stringstyle=\color{codestring},
  basicstyle=\ttfamily\footnotesize,
  breakatwhitespace=false,         
  breaklines=true,                 
  captionpos=b,                    
  keepspaces=true,                 
  numbers=left,                    
  numbersep=5pt,                  
  showspaces=false,                
  showstringspaces=false,
  showtabs=false,                  
  tabsize=2}
\lstset{style=mystyle}

\usepackage[capitalise,nameinlink]{cleveref} % Include eg. "Fig." in front of figures.
\crefname{algorithm}{Alg.}{Algs.}
\crefname{table}{Tab.}{Tabs.}
\crefname{equation}{Eq}{Eqs.}
% Equation cross-references.
%\creflabelformat{equation}{#2#1#3}
\crefformat{equation}{(#2Eq.\thinspace#1#3)}

% No parentheses in equation labels.
%\newtagform{noparen}{}{}
%\usetagform{noparen}

% Document
\begin{document}
\frenchspacing
\date{PRJ1-SRD1-v0.1 -- XXX}
\maketitle
\thispagestyle{FirstPage}

\begin{abstract}
  This document lists the requirements for every subsystem in the
  \emph{timesheeting} program.
  It answers to the top-level specification [AD1] of the program and implements
  the architecture discussed in [AD2].
\end{abstract}

\begin{versionhistory}
  \vhEntry{XXX}{XXX}{TH}{Creation}
\end{versionhistory}
\setcounter{table}{0} % Reset the table counter.

\section*{Applicable documents}
{ \centering
\begin{tabularx}{\textwidth}{| X | X | X | X | X |} \hline
  Index & Title & Reference & Revision & Author \\ \hline
  AD1   & timesheeting specification document & PRJ1-SPE1 & v1.1 & Thomas
  HOULLIER \\ \hline
  AD2   & timesheeting system architecture document & PRJ1-SAD1 & v1.0 & Thomas
  HOULLIER \\ \hline
  AD3   & TOML & \url{https://toml.io/en/v1.0.0} & v1.0.0 & Tom PRESTON-WERNER,
                                                            Pradyun GEDAM, et
                                                            al. \\ \hline
  AD4   & List of tz database time zones
        & \url{https://en.wikipedia.org/w/index.php?title=List_of_tz_database_time_zones}
        & 1269854021 & Wikipedia contributors \\
  \hline \end{tabularx} \par }

\section*{Document distribution}
The present document is distributed under the \emph{Creative Commons Attribution
4.0 International} license (\url{https://creativecommons.org/licenses/by/4.0/})
by its author Thomas HOULLIER.

Every document release is signed with the author's GPG key. A signature file
is provided along with the released document.

\tableofcontents
\printglossary[type=\acronymtype,style=index]
\pagestyle{plain}
\section{Introduction}
\subsection{Context}
In support of the project requirement R-DEX-010 [RD1], we provide a format
for timesheet data export to a file.

The format is meant to be interoperable, \textit{ie} it is meant to be
easily usable with a wide selection of external programs. The external
programs typically targetted are spreadsheet programs (\emph{Libreoffice Calc},
\emph{Gnumeric}), \gls{CLI} text programs (\emph{less}, \emph{vi}),
and Python libraries such as \emph{pandas}.

We prioritize ease of use and readability for the exported format over
compactness and efficiency.

\subsection{Document structure}
The document is structured as follows. First, definitions are given
(\cref{sec:definitions}), then the requirements for the exported file
are listed (\cref{sec:requirements}) and finally an example of a compliant
export file is given (\cref{sec:example}).

\section{Implemented architecture}
The program is built from \emph{modules}. These modules are themselves supported
by \emph{libraries}. The modules implement behavior specific to the program
while the libraries abstract generic behaviors. The libraries may in theory be
reused in other programs. \cref{fig:arch_modules} is a diagram showing the
dependency between the different modules and libraries in the program. This
diagram also shows external dependencies, only for information. A brief
description of each module and library is given below.

\begin{figure}
  \includesvg[width=\textwidth]{images/modules_include/modules.svg}
  \caption{\label{fig:arch_modules} Implemented system architecture diagram.}
\end{figure}

\subsection{Libraries}
The following libraries are included in the program,
\begin{itemize}
\item \textbf{config\textunderscore lib} loads a configuration file and reads fields
  from it.
\item \textbf{time\textunderscore lib} contains objects related to date and
  time, including durations, timezone management.
\item \textbf{log\textunderscore lib} provides logging capability of messages to
  a file.
\item \textbf{db\textunderscore lib} abstracts the interface to the \gls{DB}, it
  allows running statements for setting or querying data.
\item \textbf{ncurses\textunderscore lib} implements some objects on top
  of ncurses.
\item \textbf{suggestion} is a text suggestion engine.
\end{itemize}

\subsection{Modules}
The following modules are included in the program,
\begin{itemize}
\item \textbf{version} configures the version numbers for the program and the
  \gls{DB}.
\item \textbf{config} loads a user configuration file and transmits the
  parameters to the rest of the program.
\item \textbf{cli} creates the \gls{CLI} for the program.
\item \textbf{core} contains data objects specific to the program, to be
  manipulated across modules.
\item \textbf{db} is the \gls{DB} interface implementation.
\item \textbf{keys} manages the key bindings for the \gls{TUI}.
\item \textbf{tui} creates the \gls{TUI} for the program.
\item \textbf{exporter} allows the export of timesheet data to an external
  data format.
\end{itemize}

\section{Libraries requirements}
\subsection{config\textunderscore lib}
\subsubsection{Utilities}
\paragraph{LR-CON-010 -- Tilde expansion utility}
The config\textunderscore lib shall provide a utility function for tilde
expansion of filepaths. It replaces eventual leading \lstinline{~/}
with \lstinline{$HOME} in filepaths. It does nothing on filepaths
without tilde.

\textit{Uplink: } TODO.

\paragraph{LR-CON-020 -- First existing file}
The config\textunderscore lib shall provide a utility function, which,
given a filepath suffix and a list of folders, returns the first found
existing filepath by trying the suffix over the successive folder
candidates. It returns nothing if no possible file exists.

\textit{Uplink: } TODO.

\subsubsection{Configuration loader}
\paragraph{LR-CON-030 -- Configuration file format}
The config\textunderscore lib file loader shall use the TOML file format [AD3]
for the configuration file.

\textit{Uplink: } TODO.

\paragraph{LR-CON-040 -- String loading}
The configuration file loader shall allow reading parameters of type
\emph{string}.

\textit{Uplink: } TODO.

\paragraph{LR-CON-050 -- String empty case}
The configuration file loader shall emit an exception if a loaded
\emph{string} is empty (\textit{ie} \lstinline{""}).

\textit{Uplink: } TODO.

\paragraph{LR-CON-060 -- Filepath loading}
The configuration file loader shall allow reading parameters of type
\emph{filepath} with automatic tilde expansion.

\textit{Uplink: } TODO.

\paragraph{LR-CON-070 -- Filepath parent non-existent}
The configuration file loader shall emit an exception if the loaded
\emph{filepath} direct parent does not exist.

\textit{Uplink: } TODO.

\paragraph{LR-CON-080 -- Float loading}
The configuration file loader shall allow reading parameters of type
\emph{float}.

\textit{Uplink: } TODO.

\paragraph{LR-CON-090 -- Parameter empty case}
The configuration file loader shall emit an exception if any parameter
is empty, \textit{ie} \lstinline{parameter = }.

\textit{Uplink: } TODO.

\paragraph{LR-CON-100 -- Vector of strings loading}
The configuration file loader shall allow reading parameters of type
\emph{vector of strings}. The order of strings in the vector is preserved.

\textit{Uplink: } TODO.

\paragraph{LR-CON-110 -- Vector of non-strings case}
The configuration file loader shall emit an exception if a loaded
\emph{vector of strings} does not in fact contain strings.

\textit{Uplink: } TODO.

\paragraph{LR-CON-120 -- Configuration file nonexistent}
The configuration file loader shall emit an exception if it is provided
a configuration filepath which does not exist.

\textit{Uplink: } TODO.

\paragraph{LR-CON-130 -- Unsigned integer loading}
The configuration file loader shall allow reading parameters of type
\emph{unsigned integer}.

\textit{Uplink: } TODO.

\subsection{time\textunderscore lib}
\subsubsection{time\textunderscore zone}
The time\textunderscore zone object is a singleton which provides the
timezone information to the rest of the program.

\paragraph{LR-TMZ-010 -- Time zone initialization}
The timezone object shall be initialized with a valid TZ identifier string,
as defined in [AD4].

TODO: uplink
\paragraph{LR-TMZ-020 -- Time zone singleton}
The timezone object shall be a singleton. It is initialized once and the same
instance is retrieved through a \lstinline{get} method subsequently.

TODO: uplink
\paragraph{LR-TMZ-030 -- Invalid time zone}
The timezone object shall emit an exception if it is initialized with
an invalid TZ string (\textit{ie} a string outside of those defined in [AD4]).

TODO: uplink
\paragraph{LR-TMZ-040 -- Time zone name}
The timezone class shall provide a method to retrieve its TZ identifier string.

TODO: uplink

\paragraph{LR-TMZ-050 -- std time\textunderscore zone}
The timezone class shall provide a method to retrieve the
\lstinline{std::chrono::time_zone} representation corresponding to the set
time zone.

TODO: uplink

\subsubsection{date}
A Date is a representation of a time point in UTC using the system clock
as time reference.

\paragraph{LR-DAT-010 -- Current time initialization}
The Date shall allow initialization at the current time.

TODO: uplink

\paragraph{LR-DAT-020 -- std::time\textunderscore point initialization}
The Date shall allow initialization from a \lstinline{std::time_point}.

TODO: uplink

\paragraph{LR-DAT-030 -- Beginning of year}
The Date shall allow initialization at the beginning of the current year
in the current time zone (as set in the TimeZone singleton).

TODO: uplink

\paragraph{LR-DAT-040 -- UNIX timestamp initialization}
The Date shall allow initialization from a UNIX timestamp in seconds.

TODO: uplink

\paragraph{LR-DAT-050 -- Date string initialization}
The Date shall allow initialization from a string in the format
\lstinline{%d%b%Y %H:%M:%S}, with format specifiers as defined in [AD5].
The date initialization uses time zone currently set
in the TimeZone singleton.

An example of a valid initialization string is \lstinline {19Jan2025 09:41:34}.

TODO: uplink

\paragraph{LR-DAT-060 -- Date string shortcuts}
The Date shall allow initialization from shortened strings
with formats,
\begin{itemize}
\item \lstinline{%d%b%Y %H:%M}
\item \lstinline{%d%b%Y %H}
\item \lstinline{%d%b%Y}
\end{itemize}
(with the same set of format specifiers as in LR-DAT-050),
replacing the omitted items from the full format defined in LR-DAT-050
by zero values.

Examples of shortened initialization strings and their full equivalent are
given in \cref{tab:date_shortened}.

\begin{table}
  \caption{\label{tab:date_shortened}
    Shortened date initialization string examples}
  \begin{tabular}{| c | c |} \hline
    \textbf{Shortened} & \textbf{Full} \\ \hline
    \texttt{19Jan2025 09:41} & \texttt{19Jan2025 09:41:00} \\ \hline
    \texttt{19Jan2025 09} & \texttt{19Jan2025 09:00:00} \\ \hline
    \texttt{19Jan2025} & \texttt{19Jan2025 00:00:00} \\ \hline
  \end{tabular}\end{table}

TODO: uplink

\paragraph{LR-DAT-070 -- Date string invalid}
During Date initialization with a date string, an exception shall be
thrown if an invalid date string is used (\textit{ie} outside of the formats
specified).

TODO: uplink

\paragraph{LR-DAT-075 -- Date std::time\textunderscore point access}
The Date shall allow read access to its internal
\lstinline{std::chrono::time_point} representation.

TODO: uplink

\paragraph{LR-DAT-080 -- Date output string}
The Date shall be convertible to a string in format
\lstinline{%d%b%Y %H:%M:%S} (see [AD5]) in the current time zone.

TODO: uplink
\paragraph{LR-DAT-090 -- Date output hours/minutes}
The Date shall be convertible to a string in format
\lstinline{%H:%M} (see [AD5]) in the current time zone.

TODO: uplink
\paragraph{LR-DAT-100 -- Date output unambiguous string}
The Date shall be convertible to a string in format
\lstinline{%d%b%Y %H:%M:%S %z} (see [AD5]) in the current time zone.

TODO: uplink
\paragraph{LR-DAT-110 -- Date output UNIX timestamp}
The Date shall be convertible to a UNIX timestamp in seconds in \gls{UTC}.

TODO: uplink
\paragraph{LR-DAT-120 -- Date output day/month/year}
The Date shall be convertible to a string in format
\lstinline{%d%b%Y} (see [AD5]) in the current time zone.

TODO: uplink
\paragraph{LR-DAT-130 -- Second resolution}
The Date shall represent time points with a resolution of at most 1 second.

TODO: uplink

\paragraph{LR-DAT-140 -- Date comparison}
The Date class shall provide \emph{lesser than} and \emph{greater than}
comparison operators.

TODO: uplink

\paragraph{LR-DAT-150 -- Date ago initialization}
The Date shall allow initialization at a number of seconds ago from
\emph{now}.

TODO: uplink

\paragraph{LR-DAT-160 -- Date unambiguous string initialization}
The Date shall allow initialization from a string in the
same format as the one outputted in LR-DAT-100.

TODO: uplink

\subsubsection{DateRange}
A DateRange represents a range between a start Date and a stop Date.

\paragraph{LR-DTR-010 -- DateRange initialization}
The DateRange shall be initialized using a start Date and a stop Date.

TODO: uplink

\paragraph{LR-DTR-020 -- DateRange ordering}
The DateRange initialization shall emit an exception if the
start Date is \emph{greater than} the stop Date.

TODO: uplink

\paragraph{LR-DTR-030 -- Dates read access}
The DateRange shall allow read access to the start and stop Date.

TODO: uplink
\paragraph{LR-DTR-040 -- DateRange to strings}
The DateRange shall be convertible to a vector of two Date strings
as defined in LR-DAT-080.

TODO: uplink
\paragraph{LR-DTR-050 -- DateRange to day strings}
The DateRange shall be convertible to a vector of two Date day strings
as defined in LR-DAT-120.

TODO: uplink

\subsubsection{Day}
A Day corresponds to a DateRange covering a single calendar day in
the current TimeZone. A given day starts at midnight and ends at the next
midnight.

\paragraph{LR-DAY-010 -- Now initialization}
The default Day initialization shall be to the current calendar day,
as indicated by the system clock, in the time zone currently set
in TimeZone.

TODO: uplink

\paragraph{LR-DAY-020 -- year/month/day initialization}
Day shall allow initialization from a \lstinline{std::chrono::year_month_day}
object, which represents a calendar day.

TODO: uplink

\paragraph{LR-DAY-030 -- DateRange representation}
The Day shall be convertible to a corresponding DateRange.

TODO: uplink

\paragraph{LR-DAY-040 -- DateRange start and stop Date}
The DateRange obtained from a Day shall have a start Date set
to \lstinline{00:00:00} and stop Date set to \lstinline{00:00:00} of the
following day, in the currently set time zone and for the currently set calendar
day.

TODO: uplink

\paragraph{LR-DAY-050 -- String representation}
The Day shall be convertible to a string representation in format
\lstinline{%d%b%Y %a} (see [AD5]).

For instance, \lstinline{21Jan2025 Tue} is a valid string representation
for a Day.

TODO: uplink

\paragraph{LR-DAY-060 -- Next and previous}
Day shall include methods to select the \emph{next} and \emph{previous} days
from the currently set calendar day.

For instance, if the current Day is set to \lstinline{21Jan2025}, calling
\emph{previous} must change the Day to \lstinline{20Jan2025}.

TODO: uplink

\subsubsection{Week}
A \emph{Week} represents a DateRange from a monday on midnight to midnight of
the next monday, for a particular calendar week.
The dates are as currently defined in the TimeZone.

\paragraph{LR-WEK-010 -- Now initialization}
The Week shall allow initialization at the current calendar week, as defined by
the system clock.

TODO: uplink

\paragraph{LR-WEK-020 -- Date initialization}
The Week shall allow initialization to a week containing a given Date.

TODO: uplink

\paragraph{LR-WEK-030 -- DateRange conversion}
The Week shall be convertible to a DateRange. The start Date of the DateRange
is set to midnight of the monday of the week currently pointed to, and the
stop Date is at midnight on the next monday.

TODO: uplink

\paragraph{LR-WEK-040 -- String representation}
The Week shall be convertible to a string representation in the format
\lstinline{%d%b%Y W%V} (see [AD5]), with the day string \lstinline{%d%b%Y}
set to the monday of the week.

For instance, \lstinline{20Jan2025 W04} is a valid string representation.

TODO: uplink

\paragraph{LR-WEK-050 -- Array of days}
The Week shall be convertible to an array of all the Days contained
in the week, in order from monday to sunday.

TODO: uplink
  
\paragraph{LR-WEK-060 -- Next and previous}
The Week shall include methods to select the \emph{next} and \emph{previous}
weeks from the currently set calendar week.

TODO: uplink

\subsubsection{Duration}
A Duration is a length of time. It does not have a particular location
in terms of date.

\paragraph{LR-DUR-010 -- Zero initialization}
The Duration shall default initialize to a duration of zero.

TODO: uplink
\paragraph{LR-DUR-020 -- Seconds initialization}
The Duration shall allow initialization from a count of seconds.

TODO: uplink
\paragraph{LR-DUR-030 -- Duration resolution}
The Duration shall have a resolution of at most one second.

TODO: uplink
\paragraph{LR-DUR-040 -- Seconds string}
The Duration shall be convertible to a string in format
\lstinline{"x seconds"}, with \lstinline{x} the seconds count
for the duration.

For instance \lstinline{"104592 seconds"} is a valid string.

TODO: uplink
\paragraph{LR-DUR-050 -- Hours string}
The Duration shall be convertible to a string in format
\lstinline{"x hours"}, with \lstinline{x} the hours count
for the duration at three decimal places of precision.

For instance \lstinline{"29.053 hours"} is a valid string.

TODO: uplink
\paragraph{LR-DUR-060 -- Days string}
The Duration shall be convertible to a string in format
\lstinline{"x days"}, with \lstinline{x} the days count
for the duration at three decimal places of precision.
The number of hours in a day must be provided to the
computation externally.

For instance \lstinline{"3.773 days"} is a valid string.

TODO: uplink
\paragraph{LR-DUR-070 -- Seconds short string}
The Duration shall be convertible to a shortened string which
is the one in LR-DUR-040 without the \lstinline{" seconds"}
part.

For instance, \lstinline{"104592"} is a valid string.

TODO: uplink
\paragraph{LR-DUR-080 -- Hours short string}
The Duration shall be convertible to a shortened string
which is the one from LR-DUR-050 without the \lstinline{" hours"}
part.

For instance \lstinline{"29.053"} is a valid string.

TODO: uplink
\paragraph{LR-DUR-090 -- Days short string}
The Duration shall be convertible to a shortened string
which is the one from LR-DUR-060 without the \lstinline{" days"}
part.

For instance \lstinline{"3.773"} is a valid string.

TODO: uplink
\paragraph{LR-DUR-100 -- Zero second short string}
A Duration with a value of 0 shall return a single whitespace
\lstinline{" "} as a second short string.

TODO: uplink
\paragraph{LR-DUR-110 -- Zero hour short string}
A Duration with a value of 0 shall return a single whitespace
\lstinline{" "} as a hour short string.

TODO: uplink
\paragraph{LR-DUR-120 -- Zero day short string}
A Duration with a value of 0 shall return a single whitespace
\lstinline{" "} as a day short string.

TODO: uplink
\paragraph{LR-DUR-130 -- Day string invalid hours}
When given a count of hours in a day which is lesser than or equal to zero,
the day string method shall throw an exception.

TODO: uplink
\paragraph{LR-DUR-140 -- Day short string invalid hours}
When given a count of hours in a day which is lesser than or equal to zero,
the day short string method shall throw an exception.

TODO: uplink
\subsubsection{Duration displayer}
The DurationDisplayer is a singleton for displaying a Duration according
to a currently selected format.

\paragraph{LR-DRD-010 -- DurationDisplayer initialization pass}
The DurationDisplayer shall be initialized with a number of work hours
in a day.

TODO:uplink
\paragraph{LR-DRD-020 -- DurationDisplayer singleton}
The DurationDisplayer shall be a singleton. It is initialized onec and
the same instance is retrieved through a \lstinline{get} method
subsequently.

TODO:uplink
\paragraph{LR-DRD-030 -- Display formats}
The DurationDisplayer shall allow selecting the formats: \emph{hours},
\emph{days}, \emph{seconds}.
These refer to the second, hour and day strings for Duration.

TODO:uplink
\paragraph{LR-DRD-040 -- Default format}
The default format for displaying a Duration upon initialization shall
be \emph{hours}.

TODO:uplink
\paragraph{LR-DRD-050 -- Format cycling}
The currently selected format of the DurationDisplayer shall be selected
through a \emph{cycle} method.

The cycle between formats is as follows:
\begin{itemize}
\item hours $\rightarrow$ days
\item days $\rightarrow$ seconds
\item seconds $\rightarrow$ hours
\end{itemize}

TODO:uplink
\paragraph{LR-DRD-060 -- String display}
Given a Duration, the DurationDisplayer shall output a string depending
on the currently selected format, as follows,
\begin{itemize}
\item hours: string from LR-DUR-050,
\item days: string from LR-DUR-060 with number of hours as initialized in the
            DurationDisplayer,
\item seconds: string from LR-DUR-040.
\end{itemize}

TODO: uplink
\paragraph{LR-DRD-070 -- Short string display}
Given a Duration, the DurationDisplayer shall output a short string depending
on the currently selected format, as follows,
\begin{itemize}
\item hours: string from LR-DUR-080,
\item days: string from LR-DUR-090 with number of hours as initialized in the
            DurationDisplayer,
\item seconds: string from LR-DUR-070.
\end{itemize}

TODO:uplink
\subsection{log\textunderscore lib}
\paragraph{LR-LOG-010}
TODO: uplink

\subsection{db\textunderscore lib}
The db\textunderscore lib contains a handle to a \gls{DB}, and a statement
wrapper.

\subsubsection{DB handle}
\paragraph{LR-DBL-010 DB handle constructor}
The DB handle shall initialize a connection to a DB when provided
with a filepath to either,
\begin{enumerate}
\item a new file in an existing directory,
\item an existing DB file.
\end{enumerate}

\paragraph{LR-DBL-020 DB handle non-existing folder}
The DB handle constructor shall throw an exception when provided with a filepath
to a non-existent folder.

\paragraph{LR-DBL-030 DB handle invalid file}
The DB handle constructor shall throw an exception when provided with a
file which is not a valid DB file, eg a completely unrelated text file.

\paragraph{LR-DBL-040 check DB user version}
The DB handle shall provide a method to check the user version of the DB,
\begin{enumerate}
\item an exception is thrown in case of version mismatch,
\item the version is initialized to the provided number if it was not
      initialized.
\end{enumerate}

\paragraph{LR-DBL-050 get DB user version}
The DB handle shall provide a method to get the user version of the DB.

\paragraph{LR-DBL-060 execute statement}
The DB handle shall provide a method to execute a raw statement, provided
as a string, on an open DB.

\paragraph{LR-DBL-070 execute statement exceptions}
The DB handle shall throw an exception in case a non-nominal statement
execution return code is encountered when executing a statement (LR-DBL-060).

\paragraph{LR-DBL-080 prepare statement}
The DB handle shall provide a method to prepare a statement, provided
as a string, on an open DB.

\paragraph{LR-DBL-090 prepare statement exceptions}
The DB handle shall throw an exception in case an error is encountered when
preparing a statement (LR-DBL-080).

\subsubsection{Statement wrapper}
\paragraph{LR-DBL-100 Execute statement}
A statement shall allow simple execution (single step statement). If needed, the
statement must allow repeated execution.

\paragraph{LR-DBL-110 Execute invalid statement}
Executing an invalid statement shall raise an exception.

\paragraph{LR-DBL-120 Statement execution return value}
The statement execution shall return true on successful execution, and
false if the execution would have violated a database constraint, and thus
failed.

\paragraph{LR-DBL-130 Statement step}
A statement shall allow a step execution (single step in a multi-step
statement).

\paragraph{LR-DBL-140 Statement step invalid}
Stepping an invalid statement shall raise an exception.

\paragraph{LR-DBL-150 Statement step return value}
The stepping of a statement shall return true as long as results
are available for retrieval in the statement, and false when results
are no longer available.

\paragraph{LR-DBL-160 Statement step reset}
The stepping of a statement shall perform an automatic reset when
the end of available results is reached.

\paragraph{LR-DBL-170 Statement parameters binding}
The statements shall allow binding positional parameters of type
integer and string.

\paragraph{LR-DBL-180 Statement column return}
The statements shall allow retrieval of result columns from rows.
The results may be of type boolean, integer or string.

\paragraph{LR-DBL-190 Statement return shortcut}
Statements shall allow a shortcut operation which, in order, steps
the statement, retrieves all the result columns of the first returned
row, and then reset the statement.

\paragraph{LR-DBL-200 Statement binding too many parameters}
Statement parameter binding shall throw an exception if
too many parameters are given.

\paragraph{LR-DBL-210 Statement column return wrong number}
The statement column retrieval shall throw an exception if too many
columns are asked for.

\paragraph{LR-DBL-220 Statement column return wrong type}
The statement column retrieval shall throw an exception if the
queried type does not match the returned type.

\paragraph{LR-DBL-230 Statement column return no rows}
The statement column retrieval shall throw an exception if
there are no rows to give as result.

\paragraph{LR-DBL-240 Statement column return null}
The statement column retrieval shall return the following
default values for each queried type in case the DB returns
NULL as a result.
\begin{enumerate}
\item Boolean: false,
\item Integer: 0,
\item String: the empty string \lstinline{""}.
\end{enumerate}

\subsection{ncurses\textunderscore lib}
\paragraph{LR-NCU-010}
TODO: uplink

\subsection{suggestion\textunderscore lib}
The suggestion library provides a heuristic for selecting a suggestion
based on a set of unique strings and a query string. The suggestion
must feel natural to the user, this is achieved by combining three
elementary heuristics in a cascade:
\begin{enumerate}
\item Substring matching,
\item Substring matching ignoring case,
\item Fuzzy matching.
\end{enumerate}

\paragraph{LR-SUG-010}
Given a set of unique strings to pick from, and a query string,
the suggestion matcher shall return the string chosen from the set
by the following heuristics,
\begin{enumerate}
\item The string returned by the substring matcher, if any,
\item Otherwise, the string returned by the substring matcher
      with case ignored in all strings, if any,
\item Otherwise, the string returned by the fuzzy matcher,
\item Otherwise, the empty string is returned.
\end{enumerate}
TODO: uplink

\paragraph{LR-SUG-020}
Given an empty set of strings to pick from, and a query string,
the suggestion matcher shall return the empty string.
TODO: uplink

\paragraph{LR-SUG-030}
Given an empty query string and any set of strings to pick from,
the suggestion matcher shall return the empty string.
TODO: uplink

\paragraph{LR-SUG-040}
The suggestion matcher shall work nominally for strings containing
any of the characters in the C++ \emph{basic character set} as
defined in [AD6].
TODO: uplink

\paragraph{LR-SUG-050}
Ignoring the case when using the substring matcher shall apply
to alphabet letters only. The other characters are treated
in the same fashion whether case is ignored or not.
TODO: uplink

\subsubsection{Substring matcher}
\paragraph{LR-SUG-060}
Given a set of unique strings to pick from, and a query string,
the substring matcher shall return a string containing the
exact query string as a substring.

For instance, the substring \lstinline{ana} is found in
the string \lstinline{banana}.
TODO: uplink

\paragraph{LR-SUG-070}
In case multiple strings are found with the method described
in LR-SUG-060, a string containing the earliest occurence
of the substring shall be returned.

For instance, given the set \lstinline{bbanana, banana}, and the
query substring \lstinline{ana}, the string \lstinline{banana}
must be returned since it contains the substring at the earliest
position.
TODO: uplink

\paragraph{LR-SUG-080}
In case multiple strings are found with the method described
in LR-SUG-070, a string among the shortest shall be returned.

For instance, given the set \lstinline{banana, bananana}, and
the query substring \lstinline{ana}, the string \lstinline{banana}
must be returned.
TODO: uplink

\paragraph{LR-SUG-090}
In case multiple strings are found with the method described
in LR-SUG-080, the string which comes first in alphabetical order
(extended to an arbitrary ordering for characters outside of letters)
shall be returned. Note that since the items in the string set are unique,
this guarantees only at most one string matches.

For instance, given the set \lstinline{banana, banane}, and the query
substring \lstinline{ana}, the string \lstinline{banana} must be
returned.
TODO: uplink

\subsubsection{Fuzzy matcher}
The fuzzy matcher is less strictly specified than the other matchers. However,
some important properties must be met.

\paragraph{LR-SUG-100}
The fuzzy matcher shall ensure out of order queries are still matched.

For instance, given the set \lstinline{"The brown fox", "jumps over the fence"},
and the query \lstinline{"fence over jumps"}, the fuzzy matcher must
return \lstinline{"jumps over the fence"}.
TODO: uplink

\paragraph{LR-SUG-110}
The fuzzy matcher shall exhibit the fuzzy property. It must match
items despite small typos and change of case
which forbid exact matches.

For instance, given the set \lstinline{Jupiter, Saturn}, and the
query string \lstinline{juppiter}, it must return \lstinline{Jupiter}.
TODO: uplink

\paragraph{LR-SUG-120}
The fuzzy matcher shall not return a match if the query string
has no character in common with any of the strings in the search
set.

For instance, given the set \lstinline{banana, fox} and the query
string \lstinline{yyy}, the fuzzy matcher must match nothing.
TODO: uplink

\section{Modules requirements}
\subsection{version}
The \emph{version} module configures the versions of program components,
to ensure compatibility between different executables and \gls{DB}.

\paragraph{MR-VER-010 -- Program version number}
The \emph{version} module shall provide a program version number
as a string in the following format: \lstinline{x.y},
with \lstinline{x} the major revision number and \lstinline{y} the
minor revision number. The major version number and minor version
number are both integers. Zero is an acceptable version number.

The program version number may be suffixed with \lstinline{dev} to
indicate an internal development version.

Valid program version number examples are:
\begin{itemize}
\item \lstinline{0.5}: for major version 0, minor version 5.
\item \lstinline{3.1}: for major version 3, minor version 1.
\item \lstinline{3.1dev}: for a development version after \lstinline{3.1}.
\end{itemize}

\textit{uplink: } R-REL-010 -- Software version format

\paragraph{MR-VER-020 -- DB version number}
The \emph{version} module shall provide a \gls{DB} version number as
a positive integer.

\textit{uplink: } TODO DB module.

\subsection{config}
The \emph{config} module loads the configuration file, translates the parameters
into an internal representation and transmits them to the rest of the program.

\subsubsection{Configuration file loading}

\paragraph{MR-CON-010 -- Load from path}
The \emph{config} module shall load any compliant user configuration file when
provided with its path.

\textit{uplink: } TODO.

\paragraph{MR-CON-020 -- Non-existent path}
The \emph{config} module shall emit an exception when the path provided for the
configuration file corresponds to a file which does not exist.

\textit{uplink: } TODO.

\paragraph{MR-CON-030 -- Default configuration file locations}
The \emph{config} module shall provide a default load method which looks
for the configuration file \lstinline{timesheeting/timesheeting.toml}
in the following default locations, in order of preference,
\begin{enumerate}
  \item \lstinline{$XDG_CONFIG_HOME}
  \item \lstinline{$HOME/.config/}
  \item \lstinline{/etc/}
\end{enumerate}

\textit{uplink: } TODO.

\paragraph{MR-CON-040 -- Default configuration file not found}
The \emph{config} module default load method shall emit an exception if
no configuration file was found in any of the default locations.

\textit{uplink: } TODO.

\subsubsection{Logger configuration}
\paragraph{MR-CON-050 -- Log filepath}
The \emph{config} module shall load from the configuration file a filepath
for the log file. This parameter is at the node \lstinline{log.file}.

\textit{uplink: } TODO.

\paragraph{MR-CON-060 -- Log levels}
The \emph{config} module shall load from the configuration file a vector of
strings defining the active log levels. This parameter is at the node
\lstinline{log.active_levels}.

\textit{uplink: } TODO.

\paragraph{MR-CON-065 -- Maximum log age}
The \emph{config} module shall load from the configuration file an
unsigned integer defining the maximum age of logs (for log rotation)
in seconds. This parameter is at the node \lstinline{log.max_log_age}.

\textit{uplink: } TODO.

\subsubsection{DB configuration}
\paragraph{MR-CON-070 -- DB filepath}
The \emph{config} module shall load from the configuration file a filepath
for the DB file. This parameter is at the node \lstinline{db.file}.

\textit{uplink: } TODO.

\subsubsection{Timezone configuration}
\paragraph{MR-CON-080 -- Timezone string}
The \emph{config} module shall load from the configuration file a string
representing the timezone to use. This parameter is at the node
\lstinline{time.timezone}.

\textit{uplink: } TODO.

\subsubsection{Duration display configuration}
\paragraph{MR-CON-090 -- Duration display days}
The \emph{config} module shall load from the configuration file a float
representing the number of hours in a workday for duration display.
This parameter is at the node \lstinline{time.hours_per_workday}.

\textit{uplink: } TODO.

\subsubsection{Key bindings configuration}
The key bindings are loaded into maps from key to action. The maps are,
\begin{itemize}
\item The \emph{normal map}, which is the binding map for the normal mode,
\item The \emph{edit map}, which is the binding map for the edit mode.
\end{itemize}

The normal mode is the regular mode of operation of the program, entered
upon program launch. The edit mode is used when typing text.
For the purpose of organization of the user-facing configuration file,
the normal mode bindings are separated into the sections \lstinline{navigation}
and \lstinline{actions}, while the edit mode bindings are under the section
\lstinline{edit\_mode}.

\paragraph{MR-CON-100 -- Binding list}
The \emph{config} module shall load bindings from the following nodes,
organized hierarchically. The top level node is \lstinline{keys}.

The \emph{normal map} bindings are loaded from,
\begin{itemize}
\item navigation
  \begin{itemize}
  \item up
  \item down
  \item left
  \item right
  \item subtabs
  \item previous
  \item next
  \item duration\_display
  \item entries\_screen
  \item projects\_screen
  \item locations\_screen
  \item project\_report\_screen
  \item weekly\_report\_screen
  \item active\_visibility
  \item quit
  \end{itemize}
\item actions
  \begin{itemize}
  \item commit\_entry
  \item set\_now
  \item add
  \item rename
  \item remove
  \item active\_toggle
  \item task\_project\_change
  \end{itemize}
\end{itemize}

The \emph{edit map} bindings are loaded from,
\begin{itemize}
\item edit\_mode
  \begin{itemize}
  \item validate
  \item cancel
  \item select\_suggestion
  \end{itemize}
\end{itemize}

\textit{uplink: } TODO.

\paragraph{MR-CON-110 -- Regular keys}
The \emph{config} module shall allow binding any regular (\textit{ie}
non-escaped and non-special) character key to any binding action.
The regular keys are designated by their character in the configuration
file, \textit{ie} the key \lstinline{e} is designated by the character
\lstinline{"e"} in the configuration file.

\textit{uplink: } TODO.

\paragraph{MR-CON-120 -- Special keys}
The \emph{config} module shall allow binding the following special
keys to any binding action. The special keys are designated by strings.
\begin{itemize}
\item \lstinline{ESCAPE}: the escape key.
\item \lstinline{ENTER}: the enter or \emph{return} key.
\item \lstinline{SPACE}: the spacebar.
\item \lstinline{TAB}: the tabulation key.
\item \lstinline{UP}: the up arrow key.
\item \lstinline{DOWN}: the down arrow key.
\item \lstinline{LEFT}: the left arrow key.
\item \lstinline{RIGHT}: the right arrow key.
\end{itemize}

\textit{uplink: } TODO.

\paragraph{MR-CON-130 -- Multiple keys for an action}
The \emph{config} module shall allow binding multiple keys to the
same action. The keys are specified as a list of strings attached
to the corresponding action node.

For instance, for binding the keys \lstinline{e} and \lstinline{a}
to the action with node \lstinline{rename}, the configuration file
line is: \lstinline{rename = ["e", "a"]}.

\textit{uplink: } TODO.

\paragraph{MR-CON-140 -- Protection against duplicates}
The \emph{config} module shall emit an exception when the configuration
file contains duplicate keys bound in the same map (\textit{ie} either
\emph{normal} or \emph{edit}).

\textit{uplink: } TODO.

\paragraph{MR-CON-150 -- Backspace mapping}
The \emph{config} module shall map the backspace key to a backspace
action in the \emph{edit} map. This is done outside of the configuration
file, it is an implicit binding.

\textit{uplink: } TODO.

\paragraph{MR-CON-160 -- Unbound keys}
The maps in the \emph{config} module shall return an \emph{unbound}
action when queried with a key which was not bound during configuration.

\textit{uplink: } TODO.

\paragraph{MR-CON-170 -- Invalid key strings}
The \emph{config} module shall emit an exception when a key
string in the configuration file does not map to any regular or \emph{special}
characters.

\subsection{cli}
\paragraph{MR-CLI-010}
TODO: uplink

\subsection{core}
The \emph{core} module contains several objects used throughout the program.

\paragraph{MR-COR-010 -- Generic item}
The items Project, Task and Location shall all be represented as objects with
the following attributes,
\begin{enumerate}
\item id,
\item name,
\item active (boolean flag to indicate whether the item is active or not).
\end{enumerate}
TODO: uplink

\paragraph{MR-COR-020 -- Vector of generic item names}
A vector of generic items shall be convertible to a vector of their names,
in order.

\paragraph{MR-COR-030 -- Entry}
Entries shall be represented as objects with the following attributes,
\begin{enumerate}
\item id,
\item project name,
\item task name,
\item start date,
\item stop date,
\item location name.
\end{enumerate}

\paragraph{MR-COR-040 -- Entry to strings}
Entries shall be convertible to a vector of strings for display.
These strings are, in order,
\begin{enumerate}
\item project name,
\item task name,
\item start date display string (long format),
\item stop date display string (long format),
\item location name.
\end{enumerate}

\paragraph{MR-COR-050 -- Entry to short strings}
Entries shall be convertible to a vector of strings with the same composition
as in MR-COR-040, except the dates are outputted in short format.

\paragraph{MR-COR-060 -- Entry staging}
The entry staging state shall be represented as an object with the following
attributes,
\begin{enumerate}
\item project name,
\item task name,
\item start date,
\item stop date,
\item location name.
\end{enumerate}

These attributes are optional, meaning they may hold a value or not.

\paragraph{MR-COR-070 -- Entry staging to strings}
The entry staging object shall be convertible to a vector of string
representations of its attributes, in order, with the dates in
long format. Attributes which do not hold values are converted to a
single whitespace character.

\paragraph{MR-COR-080 -- Entry staging to short strings}
The entry staging object shall be convertible to the same vector
of strings as specified in MR-COR-070 except with the dates in short
format.

\paragraph{MR-COR-090 -- Export row}
Export rows (for CSV file export) shall be represented as an object
with the following attributes,
\begin{enumerate}
\item entry id,
\item project id,
\item project name,
\item task id,
\item task name,
\item location id,
\item location name,
\item start date,
\item stop date.
\end{enumerate}

\paragraph{MR-COR-100 -- Export csv string}
Export rows shall be convertible to a string representation with
comma-separated attributes. The dates are converted to unix timestamps.

\paragraph{MR-COR-110 -- Project total}
Project total (report of time worked per task) shall be represented
as objects with the following attributes,
\begin{enumerate}
\item project name,
\item total (Duration),
\item vector of task totals.
\end{enumerate}

Each task total has the following attributes,
\begin{enumerate}
\item task name,
\item total (Duration).
\end{enumerate}

\paragraph{MR-COR-120 -- Project total to menu items}
Project total objects shall be convertible to menu items for the TUI,
indicating,
\begin{enumerate}
\item The string to display in the cell,
\item The string to display in the status bar,
\item The string face to use (bold, italics etc.).
\end{enumerate}

The menu items are ordered in a nested fashion, with tasks grouped per
project, and the overall project total displayed first in bold.
The menu items are in lexicographic order.

An example of menu items as displayed is shown in
\cref{tab:project_total_menu_items}.

\begin{table} \caption{\label{tab:project_total_menu_items} Menu items for
    project total.}
  \begin{tabular}{| c | c |} \hline
    \textbf{Project1} & \textbf{13.202 hours} \\ \hline
    Task1 & 1.202 hours \\ \hline
    Task2 & 3.000 hours \\ \hline
    Task3 & 9.000 hours \\ \hline
  \end{tabular}
\end{table}

\paragraph{MR-COR-130 -- Weekly totals}
Weekly totals (report of time worked per task in a week) shall be represented
as objects with the following attributes,
\begin{enumerate}
\item total duration worked in the week,
\item total duration worked per day in the week,
\item vector of per project totals.
\end{enumerate}

Note a week has seven days and starts on monday.

Each per project total has the following attributes,
\begin{enumerate}
\item project name,
\item total duration worked on the project in the week,
\item total duration worked on the project per day,
\item vector of per task totals.
\end{enumerate}

Each per task total has the following attributes,
\begin{enumerate}
\item task name,
\item total duration worked on the task in the week,
\item total duration worked on the task per day.
\end{enumerate}

\paragraph{MR-COR-140 -- Weekly totals to menu items}
Weekly total objects shall be convertible to menu items for the TUI,
indicating,
\begin{enumerate}
\item The string to display in the cell,
\item The string to display in the status bar,
\item The string face to use (bold, italics etc.).
\end{enumerate}

The menu items are in lexicographic order. A column header
precedes the actual data. The menu items are, in order,
\begin{enumerate}
\item A header line indicating the columns
  \lstinline{Task, Mon, Tue, Wed, Thu, Fri, Sat, Sun, TOTAL} in normal face.
\item An all-tasks line, indicating the weekly total and daily breakdown of
  the total.
\item Per-project breakdown of weekly totals, with the project total first
  in bold face, following by all individual tasks in normal face.
\end{enumerate}

An example of menu items as displayed is shown in
\cref{tab:weekly_total_menu_items}.

\begin{table} \caption{\label{tab:weekly_total_menu_items} Menu items for
    weekly total.}
  \begin{tabular}{| c | c | c | c | c | c | c | c | c |} \hline
    Task & Mon & Tue & Wed & Thu & Fri & Sat & Sun & TOTAL \\ \hline
    ALL & 13.108 & 7.138 & 4.114 & 9.807
        & 12.498 & 12.386 & 12.039 & 71.090 \\ \hline
    \textbf{Project1} & \textbf{13.108} & \textbf{2.138} & \textbf{2.000}
                      & \textbf{4.807} & \textbf{6.000} & \textbf{7.386}
                      & \textbf{7.039} & \textbf{42.478} \\ \hline
    Task1 & 3.108 & 1.138 & 1.000
          & 2.807 & 3.000 & 2.386
          & 2.039 & 15.478 \\ \hline
    Task2 & 5.000 & 0.500 & 0.400
          & 0.500 & 2.000 & 2.500
          & 3.000 & 13.900 \\ \hline
    Task3 & 5.000 & 0.500 & 0.600
          & 1.500 & 1.000 & 2.500
          & 2.000 & 13.100 \\ \hline
    \textbf{Project2} & & \textbf{5.000} & \textbf{2.114}
                      & \textbf{5.000} & \textbf{6.498} & \textbf{5.000}
                      & \textbf{5.000} & \textbf{28.612} \\ \hline
    Task4 & & 5.000 & 2.114
          & 5.000 & 6.498 & 5.000
          & 5.000 & 28.612 \\ \hline
  \end{tabular}
\end{table}
\subsection{db}
The DB module contains a singleton which interacts with the \gls{DB}.
It is the unique point of direct interaction with the DB in the program,
either in read or write.

\paragraph{MR-DBI-010 -- DB loading}
The DB object shall be loaded using a filepath to the \gls{DB} file.
TODO: uplink

\paragraph{MR-DBI-020 -- DB get user version}
The DB object shall allow retrieving the user version number of the
loaded DB.

\subsubsection{DB tables}
\paragraph{MR-DBI-030 -- Projects table}
The \gls{DB} shall contain a \emph{projects} table with the following columns,
\begin{enumerate}
\item id,
\item name,
\item active (boolean flag).
\end{enumerate}

\paragraph{MR-DBI-040 -- Unique project names}
The \emph{projects} table shall only allow unique names.

\paragraph{MR-DBI-050 -- Tasks table}
The \gls{DB} shall contain a \emph{tasks} table with the following columns,
\begin{enumerate}
\item id,
\item name,
\item project\textunderscore id (id from the \emph{projects} table),
\item active (boolean flag).
\end{enumerate}

\paragraph{MR-DBI-060 -- Unique task names per project}
The \emph{tasks} table shall only allow unique names per project.
Two tasks are allowed to have the same name if they are in different
projects.

\paragraph{MR-DBI-070 -- Project's tasks deletion}
Upon deletion of a project, all its corresponding tasks must also
be deleted.
Note deletion is only possible if no task in the project is present
in entries.

\paragraph{MR-DBI-080 -- Locations table}
The \gls{DB} shall contain a \emph{locations} table with the following columns,
\begin{enumerate}
\item id,
\item name,
\item active (boolean flag).
\end{enumerate}

\paragraph{MR-DBI-090 -- Unique location names}
The \emph{locations} table shall only allow unique names.

\paragraph{MR-DBI-100 -- Entries table}
The \gls{DB} shall contain an \emph{entries} table with the following columns,
\begin{enumerate}
\item id,
\item task\textunderscore id (id from the \emph{tasks} table),
\item start (start date UTC UNIX timestamp),
\item stop (stop date UTC UNIX timestamp),
\item location\textunderscore id (id from the \emph{locations} table).
\end{enumerate}

\paragraph{MR-DBI-110 -- Entries locking hierarchy items removal}
The \emph{locations} and \emph{tasks} referenced to in \emph{entries} shall be
locked for removal, along with their corresponding project. \emph{Locations},
\emph{tasks} and \emph{projects} may only be removed once no part of
\emph{entries} references them.

\paragraph{MR-DBI-120 -- Entries start stop ordering}
The \emph{entries} table shall only allow \emph{start} and \emph{stop}
values for which $start < stop$.

\paragraph{MR-DBI-130 -- Entries non-overlapping dates}
The \emph{entries} table shall only allow items with non-overlapping
\emph{start} and \emph{stop} dates.

\paragraph{MR-DBI-140 -- Entry staging table}
The \gls{DB} shall contain an \emph{entrystaging} table with the following
columns,
\begin{enumerate}
\item task\textunderscore id (id from the \emph{tasks} table),
\item start,
\item stop,
\item location\textunderscore id (id from the \emph{locations} table).
\end{enumerate}

Note the \emph{entrystaging} table only ever has one row.

\subsubsection{Hierarchy items}
\paragraph{MR-DBI-150 -- Inserting projects}
The \gls{DB} shall allow inserting new \emph{projects}, identified by a name
string. The operation must return true for success and false in case
a constraint was violated.

\paragraph{MR-DBI-160 -- Projects active default}
The newly inserted projects shall be active by default.

\paragraph{MR-DBI-170 -- Inserting tasks}
The \gls{DB} shall allow inserting new \emph{tasks}, identified by the parent
project id, and a name string. The operation must return true for success
and false in case a constraint was violated.

\paragraph{MR-DBI-180 -- Tasks active default}
The newly inserted tasks shall be active by default.

\paragraph{MR-DBI-190 -- Inserting locations}
The \gls{DB} shall allow inserting new \emph{locations}, identified by a name
string. The operation must return true for success and false in case a
constraint was violated.

\paragraph{MR-DBI-200 -- Locations active default}
The newly inserted locations shall be active by default.

\paragraph{MR-DBI-210 -- Project name edit}
The \gls{DB} shall allow editing the name of existing \emph{projects}.
The projects are identified by their id. The operation must return true for
success and false in case a constraint was violated.

\paragraph{MR-DBI-220 -- Task name edit}
The \gls{DB} shall allow editing the name of existing \emph{tasks}.
The tasks are identified by their id. The operation must return true for success
and false in case a constraint was violated.

\paragraph{MR-DBI-230 -- Task project edit}
The \gls{DB} shall allow editing the project id of existing \emph{tasks}.
The tasks are identified by their id, and projects by their name.
The operation must return true for success and false in case a
constraint was violated.

\paragraph{MR-DBI-240 -- Location name edit}
The \gls{DB} shall allow editing the name of existing \emph{locations}.
The locations are identified by their id.
The operation must return true for success and false in case a
constraint was violated.

\paragraph{MR-DBI-250 -- Toggle project active}
The \gls{DB} shall allow toggling the active status of existing \emph{projects}.
The projects are identified by their id.

\paragraph{MR-DBI-260 -- Toggle task active}
The \gls{DB} shall allow toggling the active status of existing \emph{tasks}.
The tasks are identified by their id.

\paragraph{MR-DBI-270 -- Toggle location active}
The \gls{DB} shall allow toggling the active status of existing \emph{locations}.
The locations are identified by their id.

\paragraph{MR-DBI-280 -- Query projects}
The \gls{DB} shall allow querying the list of existing \emph{projects}.
The result is in alphabetical order.

\paragraph{MR-DBI-290 -- Query active projects}
The \gls{DB} shall allow querying the list of active \emph{projects}.
The result is in alphabetical order.

\paragraph{MR-DBI-300 -- Query tasks of project}
The \gls{DB} shall allow querying the list of existing \emph{tasks}
for a given project id.
The result is in alphabetical order.

\paragraph{MR-DBI-310 -- Query active tasks of project}
The \gls{DB} shall allow querying the list of active \emph{tasks}
for a given project id.
The result is in alphabetical order.

\paragraph{MR-DBI-320 -- Query locations}
The \gls{DB} shall allow querying the list of existing \emph{locations}.
The result is in alphabetical order.

\paragraph{MR-DBI-330 -- Query active locations}
The \gls{DB} shall allow querying the list of active \emph{locations}.
The result is in alphabetical order.

\paragraph{MR-DBI-340 -- Delete projects}
The \gls{DB} shall allow the deletion of existing \emph{projects},
identified by their id.
The operation must return true for success and false in case a
constraint was violated.

\paragraph{MR-DBI-350 -- Delete tasks}
The \gls{DB} shall allow the deletion of existing \emph{tasks},
identified by their id.
The operation must return true for success and false in case a
constraint was violated.

\paragraph{MR-DBI-360 -- Delete locations}
The \gls{DB} shall allow the deletion of existing \emph{locations},
identified by their id.
The operation must return true for success and false in case a
constraint was violated.

\subsubsection{Entry staging}
\paragraph{MR-DBI-370 -- Entry staging project name}
The \gls{DB} shall allow editing the \emph{entrystaging} project name.
Only active projects may be set.
The operation must return true for success and false in case a
constraint was violated.

\paragraph{MR-DBI-380 -- Entry staging task name}
The \gls{DB} shall allow editing the \emph{entrystaging} task name.
Only active tasks may be set.
The operation must return true for success and false in case a
constraint was violated.

\paragraph{MR-DBI-390 -- Entry staging start}
The \gls{DB} shall allow editing the \emph{entrystaging} start \emph{date}.
The operation must return true for success and false in case a
constraint was violated.

\paragraph{MR-DBI-400 -- Entry staging stop}
The \gls{DB} shall allow editing the \emph{entrystaping} stop \emph{date}.
The operation must return true for success and false in case a
constraint was violated.

\paragraph{MR-DBI-410 -- Entry staging location}
The \gls{DB} shall allow editing the \emph{entrystaging} location name.
Only active locations may be set.
The operation must return true for success and false in case a
constraint was violated.

\paragraph{MR-DBI-420 -- Entry staging commit}
The \gls{DB} shall allow committing the \emph{entrystaging} to an
\emph{entries} item.
The operation must return true for success and false in case a
constraint was violated.

\paragraph{MR-DBI-430 -- Entry staging query}
The \gls{DB} shall allow the retrieval of the current \emph{entrystaging} item.

\paragraph{MR-DBI-440 -- Entry staging project id}
The \gls{DB} shall allow the retrieval of the current \emph{entrystaging}
project id. The return value is optional, as there is no value to
return in the case where no project/task is set in the entry staging.

\subsubsection{Entries}
\paragraph{MR-DBI-450 -- Entry project edit}
The \gls{DB} shall allow the editing of existing \emph{entries} project
by name.
The operation must return true for success and false in case a
constraint was violated.

\paragraph{MR-DBI-460 -- Entry task edit}
The \gls{DB} shall allow the editing of existing \emph{entries} task
by name.
The operation must return true for success and false in case a
constraint was violated.

\paragraph{MR-DBI-470 -- Entry start edit}
The \gls{DB} shall allow the editing of existing \emph{entries} start
\emph{date}.
The operation must return true for success and false in case a
constraint was violated.

\paragraph{MR-DBI-480 -- Entry stop edit}
The \gls{DB} shall allow the editing of existing \emph{entries} stop
\emph{date}.
The operation must return true for success and false in case a
constraint was violated.

\paragraph{MR-DBI-490 -- Entry location edit}
The \gls{DB} shall allow the editing of existing \emph{entries} location
by name.
The operation must return true for success and false in case a
constraint was violated.

\paragraph{MR-DBI-500 -- Entries query}
The \gls{DB} shall allow the retrieval of the list of existing \emph{entries}
in a given \emph{date range}. The entries shall be returned in
increasing order of start date.

\paragraph{MR-DBI-510 -- Entry project id query}
The \gls{DB} shall allow the retrieval of the project id associated to
a given item in \emph{entries}, identified by id.

\paragraph{MR-DBI-520 -- Delete entry}
The \gls{DB} shall allow the deletion of existing \emph{entries}, by id.
The operation must return true for success and false in case a
constraint was violated.

\paragraph{MR-DBI-530 -- Entries duration over date range}
The \gls{DB} shall allow the retrieval of the total \emph{duration} worked
on \emph{entries} in a given \emph{date range}.

\subsubsection{Reports}
\paragraph{MR-DBI-540 -- Entries export}
The \gls{DB} shall allow the export of entries to \emph{export row} format,
over a given \emph{date range}. The exported rows must be ordered by
increasing entry start date.

\paragraph{MR-DBI-550 -- Project total}
The \gls{DB} shall allow the generation of a \emph{project total} report
over a given \emph{date range}. The projects must be ordered alphabetically,
and the same is required for the set of tasks inside each project.

\paragraph{MR-DBI-560 -- Weekly totals}
The \gls{DB} shall allow the generation of a \emph{weekly totals} report
over a given \emph{week}.

\subsection{keys}
\paragraph{MR-KEY-010}
TODO: uplink

\subsection{tui}
\paragraph{MR-TUI-010}
TODO: uplink

\subsection{exporter}
\paragraph{MR-EXP-010 -- Export file}
The exporter module shall produce an export file compliant with [AD7],
given a set of entry \emph{export rows} in a given \emph{date range},
and the filepath to the target export file.

\paragraph{MR-EXP-020 -- Export file directory exception}
If a directory is provided to the exporter module instead of a filepath,
then an exception shall be thrown.

\paragraph{MR-EXP-030 -- Export file exists exception}
If the filepath provided to the exporter module corresponds to
a file which already exists, then an exception shall be thrown.


\appendix

%% \include{appendices}

\apptocmd{\thebibliography}{\raggedright}{}{}
\begingroup
\setstretch{0.6}
\setlength\bibitemsep{0pt}
\printbibliography
\endgroup
\end{document}
