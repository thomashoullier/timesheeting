\section{Requirements}
\subsection{UHI -- User hierarchy interaction}
\paragraph{R-UHI-010 -- Adding hierarchy items}
The software shall allow the user to add hierarchy items.

\paragraph{R-UHI-020 -- Removing hierarchy items}
The software shall allow the user to remove hierarchy items
from the active items set.

\paragraph{R-UHI-030 -- Editing hierarchy items}
The software shall allow the user to edit the properties of
active hierarchy items.

\paragraph{R-UHI-040 -- Restoring deleted hierarchy items}
The software shall allow the restoration of previously deleted
hierarchy items. The corresponding identification must be the same as before
deletion.

\paragraph{R-UHI-050 -- Project removal effect}
The removal of a project from the active set of projects shall cause
the corresponding tasks to be removed.

\subsection{UEI -- User entries interaction}
\paragraph{R-UEI-010 -- Adding entries}
The software shall allow the user to add new entries.

\paragraph{R-UEI-020 -- Adding entries through stopwatch}
The software shall allow the user to use a stopwatch interface to
add new entries. The entry is saved to memory when the stopwatch stops.

\paragraph{R-UEI-030 -- Adding entries manually}
The software shall allow the user to add new entries by manually filling
the properties.

\paragraph{R-UEI-040 -- Removing entries}
The software shall allow the user to remove entries.

\paragraph{R-UEI-050 -- Editing entries}
The software shall allow the user to edit past entries properties.

\subsection{UGL -- User graphical layout}
\paragraph{R-UGL-010 -- UI screens breakdown}
The software shall present to the user the following UI screens:

\begin{compactitem}
\item (interaction) Daily entries
\item (interaction) Hierarchy items
\item (report) Project totals
\item (report) Weekly report
\item Export tool
\end{compactitem}

\subsection{DES -- Daily entries screen}
\paragraph{R-DES-010 -- Daily entries}
The daily entries UI screen shall implement the user interface for adding,
editing and removing entries.

\paragraph{R-DES-020 -- Day selection}
The daily entries UI screen shall allow the user to select the day for which
entries must be interacted with.

\paragraph{R-DES-030 -- Display entries of the day}
The daily entries UI screen shall display the list of entries for the currently
selected day.

\paragraph{R-DES-040 -- Running daily total}
The daily entries UI screen shall display the running duration of time worked
on the selected day. This includes any currently running stopwatch for the
current day.

\subsection{STP -- Stopwatch}
\paragraph{R-STP-010 -- Stopwatch in use}
The software shall display to the user an indicator when the stopwatch is
running. This indicator must be visible independently of the UI screen
the user is in.

\paragraph{R-STP-020 -- Running stopwatch time}
The daily entries UI screen shall display the current running duration
of the stopwatch.

\paragraph{R-STP-030 -- Stopwatch only on current day}
The stopwatch interface shall only add new entries to the current day
the system is in. Ie the stopwatch cannot add entries in the past or in
the future.

\subsection{ENI -- Entry interaction}
\paragraph{R-ENI-010 -- Entry metadata prefill}
The entry metadata creation and edition interface shall be pre-filled with
the metadata from the last added or edited entry.

\paragraph{R-ENI-020 -- Entry metadata suggestion}
The entry metadata creation and edition interface shall be fillable through
a short list of suggestions from the latest 5 added or edited entries.

\paragraph{R-ENI-030 -- Entry metadata hierarchy search}
The entry metadata fields relevant to hierarchy items shall be fillable
through a fuzzy search over the corresponding set of active hierarchy items.

\paragraph{R-ENI-040 -- Entry metadata hierarchy coherence}
The entry metadata interface shall forbid the creation or edition of
hierarchy items metadata outside of the set of active hierarchy items.

\subsection{HIS -- Hierarchy items screen}
\paragraph{R-HIS-010 -- Hierarchy items}
The hierarchy items UI screen shall implement the user interactions with
hierarchy items.

\paragraph{R-HIS-020 -- Hierarchy items display}
The hierarchy items UI screen shall display the list of active tasks
grouped per project.

\subsection{GUI -- Graphical user interface}
\paragraph{R-GUI-010 -- Keyboard usage}
The software shall allow full user interaction through a keyboard interface.

\subsection{LDC -- Logged data content}
\paragraph{R-LDC-010 -- Entry identification}
The entries shall be identified with a unique code.

\paragraph{R-LDC-020 -- Entry metadata}
The entries shall be associated with the following metadata unambiguously,
\begin{compactitem}
\item a task,
\item a start date,
\item a stop date,
\item a location.
\end{compactitem}

\paragraph{R-LDC-030 -- Company identification}
The company shall be identified by a shorthand for its name.

\paragraph{R-LDC-040 -- Company metadata}
The company shall be associated to its full name.

\paragraph{R-LDC-050 -- Project identification}
Projects shall be identified by a unique code.

\paragraph{R-LDC-060 -- Project metadata}
Projects shall be associated unambiguously to their,
\begin{compactitem}
  \item full name,
  \item external reference number.
\end{compactitem}

\paragraph{R-LDC-070 -- Task identification}
Tasks shall be identified by a unique code.

\paragraph{R-LDC-080 -- Task metadata}
Tasks shall be associated unambiguously to their,
\begin{compactitem}
  \item full name,
  \item project.
\end{compactitem}

\subsection{TIM -- Time management}
\paragraph{R-TIM-010 -- Time standard}
The software shall use UTC time internally as its datation format.

Rationale: we need to maintain the chronology of events in an interoperable
fashion.

\paragraph{R-TIM-020 -- Time reference}
The software datation shall use the system clock as its time reference.

\paragraph{R-TIM-030 -- Time zones}
The software shall display dates using a user-configurable timezone.

\paragraph{R-TIM-040 -- Time resolution}
The dates and durations in the software shall be saved and operated
on with a resolution of one second.

\subsection{SAV -- Save management}
\paragraph{R-SAV-010 -- Save}
The software shall provide a mechanism for saving the timesheet data
and hierarchy items to the system's persistent memory storage.

\paragraph{R-SAV-020 -- Transparent save}
The saving mechanism shall be transparent to the user.

\paragraph{R-SAV-030 -- Timesheet save resolution}
The saving mechanism shall save timesheet data anytime an entry is
created, modified or deleted.

\paragraph{R-SAV-031 -- Hierarchy items save resolution}
The saving mechanism shall save the hierarchy items data anytime
a hierarchy item is created, modified, deleted or restored.

\paragraph{R-SAV-040 -- Save status}
The system shall report visually to the user the save status, ie
the following states,
\begin{compactitem}
  \item save in sync with the system's persistent storage,
  \item save in progress,
  \item save failed.
\end{compactitem}

\paragraph{R-SAV-050 -- Switch saved state}
The software shall allow the user to switch between different save profiles.

\subsection{BAK -- Backup}
\paragraph{R-BAK-010 -- Backup}
The software shall provide a backup mechanism to the user.

\paragraph{R-BAK-020 -- Backup restore}
The software shall allow a backup restore mechanism to the user.

\paragraph{R-BAK-030 -- Backup completeness}
The whole data state of the software shall be saved and restored using the
backup. Logs are not included in the data state.

\paragraph{R-BAK-040 -- Backup conciseness}
The backup archive shall consist in a single file.

\paragraph{R-BAK-050 -- Backup timestamp}
The backup archive shall be timestamped with the date at which the backup was
made.

\paragraph{R-BAK-060 -- Backup naming}
The backup mechanism shall allow the user to choose a custom name for the
archive.

\paragraph{R-BAK-070 -- Backup location}
The backup mechanism shall allow the user to choose the location of the
archive on the system's directory tree.

\subsection{DEX -- Data export}
\paragraph{R-DEX-010 -- Timesheet export}
The software shall allow the user to export timesheet data in an interoperable
format compliant with [AD1].

\paragraph{R-DEX-020 -- Export naming}
The timesheet export process shall allow the user to name the export file.

\paragraph{R-DEX-030 -- Export location}
The timesheet export process shall allow the user to select the location
of the export file on the system's directory tree.

\subsection{ACC -- Accessibility}
\paragraph{R-ACC-010 -- Single user}
The software shall allow operation by a single user on a single system.

\paragraph{R-ACC-020 -- Synchronization across systems}
The software shall document a process for the user to keep timesheet data
saves in sync across systems. Note we assume the reliance upon external
data exchange mechanisms outside of the software's technical scope.

\paragraph{R-ACC-030 -- Company segregation}
The software shall keep the timesheet data segragated by company.
In effect, this means a given saved dataset is uniquely associated with a
company.

\paragraph{R-ACC-040 -- Data confidentiality}
The software interface and saved data shall remain local to the system.

\paragraph{R-ACC-050 -- Offline operation}
The software shall allow offline (no internet connection) operation for all of
its features. This excludes the build system.

\subsection{ENV -- Environment}
\paragraph{R-ENV-010 -- Target hardware}
The software shall run on a low-end desktop computer or laptop.
The representative system to consider is a desktop computer with
an Athlon 3000G CPU and a entry-level SSD.

\paragraph{R-ENV-020 -- Target OS}
The software shall run on GNU/Linux.

\paragraph{R-ENV-030 -- Target OS version}
The software shall run on the current version of Gentoo Linux.

\paragraph{R-ENV-040 -- Target graphical environment}
The software shall run on the Wayland compositor Hyprland \cite{hyprland}.

\subsection{PER -- Performance}
\paragraph{R-PER-010 -- Memory footprint}
The peak memory footprint of the software shall be less than 100 MB.
The amount of memory to consider is the \emph{resident set size}.

\subsection{URE -- User reports}
\paragraph{R-URE-010 -- Durations display format}
In the reports, the durations shall be displayed with a selectable format,
either,
\begin{compactitem}
  \item minutes,
  \item or hours,
  \item or days.
\end{compactitem}

\subsection{RPT -- Report: Project totals}
\paragraph{R-RPT-010 -- Project totals}
The project totals UI screen shall display the total durations of time
worked on projects within a user-specified time period.

\paragraph{R-RPT-020 -- Project totals time period}
The project totals UI shall allow the user to select the time period
of interest via the selection of a beginning day and an end day. The
beginning and end days are included in the time period.

\subsection{RWT -- Report: Weekly totals}
\paragraph{R-RWT-010 -- Weekly report}
The weekly report UI screen shall display the task-wise daily total
durations of time worked, grouped per project.

\paragraph{R-RWT-020 -- Weekly report daily totals}
The weekly report shall display the total duration worked per day, including
all tasks in the sum.

\paragraph{R-RWT-030 -- Weekly report weekly totals}
The weekly report shall display the total weekly work duration, including
all tasks in the sum.

\paragraph{R-RWT-040 -- Weekly report running week}
The weekly report shall be generated even in the case of the current, unfinished
week.

\paragraph{R-RWT-050 -- Weekly report week selection}
The weekly report UI shall allow the user to select the week the report is
generated for.

\paragraph{R-RWT-070 -- Weekly report timesheet export}
The weekly report UI shall allow the user to export the timesheet data in
a format compliant with AD1.

\subsection{LOG -- Logging}
\paragraph{R-LOG-010 -- User data interaction logging}
The software shall log all user interactions which modify the timesheet
data or the hierarchy items data.

\paragraph{R-LOG-020 -- Log file location}
The software logs shall be saved on the system's persistent memory.

\paragraph{R-LOG-030 -- Log depth}
The software logs shall have a depth of at least one week. Ie the
log is kept in persistent memory for at least one week.

\paragraph{R-LOG-040 -- Log cleanup}
The software log entries older than 1 month shall be deleted.

\paragraph{R-LOG-050 -- Log cleanup schedule}
The software log cleanup shall apply at every software start.

\paragraph{R-LOG-060 -- Log readability}
The software log shall be stored in plain text format readable
by the \emph{less} program.

\paragraph{R-LOG-070 -- Log accessibility}
The software log shall be readable by the user even in case the software
fails to start entirely.

\subsection{QUA -- Quality}
\paragraph{R-QUA-010 -- Version report}
The software shall display its version number to the user upon GUI query.

\paragraph{R-QUA-020 -- Saved data validation}
The software shall provide a mechanism for testing the soundness of the
saved data upon loading it.

\paragraph{R-QUA-030 -- Release signature}
Every released product, either software or documentation shall be signed
with the supplier's GPG key.

\paragraph{R-QUA-040 -- Single repository}
For the whole project lifecycle, the software and associated documentation shall
be stored in a single version-controlled repository.

\subsection{TES -- Testing}
\paragraph{R-TES-010 -- Automated build test}
The software build system shall include an automated build test, reporting
whether the build is successful in the target environment.
The target environment for building is the same as the target environment for
the software.

\subsection{DOC -- Documentation}
\paragraph{R-DOC-010 -- Development documentation}
The software documentation shall include a developer design document (eg.
Doxygen).

\paragraph{R-DOC-020 -- User manual}
The software documentation shall include a user manual (eg. troff man page).

\paragraph{R-DOC-030 -- Keyboard cheatsheet}
The software documentation shall include a cheatsheet of the keyboard commands.

\paragraph{R-DOC-040 -- Keyboard cheatsheet conciseness}
The keyboard cheatsheet shall be at most two pages long.

\paragraph{R-DOC-050 -- Software build instructions}
The software documentation shall include the instructions for building
the software from source.

\paragraph{R-DOC-060 -- Documentation build instructions}
The software documentation shall include the instructions for building
the documentation artifacts.

\paragraph{R-DOC-070 -- Matrix of conformity}
A conformity matrix with respect to the present specification document shall
be produced and released along with every major software release.

\paragraph{R-DOC-080 -- Architecture and design document}
The software documentation shall include a document describing the architecture
and design choices.

\subsection{REL -- Release}
\paragraph{R-REL-010 -- Software version format}
The software version shall include a major version and a minor version.

\paragraph{R-REL-020 -- Release notes}
The software documentation shall include release notes.

\paragraph{R-REL-030 -- Release notes granularity}
The release notes shall apply to individual minor software versions.

\paragraph{R-REL-040 -- Release notes publication}
The release notes shall be updated and released at least for every minor
version of the software.

\paragraph{R-REL-050 -- Documentation release}
The full up-to-date software documentation shall be released along with
every major software version release.

\paragraph{R-REL-060 -- Build dependencies}
The software build dependencies shall be automatically downloaded or included by
the build system.

\subsection{DEP -- Deployment}
\paragraph{R-DEP-010 -- Installation script}
The software shall provide an installation script.

\paragraph{R-DEP-020 -- Uninstallation script}
The software shall provide an uninstallation script.

\subsection{LIC -- Licensing}
\paragraph{R-LIC-010 -- Source code license}
The software license for the source code shall be a permissive open-source
license.

\paragraph{R-LIC-020 -- Documentation license}
The software documentation shall be licensed under license at most as
restrictive as the CC-BY license.