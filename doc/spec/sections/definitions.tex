\section{Definitions}
We define the concepts used within the project.

\paragraph{The software} The product answering the present requirements.
\paragraph{User} The person using the \emph{software}.
\paragraph{Work unit} The elementary real-life description of what the
  \emph{user} needs to log in the \emph{timesheets}. For instance, this could be
  a list of \emph{project}, \emph{task}, and \emph{start/stop dates}.
\paragraph{Company} The company the \emph{user} works at, for which the
  \emph{tasks} are accomplished.
\paragraph{Project} The project the \emph{user} works on when logging
  \emph{work units}.
\paragraph{Task} The particular element of work being done on a given
  \emph{project}, according to a project-wise subdivision.
\paragraph{Location} The place where a given \emph{work unit} is performed
  by the \emph{user}.
\paragraph{Hierarchy items} The category of items which are either a
  \emph{project}, a \emph{task} or a \emph{location}.
\paragraph{Active hierarchy items} The \emph{hierarchy items} which are
  visible to the \emph{user} when inputting \emph{work units} data into the
  software.
\paragraph{Inactive hierarchy items} The \emph{hierarchy items} which are
  not available to the \emph{user} when inputting new \emph{work units}
  data into the software.
\paragraph{Start/Stop dates} The dates at which a given \emph{work unit} is
  started and at which it is stopped, respectively.
\paragraph{Duration} The time difference between the \emph{start} and the
  \emph{stop dates}, in the context of a \emph{work unit}.
\paragraph{Entry} The data representation of a \emph{work unit}.
\paragraph{Time period} A continuous set of dates defined by a beginning
  date and an end date.
\paragraph{Timesheet} A collection of \emph{entries} in a given \emph{time
  period}.
\paragraph{\gls{UI} screen} A self-standing \gls{GUI} view presented to the
\emph{user} by the \emph{software}, for instance a tab.
\paragraph{Save profile} A segregated \emph{user} identity for managing save data.
