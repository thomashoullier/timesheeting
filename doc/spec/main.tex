%% Document-wide settings.
\documentclass[letterpaper]{article}
\title{timesheeting specification document}
\author{Thomas HOULLIER}

\usepackage[colorlinks=true, allcolors=blue,
            hyperfootnotes=false,
            pdfauthor={Thomas HOULLIER},
            pdftitle={timesheeting specification document}]
            {hyperref} % Links for ref/cite.

%% Loading packages
\usepackage{amsmath} % For cases in equations.
\usepackage{amsfonts} % For maths sets.
\usepackage{physics} % For \abs{} and \norm{}.
\usepackage[inkscapelatex=false]{svg} %svg graphics
\usepackage{siunitx} % units formatting

\usepackage[backend=biber,style=numeric,citestyle=numeric-comp,maxcitenames=99,dateabbrev=false]{biblatex}
\addbibresource{biblio.bib}
\usepackage{setspace} % Bibliography spacings
\DeclareSourcemap{
  \maps[datatype=bibtex]{
    \map[overwrite]{
      \step[fieldsource=doi, final]
      \step[fieldset=url, null]
      \step[fieldset=eprint, null] }}}
\setcounter{biburllcpenalty}{7000} % break long url in bibliography
\setcounter{biburlucpenalty}{8000}
\renewcommand*{\bibfont}{\footnotesize} % bibliography font size
% Format of biblatex urldate in the bibliography.
\DeclareFieldFormat{urldate}{%
  Visited on \thefield{urlday}\addspace%
  \mkbibmonth{\thefield{urlmonth}}\addspace%
  \thefield{urlyear}\isdot}
\usepackage[ruled,vlined]{algorithm2e} % Algorithms.
\DontPrintSemicolon
\SetKwInOut{Input}{Input}\SetKwInOut{Output}{Output}
\usepackage{mathtools} % Ceiling function.
\usepackage{outlines} % Nest lists.
\usepackage{interval} % Writing intervals.
\usepackage[font={footnotesize,sf}]{caption} %Caption for figures in minipages.
\usepackage{floatrow}
% Figure captions always below. Figures always centered.
\floatsetup[figure]{capposition=bottom,objectset=centering}
\usepackage{wrapfig} %Wrapping figure with text.
\usepackage{stmaryrd} % Double brackets for integers interval.
\usepackage{doi} % Hyperlink DOI
\usepackage{etoolbox} %Ragged right bibliography.
\usepackage{color, colortbl} % Coloring rows in tables.
\usepackage{subcaption} % Subfigures.
\usepackage{pdfpages} % Include PDF pages.
\usepackage{epigraph} % Quotations at beginning of chapters.
\setlength\epigraphwidth{.8\textwidth}
\usepackage[acronym,nonumberlist,nogroupskip,nopostdot]{glossaries} % Glossary for acronyms.
\renewcommand*{\glstextformat}[1]{\textcolor{black}{#1}} % No color on links for abbrev.

\DeclarePairedDelimiter{\ceil}{\lceil}{\rceil} % Ceiling function.
\DeclarePairedDelimiter{\floor}{\lfloor}{\rfloor} % Floor function.

\DeclareMathOperator*{\argmin}{argmin}

\setcounter{tocdepth}{3} % Table of content depth
\setcounter{secnumdepth}{3} % Section numbering depth

% Non-breaking around footnotes.
\makeatletter
\let\Footnote\footnote
\def\pst@@killglue{\unskip\ifdim\lastskip>\z@\expandafter\pst@@killglue\fi}
\def\footnote{\pst@@killglue\Footnote}
\makeatother

% More space below equations
\appto\normalsize{\belowdisplayshortskip=\belowdisplayskip}

% Rewrite month codes in bibliography
\DeclareSourcemap{
  \maps[datatype=bibtex]{
    \map[overwrite]{
      \step[fieldsource=month, match=\regexp{\A(j|J)an(uary)?\Z}, replace=1]
      \step[fieldsource=month, match=\regexp{\A(f|F)eb(ruary)?\Z}, replace=2]
      \step[fieldsource=month, match=\regexp{\A(m|M)ar(ch)?\Z}, replace=3]
      \step[fieldsource=month, match=\regexp{\A(a|A)pr(il)?\Z}, replace=4]
      \step[fieldsource=month, match=\regexp{\A(m|M)ay\Z}, replace=5]
      \step[fieldsource=month, match=\regexp{\A(j|J)un(e)?\Z}, replace=6]
      \step[fieldsource=month, match=\regexp{\A(j|J)ul(y)?\Z}, replace=7]
      \step[fieldsource=month, match=\regexp{\A(a|A)ug(ust)?\Z}, replace=8]
      \step[fieldsource=month, match=\regexp{\A(s|S)ep(tember)?\Z}, replace=9]
      \step[fieldsource=month, match=\regexp{\A(o|O)ct(ober)?\Z}, replace=10]
      \step[fieldsource=month, match=\regexp{\A(n|N)ov(ember)?\Z}, replace=11]
      \step[fieldsource=month, match=\regexp{\A(d|D)ec(ember)?\Z}, replace=12]}}}

% Footnotes marker color
\renewcommand\thefootnote{\textcolor{blue}{\arabic{footnote}}}

\pdfsuppresswarningpagegroup=1 % Silence warnings about pagegroups for figures.
\pdfminorversion=6 % PDF version 1.6 since we include articles in 1.6.

% Allow an extra pass to fix overfull hboxes by allowing more whitespace.
\emergencystretch=1em

% Page numbering and copyright notice.
\usepackage{fancyhdr}
\usepackage{lastpage}

\fancypagestyle{FirstPage}{
\fancyhf{} % Clear footer.
\rfoot{\thepage \hspace{1pt} of \pageref*{LastPage}}
\renewcommand{\headrulewidth}{0pt} % Remove rule at top of page
\lfoot{\href{https://creativecommons.org/licenses/by/4.0/}
       {\includesvg[inkscapelatex=false,height=14pt]{images/ccby.svg}}}}

\fancypagestyle{plain}{
\fancyhf{} % Clear footer.
\rfoot{\thepage \hspace{1pt} of \pageref*{LastPage}}
\renewcommand{\headrulewidth}{0pt} % Remove rule at top of page
}

% Version history
\usepackage{vhistory}

% Keywords
\providecommand{\keywords}[1]{\textbf{Keywords --} #1}

% Glossary
\makeglossaries
\loadglsentries{../glossary/glossary.tex}

\usepackage{fontawesome} %inline icons
\usepackage{xcolor}
\usepackage{listings} % Code listings
\definecolor{codeback}{rgb}{0.99,0.99,0.98}
\definecolor{codecomment}{HTML}{0588fc}
\definecolor{codekeyword}{HTML}{af5f00}
\definecolor{codestring}{HTML}{ffa07a}
\lstdefinestyle{mystyle}{
  backgroundcolor=\color{codeback},
  commentstyle=\color{codecomment},
  keywordstyle=\color{codekeyword},
  stringstyle=\color{codestring},
  basicstyle=\ttfamily\footnotesize,
  breakatwhitespace=false,         
  breaklines=true,                 
  captionpos=b,                    
  keepspaces=true,                 
  numbers=left,                    
  numbersep=5pt,                  
  showspaces=false,                
  showstringspaces=false,
  showtabs=false,                  
  tabsize=2}
\lstset{style=mystyle}

\usepackage[capitalise,nameinlink]{cleveref} % Include eg. "Fig." in front of figures.
\crefname{algorithm}{Alg.}{Algs.}
\crefname{table}{Tab.}{Tabs.}
\crefname{equation}{Eq}{Eqs.}
% Equation cross-references.
%\creflabelformat{equation}{#2#1#3}
\crefformat{equation}{(#2Eq.\thinspace#1#3)}

% No parentheses in equation labels.
%\newtagform{noparen}{}{}
%\usetagform{noparen}

% Document
\begin{document}
\frenchspacing
\date{PRJ1-SPE1-v0.0 -- \today}
\maketitle
\thispagestyle{FirstPage}

\begin{abstract}
  This is the specification document for the timesheeting project.
  It describes the requirements for a time management software for
  personal timesheets.
\end{abstract}

\begin{versionhistory}
\vhEntry{0.0}{XXX}{TH}{creation}
\end{versionhistory}
\setcounter{table}{0} % Reset the table counter.

\section*{Applicable documents}
{ \centering
\begin{tabularx}{\textwidth}{| c | X | c | c | X |} \hline
  Index & Title & Reference & Revision & Author \\ \hline
  AD1   & External timesheet format & PRJ1-IRS1 & v1.0 & Thomas HOULLIER \\
\hline \end{tabularx} \par }
\tableofcontents
\printglossary[type=\acronymtype,style=index]
\pagestyle{plain}
\section{Introduction}
\subsection{Context}
In support of the project requirement R-DEX-010 [RD1], we provide a format
for timesheet data export to a file.

The format is meant to be interoperable, \textit{ie} it is meant to be
easily usable with a wide selection of external programs. The external
programs typically targetted are spreadsheet programs (\emph{Libreoffice Calc},
\emph{Gnumeric}), \gls{CLI} text programs (\emph{less}, \emph{vi}),
and Python libraries such as \emph{pandas}.

We prioritize ease of use and readability for the exported format over
compactness and efficiency.

\subsection{Document structure}
The document is structured as follows. First, definitions are given
(\cref{sec:definitions}), then the requirements for the exported file
are listed (\cref{sec:requirements}) and finally an example of a compliant
export file is given (\cref{sec:example}).

\section{Definitions}
We define the concepts used within the project.

\begin{itemize}
\item \textbf{The software}: The product answering the present requirements.
\item \textbf{User}: The person using the \emph{software}.
\item \textbf{Work unit}: The elementary real-life description of what the
  \emph{user} needs to log in the \emph{timesheets}. For instance, this could be
  a list of \emph{project}, \emph{task}, and \emph{start/stop dates}.
\item \textbf{Company}: The company the user works at, for which the
  \emph{tasks} are accomplished.
\item \textbf{Project}: The project the \emph{user} works on when logging
  \emph{work units}.
\item \textbf{Task}: The particular element of work being done on a given
  \emph{project}, according to a project-wise subdivision.
\item \textbf{Hierarchy items}: The category of items which are either a
  \emph{project} or a \emph{task}.
\item \textbf{Active hierarchy items}: The \emph{hierarchy items} which are
  visible to the \emph{user} when inputting \emph{work units} data into the
  software.
\item \textbf{Start/Stop dates}: The dates at which a given \emph{work unit} is
  started and at which it is stopped, respectively.
\item \textbf{Duration}: The time difference between the \emph{start} and the
  \emph{stop dates}.
\item \textbf{Stopwatch}: A tool for tracking the \emph{start and stop dates} of
  a \emph{work unit}.
\item \textbf{Location}: The place where a given \emph{work unit} is performed
  by the \emph{user}.
\item \textbf{Entry}: The data representation of a \emph{work unit}.
\item \textbf{Time period}: A continuous set of dates defined by a beginning
  date and an end date.
\item \textbf{Timesheet}: A collection of entries in a given time period.
\end{itemize}

\appendix
\cleardoublepage

%% \include{appendices}

\apptocmd{\thebibliography}{\raggedright}{}{}
\begingroup
\setstretch{0.6}
\setlength\bibitemsep{0pt}
\printbibliography
\endgroup
\end{document}
