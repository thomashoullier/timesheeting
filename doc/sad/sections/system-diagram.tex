\section{System architecture diagram} \label{sec:system-diagram}
The proposed system is decomposed into the following subsystems. We distinguish
between the subsystems active while the user interacts with the software (online
subsystems), and the auxiliary susbsystems (offline subsystems). We outline
the function of each subsystem and provide a diagram for illustration.

\subsection{Online subsystems}
The software comprises the following online subsystems,
\begin{itemize}
\item The \textbf{\gls{UI}} displays the \gls{GUI} screens and makes the
  \gls{CLI} available to the user. The \gls{GUI} and \gls{CLI} both allow
  user input.
\item The \textbf{Core logic} handles the manipulation of timesheet data for
  presentation to the user. It decouples the \gls{UI} subsystem from the rest
  of the application.
\item The \textbf{Configuration manager} loads the user configuration file, and
  propagates the settings to the application.
\item The \textbf{Exporter} is in charge of generating the export file
  for timesheet data.
\item The \textbf{\gls{DB} backend} interacts with the \gls{DB}.
\item The \textbf{Logger} records the application logs.
\end{itemize}

\cref{fig:arch-diagram} illustrates the interaction between these
subsystems.

The data files involved in the operation of the software are,
\begin{itemize}
\item The \textbf{User configuration file} allows the user to configure the
  application. This is a text file the user edits.
\item The \textbf{Logs} record application events for debugging purposes.
\item The \textbf{Database} allows the application to store and retrieve
  data contents. It is persistent.
\item The \textbf{Export file} is a user-readable file generated on-demand
  from the timesheet data.
\end{itemize}

\begin{figure}
  \includesvg[width=\textwidth]{images/architecture-diagram.svg}
  \caption{\label{fig:arch-diagram} System architecture diagram.}
\end{figure}

\subsection{Offline subsystems}
The software comprises the following offline subsystems,
\begin{itemize}
\item The \textbf{Versioning} system tracks the state of both the software
  source and documentation source.
\item The \textbf{Software distribution} system manages the release of
  software source to the user.
\item The \textbf{Build} system generates a binary from the software source.
\item The \textbf{Automated testing} comprises the tests that are run
  automatically for every software release. It is part of the \gls{CI} pipeline.
\item The \textbf{Manual testing} comprises the tests performed by the
  \textbf{Tester}, manually, for every software release. It is part of the
  tester pipeline.
\item The \textbf{Documentation generator} generates the documentation
  releases from the documentation source.
\item The \textbf{Documentation distribution} manages the release of
  documentation reports.
\item The \textbf{Signature system} authenticates the released source and
  released documentation as coming from the author.
\item The \textbf{Issue tracking} allows the \emph{user} to file bug reports.
\end{itemize}

\cref{fig:offline-diagram} illustrates the interaction between these subsystems.

\begin{figure}
  \includesvg[width=\textwidth]{images/offline-diagram.svg}
  \caption{\label{fig:offline-diagram} Offline subsystems diagram.}
\end{figure}


There are three build pipelines in parallel,
\begin{itemize}
\item The \textbf{User} pipeline: the \emph{user} receives the \emph{released
    source}, and uses the \emph{build} system to generate a binary. The
    \emph{user} then uses the binary.
  \item The \textbf{\gls{CI}} pipeline: The software source for every version
    are automatically sent to a \gls{CI} system, where the source are built into
    a binary. Assuming the build was successful, automated tests are then run
    on the binary. The results are reported publicly in the software
    distribution system.
  \item The \textbf{Tester} pipeline: The software source is built by the tester
    using the build system. The binary is then tested against a set of manual
    cases.
\end{itemize}
