\section{System architecture diagram}
The proposed system is decomposed into the following subsystems. We distinguish
between the subsystems active while the user interacts with the software (online
subsystems), and the auxiliary susbsystems (offline subsystems). We outline
the function of each subsystem and provide a diagram for illustration.

The stakeholders are,
\begin{compactitem}
\item The user,
\item the developer.
\end{compactitem}

\subsection{Online subsystems}
We decompose the software into subsystems.
The software comprises the following online subsystems,
\begin{compactitem}
\item The \textbf{\gls{GUI}} displays the \gls{UI} screens to the user. It
  allows the user to input information into the software and interact.
\item The \textbf{Core logic} handles the manipulation of timesheet data for
  presentation to the user. It decouples the GUI subsystem from the rest
  of the application.
\item The \textbf{Settings manager} loads the user configuration file, and
  propgates the settings to the application.
\item The \textbf{Exporter} is in charge of generating the export file
  for timesheet data.
\item The \textbf{\gls{DB} backend} interacts with the \gls{DB}. It decouples
  the database operation from the rest of the application.
\item The \textbf{Backup manager} creates and restores backup files for the
  current application state.
\item The \textbf{Logger} records the application logs.
\end{compactitem}

\cref{fig:arch-diagram} illustrates the interaction between these
subsystems.


The data files involves in the operation of the software are,
\begin{compactitem}
\item The \textbf{User configuration file} allows the user to configure the
  application. This is a text file the user edits.
\item The \textbf{User state file} records some of the application states
  for ergonomy and interaction. It does not contain any critical data such
  as timesheet data.
\item The \textbf{Logs} record application events for debugging purposes.
\item The \textbf{Database} allows the application to store and retrieve
  critical data. It is persistent.
\item The \textbf{Export file} is a user-readable file generated on-demand
  from the timesheet data.
\item The \textbf{Backup} is an archive recording the critical application
  data, for saving the data elsewhere, or transfering between systems.
  It allows restoring the database.
\end{compactitem}

\begin{figure}
  \includesvg[width=\textwidth]{images/architecture-diagram.svg}
  \caption{\label{fig:arch-diagram} System architecture diagram.}
\end{figure}

\subsection{Offline subsystems}
We outline the high-level build process.

The software comprises the following offline subsystems,
\begin{compactitem}
\item Versioning system
\item Distribution system
\item Build system
\item Testing
\item Documentation generator
\item Documentation distributor
\item Signature system
\item Bug report
\end{compactitem}

FIG of the interactions.