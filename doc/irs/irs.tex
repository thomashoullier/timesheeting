%% Document-wide settings.
\documentclass[letterpaper]{article}
\title{External timesheet format}
\author{Thomas HOULLIER \href{mailto:pro@houllier.net}
         {\texttt{\textlangle pro@houllier.net\textrangle}}}

\usepackage[colorlinks=true, allcolors=blue,
            hyperfootnotes=false,
            pdfauthor={Thomas HOULLIER},
            pdftitle={External timesheet format}]
            {hyperref} % Links for ref/cite.

\input{preamble}

% Document
\begin{document}
\frenchspacing
\date{PRJ1-IRS1-v1.x -- XXX}
\maketitle
\thispagestyle{FirstPage}

\begin{abstract}
  This document describes the external timesheet format used by the
  timesheeting program to export timesheet data.
\end{abstract}

\begin{versionhistory}
  \vhEntry{1.0}{26OCT2024}{TH}{Creation}
  \vhEntry{1.1}{02NOV2024}{TH}
  {Added: IRS1-HED-075, IRS1-HED-085, IRS1-HED-150, IRS1-BDY-090, IRS1-BDY-100.
   Modified: IRS1-HED-150.
   Example file updated accordingly.}
 \vhEntry{1.2}{XXX}{TH}{
   Modified: IRS1-HED-080 changed fixed-size to variable-size string.
   Example file updated accordingly.}
\end{versionhistory}
\setcounter{table}{0} % Reset the table counter.

\section*{Reference documents}
{ \centering
  \begin{tabularx}{\textwidth}{| c | X | c | c | X |} \hline
    Index & Title & Reference & Revision & Author \\ \hline
    RD1 & timesheeting specification document & PRJ1-SPE1 & v1.0 & Thomas
    HOULLIER \\ \hline
    RD2 & List of tz database time zones &
    \cite{wiki:tz_list} & 1246415064 & Wikipedia \\ \hline
  \end{tabularx} \par }

\section*{Document distribution}
The present document is distributed under the \emph{Creative Commons Attribution
  4.0 International} license (\url{https://creativecommons.org/licenses/by/4.0/})
by its author Thomas HOULLIER.

Every document release is signed with the author's GPG key. A signature file
is provided along with the released document.

\tableofcontents
\printglossary[type=\acronymtype,style=index]
\pagestyle{plain}
\section{Introduction}
\subsection{Context}
We answer to the project requirement R-DEX-010 [AD1] by providing a format
for timesheet data export to a file.

The format is meant to be interoperable, \textit{ie} it is meant to be
easily usable with a wide selection of external programs. The external
programs typically targetted are spreadsheet programs (\emph{Libreoffice Calc},
\emph{Gnumeric}), \gls{CLI} text programs (\emph{less}, \emph{vi}),
and Python libraries such as \emph{pandas}.

We prioritize ease of use and readability for the exported format over
compactness and efficiency.

\subsection{Document structure}
The document is structured as follows. First, definitions are given
(\cref{sec:definitions}), then the requirements for the exported file
are listed (\cref{sec:requirements}) and finally an example of a compliant
export file is given (\cref{sec:example}).

\section{Definitions}
We define the concepts used within the project.

\paragraph{The software} The product answering the present requirements.
\paragraph{User} The person using the \emph{software}.
\paragraph{Work unit} The elementary real-life description of what the
  \emph{user} needs to log in the \emph{timesheets}. For instance, this could be
  a list of \emph{project}, \emph{task}, and \emph{start/stop dates}.
\paragraph{Company} The company the \emph{user} works at, for which the
  \emph{tasks} are accomplished.
\paragraph{Project} The project the \emph{user} works on when logging
  \emph{work units}.
\paragraph{Task} The particular element of work being done on a given
  \emph{project}, according to a project-wise subdivision.
\paragraph{Location} The place where a given \emph{work unit} is performed
  by the \emph{user}.
\paragraph{Hierarchy items} The category of items which are either a
  \emph{project}, a \emph{task} or a \emph{location}.
\paragraph{Active hierarchy items} The \emph{hierarchy items} which are
  visible to the \emph{user} when inputting \emph{work units} data into the
  software.
\paragraph{Inactive hierarchy items} The \emph{hierarchy items} which are
  not available to the \emph{user} when inputting new \emph{work units}
  data into the software.
\paragraph{Start/Stop dates} The dates at which a given \emph{work unit} is
  started and at which it is stopped, respectively.
\paragraph{Duration} The time difference between the \emph{start} and the
  \emph{stop dates}, in the context of a \emph{work unit}.
\paragraph{Entry} The data representation of a \emph{work unit}.
\paragraph{Time period} A continuous set of dates defined by a beginning
  date and an end date.
\paragraph{Timesheet} A collection of \emph{entries} in a given \emph{time
  period}.
\paragraph{\gls{UI} screen} A self-standing \gls{GUI} view presented to the
\emph{user} by the \emph{software}, for instance a tab.
\paragraph{Save profile} A segregated \emph{user} identity for managing save data.

\section{Requirements}
\subsection{UHI -- User hierarchy interaction}
\paragraph{R-UHI-010 -- Adding hierarchy items}

\paragraph{Removing hierarchy items}

\paragraph{Editing hierarchy items}

\subsection{UEI -- User entries interaction}
\paragraph{Adding entries}
\paragraph{Adding entries through stopwatch}
\paragraph{Adding entries manually}
\paragraph{Removing entries}
\paragraph{Editing entries}

\subsection{UGL -- User graphical layout}
\paragraph{UI screens breakdown}
The software shall present to the user the following UI screens:

\begin{itemize}
\item Daily entries
\item Hierarchy items
\item Project totals
\item Weekly report
\item Export tool
\end{itemize}

\paragraph{Daily entries}

\paragraph{Running daily total}

\paragraph{Clock in use}

\paragraph{Running clock time}

\paragraph{Hierarchy items}

\subsection{GUI -- Graphical user interface}
\paragraph{Entry metadata prefill}

\paragraph{Entry metadata suggestion}

\paragraph{Entry metadata hierarchy search}

\paragraph{Entry metadata hierarchy coherence}

\subsection{LDC -- Logged data content}
\paragraph{Entry dates}

\paragraph{Entry metadata}

\paragraph{Company identification}

\paragraph{Company metadata}

\paragraph{Project identification}

\paragraph{Project metadata}

\paragraph{Task identification}

\paragraph{Task metadata}

\subsection{TIM -- Time management}

\paragraph{Time resolution}

\paragraph{Time reference}
TODO: UTC as logged time reference, to maintain chronology.

\subsection{DTM -- Data management}
\paragraph{Save}
\paragraph{Transparent save}
\paragraph{Backup}

\paragraph{Backup restore}

\paragraph{Backup completeness}
The whole data state of the software shall be saved and restored using the
backup. Logs are not included in the data state.

\paragraph{Backup conciseness}
The backup archive shall consist in a single file.

\paragraph{Backup timestamp}

\paragraph{Backup user naming}

\subsection{DEX -- Data export}
\paragraph{Timesheet export}

\paragraph{Export naming}

\subsection{ACC -- Accessibility}
\paragraph{Single user}
\paragraph{Synchronization across systems}
\paragraph{Company segregation}
\paragraph{Data confidentiality}

\subsection{ENV -- Environment}
\paragraph{Target hardware}
\paragraph{Target OS}
\paragraph{Target graphical environment}

\subsection{PER -- Performance}
\paragraph{Memory footprint}

\subsection{URE -- User reports}
\paragraph{Durations display format}

\subsection{RPT -- Report: Project totals}
\paragraph{Project totals}

\subsection{RWT -- Report: Weekly totals}
\paragraph{Weekly report}
\paragraph{Weekly report daily totals}
\paragraph{Weekly report weekly totals}
\paragraph{Weekly report weekly totals}
\paragraph{Weekly totals timesheet export}

\subsection{LOG -- Logging}
TODO: log user actions in a separate log file.

\subsection{QUA -- Quality}
\paragraph{Version report}
\paragraph{Saved data validation}
\section{Example file} \label{sec:example}
We provide an example of compliant file format in \cref{lst:example}.

\begin{minipage}{\linewidth}
\begin{lstlisting}[caption={Compliant exported timesheet file.},
                   label={lst:example}]
# Export date: 26Oct2024 14:52:28 +0200
# Export start date: 01Jan2024 00:00:00 +0100
# Export stop date: 31Dec2024 23:59:59 +0100
# Header timezone: Europe/Paris
# timesheeting version: 3.26
# timesheeting DB version: 23
Entry ID, Project ID, Project name, Task ID, Task name, Location ID, Location name, Start date, Stop date
4, 1, Project1, 3, Task3, 1, Location1, 1729952454, 1729953654
8, 1, Project1, 11, Task11, 1, Location1, 1729953659, 1729953789
9, 15, Project15, 5, Task5, 3, Location3, 1729953888, 1729953988
\end{lstlisting} \end{minipage}


\appendix

%% \include{appendices}

\apptocmd{\thebibliography}{\raggedright}{}{}
\begingroup
\setstretch{0.6}
\setlength\bibitemsep{0pt}
\printbibliography
\endgroup
\end{document}
