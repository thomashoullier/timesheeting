%% Document-wide settings.
\documentclass[letterpaper]{article}
\title{External timesheet format}
\author{Thomas HOULLIER \href{mailto:pro@houllier.net}
         {\texttt{\textlangle pro@houllier.net\textrangle}}}

\usepackage[colorlinks=true, allcolors=blue,
            hyperfootnotes=false,
            pdfauthor={Thomas HOULLIER},
            pdftitle={External timesheet format}]
            {hyperref} % Links for ref/cite.

%% Loading packages
\usepackage{amsmath} % For cases in equations.
\usepackage{amsfonts} % For maths sets.
\usepackage{physics} % For \abs{} and \norm{}.
\usepackage[inkscapelatex=false]{svg} %svg graphics
\usepackage{siunitx} % units formatting

\usepackage[backend=biber,style=numeric,citestyle=numeric-comp,maxcitenames=99,dateabbrev=false]{biblatex}
\addbibresource{biblio.bib}
\usepackage{setspace} % Bibliography spacings
\DeclareSourcemap{
  \maps[datatype=bibtex]{
    \map[overwrite]{
      \step[fieldsource=doi, final]
      \step[fieldset=url, null]
      \step[fieldset=eprint, null] }}}
\setcounter{biburllcpenalty}{7000} % break long url in bibliography
\setcounter{biburlucpenalty}{8000}
\renewcommand*{\bibfont}{\footnotesize} % bibliography font size
% Format of biblatex urldate in the bibliography.
\DeclareFieldFormat{urldate}{%
  Visited on \thefield{urlday}\addspace%
  \mkbibmonth{\thefield{urlmonth}}\addspace%
  \thefield{urlyear}\isdot}
\usepackage[ruled,vlined]{algorithm2e} % Algorithms.
\DontPrintSemicolon
\SetKwInOut{Input}{Input}\SetKwInOut{Output}{Output}
\usepackage{mathtools} % Ceiling function.
\usepackage{outlines} % Nest lists.
\usepackage{interval} % Writing intervals.
\usepackage[font={footnotesize,sf}]{caption} %Caption for figures in minipages.
\usepackage{floatrow}
% Figure captions always below. Figures always centered.
\floatsetup[figure]{capposition=bottom,objectset=centering}
\usepackage{wrapfig} %Wrapping figure with text.
\usepackage{stmaryrd} % Double brackets for integers interval.
\usepackage{doi} % Hyperlink DOI
\usepackage{etoolbox} %Ragged right bibliography.
\usepackage{color, colortbl} % Coloring rows in tables.
\usepackage{subcaption} % Subfigures.
\usepackage{pdfpages} % Include PDF pages.
\usepackage{epigraph} % Quotations at beginning of chapters.
\setlength\epigraphwidth{.8\textwidth}
\usepackage[acronym,nonumberlist,nogroupskip,nopostdot]{glossaries} % Glossary for acronyms.
\renewcommand*{\glstextformat}[1]{\textcolor{black}{#1}} % No color on links for abbrev.

\DeclarePairedDelimiter{\ceil}{\lceil}{\rceil} % Ceiling function.
\DeclarePairedDelimiter{\floor}{\lfloor}{\rfloor} % Floor function.

\DeclareMathOperator*{\argmin}{argmin}

\setcounter{tocdepth}{3} % Table of content depth
\setcounter{secnumdepth}{3} % Section numbering depth

% Non-breaking around footnotes.
\makeatletter
\let\Footnote\footnote
\def\pst@@killglue{\unskip\ifdim\lastskip>\z@\expandafter\pst@@killglue\fi}
\def\footnote{\pst@@killglue\Footnote}
\makeatother

% More space below equations
\appto\normalsize{\belowdisplayshortskip=\belowdisplayskip}

% Rewrite month codes in bibliography
\DeclareSourcemap{
  \maps[datatype=bibtex]{
    \map[overwrite]{
      \step[fieldsource=month, match=\regexp{\A(j|J)an(uary)?\Z}, replace=1]
      \step[fieldsource=month, match=\regexp{\A(f|F)eb(ruary)?\Z}, replace=2]
      \step[fieldsource=month, match=\regexp{\A(m|M)ar(ch)?\Z}, replace=3]
      \step[fieldsource=month, match=\regexp{\A(a|A)pr(il)?\Z}, replace=4]
      \step[fieldsource=month, match=\regexp{\A(m|M)ay\Z}, replace=5]
      \step[fieldsource=month, match=\regexp{\A(j|J)un(e)?\Z}, replace=6]
      \step[fieldsource=month, match=\regexp{\A(j|J)ul(y)?\Z}, replace=7]
      \step[fieldsource=month, match=\regexp{\A(a|A)ug(ust)?\Z}, replace=8]
      \step[fieldsource=month, match=\regexp{\A(s|S)ep(tember)?\Z}, replace=9]
      \step[fieldsource=month, match=\regexp{\A(o|O)ct(ober)?\Z}, replace=10]
      \step[fieldsource=month, match=\regexp{\A(n|N)ov(ember)?\Z}, replace=11]
      \step[fieldsource=month, match=\regexp{\A(d|D)ec(ember)?\Z}, replace=12]}}}

% Footnotes marker color
\renewcommand\thefootnote{\textcolor{blue}{\arabic{footnote}}}

\pdfsuppresswarningpagegroup=1 % Silence warnings about pagegroups for figures.
\pdfminorversion=6 % PDF version 1.6 since we include articles in 1.6.

% Allow an extra pass to fix overfull hboxes by allowing more whitespace.
\emergencystretch=1em

% Page numbering and copyright notice.
\usepackage{fancyhdr}
\usepackage{lastpage}

\fancypagestyle{FirstPage}{
\fancyhf{} % Clear footer.
\rfoot{\thepage \hspace{1pt} of \pageref*{LastPage}}
\renewcommand{\headrulewidth}{0pt} % Remove rule at top of page
\lfoot{\href{https://creativecommons.org/licenses/by/4.0/}
       {\includesvg[inkscapelatex=false,height=14pt]{images/ccby.svg}}}}

\fancypagestyle{plain}{
\fancyhf{} % Clear footer.
\rfoot{\thepage \hspace{1pt} of \pageref*{LastPage}}
\renewcommand{\headrulewidth}{0pt} % Remove rule at top of page
}

% Version history
\usepackage{vhistory}

% Keywords
\providecommand{\keywords}[1]{\textbf{Keywords --} #1}

% Glossary
\makeglossaries
\loadglsentries{../glossary/glossary.tex}

\usepackage{fontawesome} %inline icons
\usepackage{xcolor}
\usepackage{listings} % Code listings
\definecolor{codeback}{rgb}{0.99,0.99,0.98}
\definecolor{codecomment}{HTML}{0588fc}
\definecolor{codekeyword}{HTML}{af5f00}
\definecolor{codestring}{HTML}{ffa07a}
\lstdefinestyle{mystyle}{
  backgroundcolor=\color{codeback},
  commentstyle=\color{codecomment},
  keywordstyle=\color{codekeyword},
  stringstyle=\color{codestring},
  basicstyle=\ttfamily\footnotesize,
  breakatwhitespace=false,         
  breaklines=true,                 
  captionpos=b,                    
  keepspaces=true,                 
  numbers=left,                    
  numbersep=5pt,                  
  showspaces=false,                
  showstringspaces=false,
  showtabs=false,                  
  tabsize=2}
\lstset{style=mystyle}

\usepackage[capitalise,nameinlink]{cleveref} % Include eg. "Fig." in front of figures.
\crefname{algorithm}{Alg.}{Algs.}
\crefname{table}{Tab.}{Tabs.}
\crefname{equation}{Eq}{Eqs.}
% Equation cross-references.
%\creflabelformat{equation}{#2#1#3}
\crefformat{equation}{(#2Eq.\thinspace#1#3)}

% No parentheses in equation labels.
%\newtagform{noparen}{}{}
%\usetagform{noparen}

% Document
\begin{document}
\frenchspacing
\date{PRJ1-IRS1-v1.x -- XXX}
\maketitle
\thispagestyle{FirstPage}

\begin{abstract}
  This document describes the external timesheet format used by the
  timesheeting program to export timesheet data.
\end{abstract}

\begin{versionhistory}
  \vhEntry{1.0}{26OCT2024}{TH}{Creation}
  \vhEntry{1.1}{02NOV2024}{TH}
  {Added: IRS1-HED-075, IRS1-HED-085, IRS1-HED-150, IRS1-BDY-090, IRS1-BDY-100.
   Modified: IRS1-HED-150.
   Example file updated accordingly.}
 \vhEntry{1.2}{XXX}{TH}{
   Modified: IRS1-HED-080 changed fixed-size to variable-size string,
   IRS1-HED-090 added the UTC offset on dates.
   Example file updated accordingly.}
\end{versionhistory}
\setcounter{table}{0} % Reset the table counter.

\section*{Reference documents}
{ \centering
  \begin{tabularx}{\textwidth}{| c | X | c | c | X |} \hline
    Index & Title & Reference & Revision & Author \\ \hline
    RD1 & timesheeting specification document & PRJ1-SPE1 & v1.0 & Thomas
    HOULLIER \\ \hline
    RD2 & List of tz database time zones &
    \cite{wiki:tz_list} & 1246415064 & Wikipedia \\ \hline
  \end{tabularx} \par }

\section*{Document distribution}
The present document is distributed under the \emph{Creative Commons Attribution
  4.0 International} license (\url{https://creativecommons.org/licenses/by/4.0/})
by its author Thomas HOULLIER.

Every document release is signed with the author's GPG key. A signature file
is provided along with the released document.

\tableofcontents
\printglossary[type=\acronymtype,style=index]
\pagestyle{plain}
\section{Introduction}
\subsection{Context}
In support of the project requirement R-DEX-010 [RD1], we provide a format
for timesheet data export to a file.

The format is meant to be interoperable, \textit{ie} it is meant to be
easily usable with a wide selection of external programs. The external
programs typically targetted are spreadsheet programs (\emph{Libreoffice Calc},
\emph{Gnumeric}), \gls{CLI} text programs (\emph{less}, \emph{vi}),
and Python libraries such as \emph{pandas}.

We prioritize ease of use and readability for the exported format over
compactness and efficiency.

\subsection{Document structure}
The document is structured as follows. First, definitions are given
(\cref{sec:definitions}), then the requirements for the exported file
are listed (\cref{sec:requirements}) and finally an example of a compliant
export file is given (\cref{sec:example}).

\section{Definitions}
We define the concepts used within the project.

\begin{itemize}
\item \textbf{The software}: The product answering the present requirements.
\item \textbf{User}: The person using the \emph{software}.
\item \textbf{Work unit}: The elementary real-life description of what the
  \emph{user} needs to log in the \emph{timesheets}. For instance, this could be
  a list of \emph{project}, \emph{task}, and \emph{start/stop dates}.
\item \textbf{Company}: The company the user works at, for which the
  \emph{tasks} are accomplished.
\item \textbf{Project}: The project the \emph{user} works on when logging
  \emph{work units}.
\item \textbf{Task}: The particular element of work being done on a given
  \emph{project}, according to a project-wise subdivision.
\item \textbf{Hierarchy items}: The category of items which are either a
  \emph{project} or a \emph{task}.
\item \textbf{Active hierarchy items}: The \emph{hierarchy items} which are
  visible to the \emph{user} when inputting \emph{work units} data into the
  software.
\item \textbf{Start/Stop dates}: The dates at which a given \emph{work unit} is
  started and at which it is stopped, respectively.
\item \textbf{Duration}: The time difference between the \emph{start} and the
  \emph{stop dates}.
\item \textbf{Stopwatch}: A tool for tracking the \emph{start and stop dates} of
  a \emph{work unit}.
\item \textbf{Location}: The place where a given \emph{work unit} is performed
  by the \emph{user}.
\item \textbf{Entry}: The data representation of a \emph{work unit}.
\item \textbf{Time period}: A continuous set of dates defined by a beginning
  date and an end date.
\item \textbf{Timesheet}: A collection of entries in a given time period.
\end{itemize}
\section{Requirements} \label{sec:requirements}
The requirements for the timesheet exported file format are listed here.
They are split into general, header and body sections.

\subsection{General}
The general requirements apply to the \emph{file} as a whole.

\paragraph{IRS1-GEN-010 -- File type}
The \emph{file} shall be a text file.

\paragraph{IRS1-GEN-020 -- File extension}
The \emph{file} extension shall be \lstinline{.csv}.

\paragraph{IRS1-GEN-030 -- File encoding}
The \emph{file} shall be encoded in \lstinline{UTF-8}.

\paragraph{IRS1-GEN-040 -- Line endings}
The \emph{file} shall use UNIX line endings (LF).

\paragraph{IRS1-GEN-050 -- File ending}
The \emph{file} shall end with a newline character.

\paragraph{IRS1-GEN-060 -- File structure}
The \emph{file} shall contain
\begin{itemize}
\item a header,
\item a body.
\end{itemize}

\paragraph{IRS1-GEN-070 -- Header location}
The header of the \emph{file} shall be a contiguous set of text lines
at the beginning of the \emph{file}.

\paragraph{IRS1-GEN-080 -- Body location}
The body of the \emph{file} shall be a contiguous set of text lines
immediately following the header lines (\textit{ie} without gap or empty lines
in between).

\subsection{Header}
The header requirements apply to the header part of the \emph{file}.

\paragraph{IRS1-HED-010 -- Header format}
Every line in the header shall begin with a \lstinline{#} character
followed by a whitespace.

\paragraph{IRS1-HED-020 -- Header export date}
The header shall contain the date at which the \emph{file} was generated.

Note the time reference is the local system clock, as is the case in the
rest of the program.

\paragraph{IRS1-HED-030 -- Period start date}
The header shall contain the start date for the time period specified
during the export.

\paragraph{IRS1-HED-040 -- Period stop date}
The header shall contain the stop date for the time period specified
during the export.

\paragraph{IRS1-HED-050 -- Header dates timezone}
The dates in the header shall be expressed in the timezone set
in the program at the time of export.

\paragraph{IRS1-HED-060 -- Header timezone}
The timezone used to generate the dates in the header shall be indicated
with a TZ identifier string (\textit{eg} \lstinline{Europe/Paris}).
The list of current TZ identifiers may be found at [RD2].

\paragraph{IRS1-HED-070 -- Header program version}
The version of the program at the time of export shall be written
in the header.

\paragraph{IRS1-HED-075 -- Header database version}
The \gls{DB} version of the program at the time of export shall
be written in the header.

\paragraph{IRS1-HED-080 -- Program version string}
The program version (R-REL-010 [AD1]) shall be indicated by a string
\lstinline{X.Y}. With,
\begin{itemize}
\item \lstinline{X}: The major program version (\textit{eg} \lstinline{3}),
\item \lstinline{Y}: The minor program version (\textit{eg} \lstinline{26}).
\end{itemize}

A dot character separates the major version and minor version.
Note the major and minor program version strings have a variable size.

\paragraph{IRS1-HED-085 -- Database version string}
The program \gls{DB} version shall be indicated by an integer string
of variable length. An example is \lstinline{23}.

\paragraph{IRS1-HED-090 -- Header date format}
The dates in the header shall conform to the format
\lstinline{DDMMMYYYY HH:MM:SS ZZZZZ}.
Where,
\begin{itemize}
\item \lstinline{DD} is the number of the day of the month
  (\textit{eg} \lstinline{09}),
\item \lstinline{MMM} is the abbreviated month name, with first letter
  capitalized and remaining letters in lower case (\textit{eg} \lstinline{Oct}),
\item \lstinline{YYYY} is the year number (\textit{eg} \lstinline{2024}),
\item \lstinline{HH} is the hours number (\textit{eg} \lstinline{07}),
\item \lstinline{MM} is the minutes number (\textit{eg} \lstinline{39}),
\item \lstinline{ZZZZZ} is the timezone offset from \gls{UTC}.
\end{itemize}

A \lstinline{:} separator is present between hours and minutes, and between
minutes and seconds. The timezone offset is prefixed by a \lstinline{+} sign
or a \lstinline{-} sign.

Note the corresponding \lstinline{strftime} \cite{cpp:strftime} format string
for the date format is \lstinline{%d%b%Y %H:%M:%S %z}.

\paragraph{IRS1-HED-100 -- Export date format}
The export date shall be written on a single text line beginning with
\lstinline{Export date: } followed by a whitespace and the export date
formatted according to IRS1-HED-090.

An example line is,
\begin{lstlisting}[numbers=none]
  # Export date: 26Oct2024 14:52:28 +0200
\end{lstlisting}

\paragraph{IRS1-HED-110 -- Period start date format}
The export period start date shall be written on a single text line beginning
with \lstinline{Export start date:} followed by a whitespace and the period
start date formatted according to IRS1-HED-090.

An example line is,
\begin{lstlisting}[numbers=none]
  # Export start date: 01Jan2024 00:00:00 +0100
\end{lstlisting}

\paragraph{IRS1-HED-120 -- Period stop date format}
The export period stop date shall be written on a single text line beginning
with \lstinline{Export stop date:} followed by a whitespace and the period
stop date formatted according to IRS1-HED-090.

An example line is,
\begin{lstlisting}[numbers=none]
  # Export stop date: 31Dec2024 23:59:59 +0100
\end{lstlisting}

\paragraph{IRS1-HED-130 -- Header timezone format}
The header timezone shall be written on a single text line
beginning with \lstinline{Header timezone:} followed by a whitespace
and the TZ identifier string (IRS1-HED-060).

An example line is,
\begin{lstlisting}[numbers=none]
  # Header timezone: Europe/Paris
\end{lstlisting}

\paragraph{IRS1-HED-140 -- Program version format}
The program version shall be written in the header on a single text line
beginning with \lstinline{timesheeting version:} followed by a whitespace
and the program version string (IRS1-HED-080).

An example line is,
\begin{lstlisting}[numbers=none]
  # timesheeting version: 3.26
\end{lstlisting}

\paragraph{IRS1-HED-150 -- Header ordering}
The elements of the header shall be ordered as follows,
\begin{enumerate}
\item Export date (IRS1-HED-020),
\item Period start date (IRS1-HED-030),
\item Period stop date (IRS1-HED-040),
\item Header timezone (IRS1-HED-060),
\item Program version (IRS1-HED-070),
\item Program \gls{DB} version (IRS1-HED-075).
\end{enumerate}

\subsection{Body}
\paragraph{IRS1-BDY-010 -- CSV format}
The body of the \emph{file} shall be in \gls{CSV} format.

\paragraph{IRS1-BDY-020 -- CSV delimiter}
The delimiter character used by the \gls{CSV} format shall be a comma followed
by a whitespace (\lstinline{, }).

\paragraph{IRS1-BDY-030 -- Body structure}
The body of the \emph{file} shall contain,
\begin{itemize}
\item A line of column names at the top,
\item Timesheet data entries.
\end{itemize}

\paragraph{IRS1-BDY-040 -- Column list}
The body \gls{CSV} columns shall be, in order,
\begin{enumerate}
\item Entry ID,
\item Project ID,
\item Project name,
\item Task ID,
\item Task name,
\item Location ID,
\item Location name,
\item Start date,
\item Stop date.
\end{enumerate}

\paragraph{IRS1-BDY-050 -- No empty fields}
The timesheet entries in the \emph{file} body shall not contain any empty
fields.

\paragraph{IRS1-BDY-060 -- Id format}
The ID fields in timesheet entries shall be represented as a number string
of variable length.

\paragraph{IRS1-BDY-070 -- Timesheet date format}
The date fields in timesheet entries shall be represented as a \gls{UTC}
UNIX timestamp in seconds.

\paragraph{IRS1-BDY-080 -- Timesheet name format}
The name fields in timesheet entries shall be represented as a string
which may contain whitespace.

\paragraph{IRS1-BDY-090 -- Timesheet entries ordering}
The timesheet entries in the \emph{file} body shall be ordered by
increasing entry start date.

\paragraph{IRS1-BDY-100 -- Timesheet entries period}
The timesheet entries in the \emph{file} body shall have a
start date chronologically contained within the period specified
by the header start date and stop date.

\section{Example file} \label{sec:example}
We provide an example of compliant file format in \cref{lst:example}.

\begin{minipage}{\linewidth}
\begin{lstlisting}[caption={Compliant exported timesheet file.},
                   label={lst:example}]
# Export date: 26Oct2024 14:52:28 +0200
# Export start date: 01Jan2024 00:00:00 +0100
# Export stop date: 31Dec2024 23:59:59 +0100
# Header timezone: Europe/Paris
# timesheeting version: 3.26
# timesheeting DB version: 23
Entry ID, Project ID, Project name, Task ID, Task name, Location ID, Location name, Start date, Stop date
4, 1, Project1, 3, Task3, 1, Location1, 1729952454, 1729953654
8, 1, Project1, 11, Task11, 1, Location1, 1729953659, 1729953789
9, 15, Project15, 5, Task5, 3, Location3, 1729953888, 1729953988
\end{lstlisting} \end{minipage}


\appendix

%% \include{appendices}

\apptocmd{\thebibliography}{\raggedright}{}{}
\begingroup
\setstretch{0.6}
\setlength\bibitemsep{0pt}
\printbibliography
\endgroup
\end{document}
