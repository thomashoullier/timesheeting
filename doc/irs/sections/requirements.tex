\section{Requirements} \label{sec:requirements}
The requirements for the timesheet exported file format are listed here.
They are split into general, header and body sections.

\subsection{General}
The general requirements apply to the \emph{file} as a whole.

\paragraph{IRS1-GEN-010 -- File type}
The \emph{file} shall be a text file.

\paragraph{IRS1-GEN-020 -- File extension}
The \emph{file} extension shall be \lstinline{.csv}.

\paragraph{IRS1-GEN-030 -- File encoding}
The \emph{file} shall be encoded in \lstinline{UTF-8}.

\paragraph{IRS1-GEN-040 -- Line endings}
The \emph{file} shall use UNIX line endings (LF).

\paragraph{IRS1-GEN-050 -- File ending}
The \emph{file} shall end with a newline character.

\paragraph{IRS1-GEN-060 -- File structure}
The \emph{file} shall contain
\begin{itemize}
\item a header,
\item a body.
\end{itemize}

\paragraph{IRS1-GEN-070 -- Header location}
The header of the \emph{file} shall be a contiguous set of text lines
at the beginning of the \emph{file}.

\paragraph{IRS1-GEN-080 -- Body location}
The body of the \emph{file} shall be a contiguous set of text lines
immediately following the header lines (\textit{ie} without gap or empty lines
in between).

\subsection{Header}
The header requirements apply to the header part of the \emph{file}.

\paragraph{IRS1-HED-010 -- Header format}
Every line in the header shall begin with a \lstinline{#} character
followed by a whitespace.

\paragraph{IRS1-HED-020 -- Header export date}
The header shall contain the date at which the \emph{file} was generated.

Note the time reference is the local system clock, as is the case in the
rest of the program.

\paragraph{IRS1-HED-030 -- Period start date}
The header shall contain the start date for the time period specified
during the export.

\paragraph{IRS1-HED-040 -- Period stop date}
The header shall contain the stop date for the time period specified
during the export.

\paragraph{IRS1-HED-050 -- Header dates timezone}
The dates in the header shall be expressed in the timezone set
in the program at the time of export.

\paragraph{IRS1-HED-060 -- Header timezone}
The timezone used to generate the dates in the header shall be indicated
with a TZ identifier string (\textit{eg} \lstinline{Europe/Paris}).
The list of current TZ identifiers may be found at [RD2].

\paragraph{IRS1-HED-070 -- Header program version}
The version of the program at the time of export shall be written
in the header.

\paragraph{IRS1-HED-075 -- Header database version}
The \gls{DB} version of the program at the time of export shall
be written in the header.

\paragraph{IRS1-HED-080 -- Program version string}
The program version (R-REL-010 [AD1]) shall be indicated by a string
\lstinline{X.Y}. With,
\begin{itemize}
\item \lstinline{X}: The major program version (\textit{eg} \lstinline{3}),
\item \lstinline{Y}: The minor program version (\textit{eg} \lstinline{26}).
\end{itemize}

A dot character separates the major version and minor version.
Note the major and minor program version strings have a variable size.

\paragraph{IRS1-HED-085 -- Database version string}
The program \gls{DB} version shall be indicated by an integer string
of variable length. An example is \lstinline{23}.

\paragraph{IRS1-HED-090 -- Header date format}
The dates in the header shall conform to the format
\lstinline{DDMMMYYYY HH:MM:SS ZZZZZ}.
Where,
\begin{itemize}
\item \lstinline{DD} is the number of the day of the month
  (\textit{eg} \lstinline{09}),
\item \lstinline{MMM} is the abbreviated month name, with first letter
  capitalized and remaining letters in lower case (\textit{eg} \lstinline{Oct}),
\item \lstinline{YYYY} is the year number (\textit{eg} \lstinline{2024}),
\item \lstinline{HH} is the hours number (\textit{eg} \lstinline{07}),
\item \lstinline{MM} is the minutes number (\textit{eg} \lstinline{39}),
\item \lstinline{ZZZZZ} is the timezone offset from \gls{UTC}.
\end{itemize}

A \lstinline{:} separator is present between hours and minutes, and between
minutes and seconds. The timezone offset is prefixed by a \lstinline{+} sign
or a \lstinline{-} sign.

Note the corresponding \lstinline{strftime} \cite{cpp:strftime} format string
for the date format is \lstinline{%d%b%Y %H:%M:%S %z}.

\paragraph{IRS1-HED-100 -- Export date format}
The export date shall be written on a single text line beginning with
\lstinline{Export date: } followed by a whitespace and the export date
formatted according to IRS1-HED-090.

An example line is,
\begin{lstlisting}[numbers=none]
  # Export date: 26Oct2024 14:52:28 +0200
\end{lstlisting}

\paragraph{IRS1-HED-110 -- Period start date format}
The export period start date shall be written on a single text line beginning
with \lstinline{Export start date:} followed by a whitespace and the period
start date formatted according to IRS1-HED-090.

An example line is,
\begin{lstlisting}[numbers=none]
  # Export start date: 01Jan2024 00:00:00 +0100
\end{lstlisting}

\paragraph{IRS1-HED-120 -- Period stop date format}
The export period stop date shall be written on a single text line beginning
with \lstinline{Export stop date:} followed by a whitespace and the period
stop date formatted according to IRS1-HED-090.

An example line is,
\begin{lstlisting}[numbers=none]
  # Export stop date: 31Dec2024 23:59:59 +0100
\end{lstlisting}

\paragraph{IRS1-HED-130 -- Header timezone format}
The header timezone shall be written on a single text line
beginning with \lstinline{Header timezone:} followed by a whitespace
and the TZ identifier string (IRS1-HED-060).

An example line is,
\begin{lstlisting}[numbers=none]
  # Header timezone: Europe/Paris
\end{lstlisting}

\paragraph{IRS1-HED-140 -- Program version format}
The program version shall be written in the header on a single text line
beginning with \lstinline{timesheeting version:} followed by a whitespace
and the program version string (IRS1-HED-080).

An example line is,
\begin{lstlisting}[numbers=none]
  # timesheeting version: 3.26
\end{lstlisting}

\paragraph{IRS1-HED-150 -- Header ordering}
The elements of the header shall be ordered as follows,
\begin{enumerate}
\item Export date (IRS1-HED-020),
\item Period start date (IRS1-HED-030),
\item Period stop date (IRS1-HED-040),
\item Header timezone (IRS1-HED-060),
\item Program version (IRS1-HED-070),
\item Program \gls{DB} version (IRS1-HED-075).
\end{enumerate}

\subsection{Body}
\paragraph{IRS1-BDY-010 -- CSV format}
The body of the \emph{file} shall be in \gls{CSV} format.

\paragraph{IRS1-BDY-020 -- CSV delimiter}
The delimiter character used by the \gls{CSV} format shall be a comma followed
by a whitespace (\lstinline{, }).

\paragraph{IRS1-BDY-030 -- Body structure}
The body of the \emph{file} shall contain,
\begin{itemize}
\item A line of column names at the top,
\item Timesheet data entries.
\end{itemize}

\paragraph{IRS1-BDY-040 -- Column list}
The body \gls{CSV} columns shall be, in order,
\begin{enumerate}
\item Entry ID,
\item Project ID,
\item Project name,
\item Task ID,
\item Task name,
\item Location ID,
\item Location name,
\item Start date,
\item Stop date.
\end{enumerate}

\paragraph{IRS1-BDY-050 -- No empty fields}
The timesheet entries in the \emph{file} body shall not contain any empty
fields.

\paragraph{IRS1-BDY-060 -- Id format}
The ID fields in timesheet entries shall be represented as a number string
of variable length.

\paragraph{IRS1-BDY-070 -- Timesheet date format}
The date fields in timesheet entries shall be represented as a \gls{UTC}
UNIX timestamp in seconds.

\paragraph{IRS1-BDY-080 -- Timesheet name format}
The name fields in timesheet entries shall be represented as a string
which may contain whitespace.

\paragraph{IRS1-BDY-090 -- Timesheet entries ordering}
The timesheet entries in the \emph{file} body shall be ordered by
increasing entry start date.

\paragraph{IRS1-BDY-100 -- Timesheet entries period}
The timesheet entries in the \emph{file} body shall have a
start date chronologically contained within the period specified
by the header start date and stop date.
